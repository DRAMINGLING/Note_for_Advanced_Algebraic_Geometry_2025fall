\documentclass{ctexart}
\usepackage[a4paper,left=3cm,right=3cm,top=3cm,bottom=3cm]{geometry}
\usepackage{amssymb,amsfonts,amsmath,amsthm}
\usepackage{anyfontsize}
\usepackage{graphicx}
\usepackage{tikz-cd}
\usepackage{bm}
\usepackage{bbm}
\usepackage{mathrsfs}
\usepackage[colorlinks = true,      % 启用彩色链接而非带框链接
            linkcolor = blue,       % 内部链接(如目录、交叉引用)颜色
            urlcolor = blue,        % 外部链接颜色
            citecolor = blue]{hyperref} % 参考文献链接颜色

\newtheorem{definition}{Definition}[section]
\newtheorem{remark}{Remark}[section]
\newtheorem{theorem}{Theorem}[section]
\newtheorem{proposition}{Proposition}[section]
\newtheorem{setting}{Setting}[section]
\newtheorem{lemma}{Lemma}[section]
\newtheorem{reason}{Reason}[section]
\newtheorem{example}{Example}[section]
\newtheorem{corollary}{Corollary}[section]
\newtheorem{exercise}{Exercise}[section]

\renewcommand{\proofname}{\textit{Proof}}
\renewcommand{\contentsname}{Contents}
\renewcommand{\refname}{References}
\renewcommand{\hom}{\operatorname{Hom}}

\newcommand{\leftquotes}{``}
\newcommand{\rightquotes}{''}

\newcommand{\mapsfrom}{\mathrel{\reflectbox{\ensuremath{\mapsto}}}}
\newcommand{\longmapsfrom}{\mathrel{\reflectbox{\ensuremath{\longmapsto}}}}
\newcommand{\groupaction}{\mathrel{\text{\scalebox{1.5}{\rotatebox[origin=c]{90}{$\circlearrowright$}}}}}

\newcommand{\rank}{\operatorname{rank}}
\newcommand{\im}{\operatorname{im}}
\newcommand{\identity}{\operatorname{id}}
\newcommand{\trdeg}{\operatorname{trdeg}}
\newcommand{\codim}{\operatorname{codim}}
\newcommand{\depth}{\operatorname{depth}}
\newcommand{\height}{\operatorname{height}}
\newcommand{\projdim}{\dim_{\operatorname{proj}}}
\newcommand{\characteristic}{\operatorname{Char}}

\newcommand{\derivation}{\operatorname{Der}}
\newcommand{\ext}{\operatorname{Ext}}
\newcommand{\mor}{\operatorname{Mor}}
\newcommand{\aut}{\operatorname{Aut}}
\newcommand{\homend}{\operatorname{End}}

\newcommand{\sheafhom}{\mathcal{H}\kern -.5pt om}
\newcommand{\sheaftor}{\mathcal{T}\kern -.5pt or}
\newcommand{\sheafext}{\mathcal{E}\kern -.5pt xt}
\newcommand{\sheafend}{\mathcal{E}\kern -.5pt nd}

\newcommand{\coker}{\operatorname{coker}}
\newcommand{\coim}{\operatorname{coim}}
\newcommand{\cone}{\operatorname{cone}}
\newcommand{\totalcomplex}{\operatorname{Tot}}

\newcommand{\spec}{\operatorname{Spec}}
\newcommand{\proj}{\operatorname{Proj}}
\newcommand{\spf}{\operatorname{Spf}}
\newcommand{\fraction}{\operatorname{Frac}}
\newcommand{\singular}{\operatorname{Sing}}
\newcommand{\support}{\operatorname{Supp}}

\newcommand{\reduction}{\operatorname{red}}
\newcommand{\normalization}{\operatorname{nor}}
\newcommand{\regular}{\operatorname{reg}}
\newcommand{\smooth}{\operatorname{sm}}
\newcommand{\analytification}{\operatorname{an}}

\newcommand{\inverselimit}{\underset{\longleftarrow}{\lim}} 
\newcommand{\colimit}{\underset{\longrightarrow}{\lim}}
\newcommand{\derivedotimes}{\overset{\llcorner}{\otimes}}

\newcommand{\grassmannianscheme}[2]{\operatorname{Gr}(#1, #2)}
\newcommand{\grassmanmianfunctor}[2]{\mathcal{G}\kern -.5pt r(#1, #2)}
\newcommand{\hilbertscheme}{\operatorname{Hilb}}
\newcommand{\hilbertfunctor}{\mathcal{H}\kern -.5pt ilb}
\newcommand{\picardscheme}{\operatorname{Pic}}
\newcommand{\picardfunctor}{\mathcal{P}\kern -.5pt ic}
\newcommand{\isomorphismscheme}{\operatorname{Isom}}
\newcommand{\isomorphismfunctor}{\mathcal{I}\kern -.5pt som}

\newcommand{\abeliangroups}{\mathbf{Ab}}
\newcommand{\rings}{\mathbf{Ring}}
\newcommand{\commutativerings}{\mathbf{CRing}}

\newcommand{\module}{\operatorname{Mod}}
\newcommand{\algebra}{\operatorname{Alg}}
\newcommand{\scheme}{\operatorname{Sch}}

\newcommand{\matrices}[2]{\operatorname{Mat}_{#1}(#2)}
\newcommand{\generallineargroup}[2]{\operatorname{GL}_{#1}(#2)}
\newcommand{\speciallineargroup}[2]{\operatorname{SL}_{#1}(#2)}
\newcommand{\projectivespeciallineargroup}[2]{\operatorname{PSL}_{#1}(#2)}

\newcommand{\order}{\operatorname{ord}}
\newcommand{\divisor}{\operatorname{div}}

\title{代数几何进阶-2025秋}
\author{}
\date{}

\begin{document}
\maketitle
\newpage

\tableofcontents
\newpage

\section{Deformation Theory}
\label{section:Deformation_Theory}

Let's begin with a classical example.
\begin{example}[\textbf{\emph{27 lines in smooth cubic hypersurface in $\mathbb{P}^{3}$}}]
    Step 1(moduli space):
    Consider the moduli space of $\{\text{lines in } \mathbb{P}^{3}\}$。
    When the base scheme is $\spec\mathbb{C}$, the set is just $\grassmanmianfunctor{2}{4}(\mathbb{C})$.
    Denote $\grassmannianscheme{2}{4}$ by $G$, then for any scheme $S$
    \begin{equation}
        \begin{split}
            Mor(S, G) & = \text{ family of $2$-dim subspaces in $\mathbb{C}^{4}$ parametrized by $S$} \\
            & = \mathcal{E} \overset{f}{\hookrightarrow} \mathcal{O}_{S}^{\oplus 4} \text{ injective as a bundle map, where $\mathcal{E}$ is bundle of rank $2$}
        \end{split}
    \end{equation}
    In addition, the moduli space is also equal to $\hilbertscheme^{P}$ where Hilbert polynomial $P = n + 1$.
    The natural isomorphism $\hilbertfunctor^{P} \overset{\sim}{\longrightarrow} \grassmanmianfunctor{2}{4}$ can be given as following
    \begin{equation}
        \begin{aligned}
            "\longrightarrow": &
            \begin{tikzcd}
                Z \arrow[r, hookrightarrow] \arrow[dr] &
                \mathbb{P}^{3} \times S \arrow[d, "\pi"] \\
                & S
            \end{tikzcd}
            \text{ gives }
            \begin{tikzcd}
                \pi_{\ast}(\mathcal{O}_{\mathbb{P}_{S}^{3}}(1)) \arrow[r] \arrow[d, equal] &
                \pi_{\ast}(\mathcal{O}_{\mathbb{P}_{S}^{3}}(1)\big{|}_{Z}) \\
                \mathcal{O}_{S}^{\oplus 4} \arrow[r, twoheadrightarrow] &
                \mathcal{F}
            \end{tikzcd}
            \\
            "\longleftarrow": &
            \mathcal{E} \hookrightarrow \mathcal{O}_{S}^{\oplus 4} \twoheadrightarrow \mathcal{F}
            \text{ gives }
            \begin{tikzcd}
               \proj(Sym\mathcal{O}_{S}^{\oplus 4}) \arrow[r, hookleftarrow] \arrow[d, equal] &
               \proj(Sym\mathcal{F}) \arrow[d, equal] \\
                \mathbb{P}^{3} \times S &
                Z
            \end{tikzcd}
        \end{aligned}
    \end{equation}
    \par
    Step 2(intersection number): 
    Assume $X \subset \mathbb{P}^{3}$ is a cubic hypersurface defined by section $F\in H^{0}(\mathbb{P}^{3}, \mathcal{O}(3))$.
    Consider the following diagram 
    \begin{equation}
        \begin{tikzcd}
            Z \arrow[r, hookrightarrow, "i"] \arrow[dr, "\rho"] &
            \mathbb{P}^{3} \times \hilbertscheme^{P} \arrow[r] \arrow[d] &
            \mathbb{P}^{3} \\
            & \hilbertscheme^{P} &
        \end{tikzcd}
    \end{equation}
    For each point $x\in \hilbertscheme^{P}$, we have that 
    \begin{equation}
        Z_x \subseteq X \iff i^{\ast}F\big{|}_{Z_{x}} = 0 \iff \rho_{\ast}i^{\ast}F \text{ vanishes at $x$ as a section of } \rho_{\ast}i^{\ast}\mathcal{O}(3)
    \end{equation}
    In particular, when $x$ is a $k$-point.
    Then $Z_{x}$ is a line in $\mathbb{P}_{k}^{3}$.
    \par
    Step 3(deformation theory):
    The key is that Zariski tangent space is $0$-dim everywhere if and only if 1st order deformations are trivial.
    Recall that for scheme $M$ and $k$-point $x\in M(k)$, the Zariski tangent space is defined as following
    \begin{equation}
        T_{x}M :=
        \left\{
            \begin{tikzcd}
                \spec(k[t]/(t^2)) \arrow[rr] &&
                M \\
                & \spec k \arrow[ul, "t \mapsto 0"] \arrow[ur, "x"] &
            \end{tikzcd}
        \right\}
    \end{equation}
    It is clear that the set of 1st order deformations of $L$ defined below is one-to-one corresponding to $T_{[L]}\mathcal{H}_{X}$,
    where $\mathcal{H}_{X}$ is the moduli space of lines in $X$ and $[L]$ is the $k$-point corresponding to $L$.
    In fact, by step 2, $\mathcal{H}_{X} = (\rho_{\ast}i^{\ast}F)^{-1}(0) \subset \hilbertscheme^{P}$.
    And the correspondence can be precisely written as following
    \begin{equation}
        \begin{tikzcd}
            L' \arrow[r, hookrightarrow] \arrow[d, "flat"] &
            \mathbb{P}^{3} \times \spec(k[t]/(t^{2})) \arrow[dl] \\
            \spec(k[t]/(t^{2})) &
        \end{tikzcd}
        \leftrightsquigarrow 
        \spec(k[t]/(t^{2})) \longrightarrow \mathcal{H}_{X}
    \end{equation}
    and 
    \begin{equation}
        \begin{tikzcd}
            L \arrow[r, hookrightarrow] \arrow[d] &
            \mathbb{P}^{3} \arrow[dl] \\
            \spec k
        \end{tikzcd}
        \leftrightsquigarrow
        \spec k \longrightarrow \mathcal{H}_{X}
    \end{equation}
    The commutativity in 1st order deformation is equivalent to the commutativity in Zariski tangent space.
\end{example}
\begin{definition}
    Given line $L \subset X$, where $X \subset \mathbb{P}^{3}$ is a cubic hypersurface.
    A 1st order deformation of $L$ to $L'$ is a commutative diagram
    \begin{equation}
        \begin{tikzcd}
            L \arrow[r] \arrow[d] &
            L' \arrow[r, hookrightarrow] \arrow[d, "flat"] &
            \mathbb{P}^{3} \times \spec(k[t]/(t^2)) \arrow[dl] \\
            \spec k \arrow[r] &
            \spec(k[t]/(t^2)) &
        \end{tikzcd}
    \end{equation}
\end{definition}

\subsection{1st deformation of closed subschemes}
\label{subsection:Deformation_Theory_1st_deformation_of_closed_subschemes}

From now on, we would denote $k[t]/(t^{2})$ by $k[\varepsilon]$, called dual numbers.
Fix scheme $X$ and closed subscheme $Z \subset X$.
\begin{definition}
    A 1st order deformation of $Z$ to $Z'$ is a commutative diagram
    \begin{equation}
        \begin{tikzcd}
            Z \arrow[r] \arrow[d] &
            Z' \arrow[r, hookrightarrow] \arrow[d, "flat"] &
            X \times \spec k[\varepsilon] \arrow[dl] \\
            \spec k \arrow[r] &
            \spec k[\varepsilon] &
        \end{tikzcd}
    \end{equation}
\end{definition}
\begin{proposition}
    Let $0 \rightarrow J \rightarrow A' \rightarrow A \rightarrow 0$ be an exact sequence,
    where $J \subseteq A'$ is an ideal such that $J^{2} = 0$.
    Then for any $A'$-module $M'$, $M'$ is flat over $A'$ if and only if $M := M' \otimes_{A'} A$ is flat over $A$ and $J \otimes_{A} M = J \otimes_{A'} M' \hookrightarrow M'$ is injective.
    \label{Proposition 1.1}
\end{proposition}
\begin{theorem}
    There is a one-to-one correspondence between 1st deformations of $Z$ in $X$ and $H^{0}(Z, \mathcal{N}_{Z/X})$,
    where $\mathcal{N}_{Z/X} := \sheafhom_{\mathcal{O}_{Z}}(\mathcal{I}_{Z/X}/\mathcal{I}_{Z/X}^{2}, \mathcal{O}_{Z})$ is the normal sheaf.
    \label{Theorem 1.1}
\end{theorem}
\begin{proof}
    Reduce to affine case, assume that $X = \spec A$ and $Z = \spec(A/I)$.
    Denote $A' = A \otimes_{k} k[\varepsilon]$.
    Then a 1st deformation of $Z$ can be expressed by following conditions
    \begin{itemize}
        \item $I' \subset A'$ ideal 
        \item $A'/I' \otimes_{k[\varepsilon]} k \overset{\sim}{\longrightarrow} A/I$
        \item $A'/I'$ flat over $k[\varepsilon]$
    \end{itemize}
    By Proposition \ref{Proposition 1.1}, we can rewrite the conditions
    \begin{itemize}
        \item $I' \subset A'$ ideal
        \item $A'/I' \otimes_{k[\varepsilon]} k \overset{\sim}{\longrightarrow} A/I$
        \item $A'/I' \text{ flat over } k[t]$ under map $t \longmapsto \varepsilon$
        \item $A/I \overset{\cdot \varepsilon}{\longrightarrow} A'/I'$ injective
    \end{itemize}
    inducing a commutative diagram with exact rows and columns
    \begin{equation}
        \begin{tikzcd}
            & 0 \arrow[d] &
            0 \arrow[d] &
            0 \arrow[d] & \\
            0 \arrow[r] &
            I \arrow[r, "\cdot \varepsilon"] \arrow[d] &
            I' \arrow[r] \arrow[d] &
            I \arrow[r] \arrow[d] &
            0 \\
            0 \arrow[r] &
            A \arrow[r, "\cdot \varepsilon"] \arrow[d] &
            A' \arrow[r] \arrow[d] &
            A \arrow[r] \arrow[d] &
            0 \\
            0 \arrow[r] &
            A/I \arrow[r, "\cdot \varepsilon"] \arrow[d] &
            A'/I'\arrow[r] \arrow[d] &
            A/I \arrow[r] \arrow[d] &
            0 \\
            & 0 &
            0 &
            0 &
        \end{tikzcd}
    \end{equation}
    By chasing diagram, we get an $A$-module homomorphism $\theta: I \longrightarrow A/I$.
    In fact, the correspondence between the diagram and $\theta$ is bijective.
    Thus the set of 1st deformations is just
    \begin{equation}
        \begin{split}
            \hom_{A}(I, A/I) & = \hom_{A/I}(I/I^{2}, A/I) \\
            & = \hom_{\mathcal{O}_{Z}}(\mathcal{I}_{Z/X}/\mathcal{I}_{Z/X}^{2}, \mathcal{O}_{Z}) \\
            & = \Gamma(Z, \mathcal{N}_{Z/X})
        \end{split}
    \end{equation}
\end{proof}

\subsection{1st deformation of vector bundles}
\label{subsection:Deformation_Theory_1st_deformation_of_vector_bundles}

Fix scheme $X$ and vector bundle $\mathcal{E}$ on $X$.
Set $X' = X \times_{k} k[\varepsilon]$.
\begin{definition}
    A 1st deformation of $\mathcal{E}$ is a vector bundle $\mathcal{E}'$ on $X'$, 
    flat over $\spec k[\varepsilon]$, with isomorphism $\mathcal{E}'\big{|}_{X} \overset{\sim}{\rightarrow} \mathcal{E}$.
\end{definition}
Take affine open covering $\mathcal{U} = \{U_{i} = \spec A_{i}\}$ trivializing $\mathcal{E}$.
For any 1st deformation $\mathcal{E}'$, as $\mathcal{E}'\big{|}_{X}$ is isomorphic to $\mathcal{E}$, $\mathcal{E}'\big{|}_{U_{i}}$ are also trivial.
Denote $\mathcal{E}_{0}'$ to be the trivial deformation.
Given $\mathcal{E}'$, there are isomorphisms $\varphi_{i}: \mathcal{E}'\big{|}_{U_{i}} \rightarrow \mathcal{E}_{0}'\big{|}_{U_{i}}$.
Then for any $i, j$, $\varphi_{i}^{-1}\big{|}_{U_{i} \cap U_{j}} \circ \varphi_{j}\big{|}_{U_{i} \cap U_{j}}\in Aut(\mathcal{E}'\big{|}_{U_{i} \cap U_{j}})$.
Thus to figure out what the set of 1st deformations likes, we would firstly consider automorphisms of deformation.
\par
Still consider locally, assume $X = \spec A$ and $X' = \spec A'$, where $A' = A \otimes_{k} k[\varepsilon]$.
Note that there is an exact sequence
\begin{equation}
    0 \longrightarrow (\varepsilon) \longrightarrow k[\varepsilon] \longrightarrow k \longrightarrow 0
\end{equation}
Tensor by $M'$ over $k[\varepsilon]$, get exact sequence
\begin{equation}
    0 \longrightarrow M \overset{\cdot \varepsilon}{\longrightarrow} M' \longrightarrow M \longrightarrow 0
\end{equation}
For an automorphism $\varphi$ of $\mathcal{E}'$, consider the following commutative diagram
\begin{equation}
    \begin{tikzcd}
        0 \arrow[r] &
        M \arrow[r, "\cdot \varepsilon"] \arrow[d, equal] &
        M' \arrow[r] \arrow[d, "\varphi"] &
        M \arrow[r] \arrow[d, equal] &
        0 \\
        0 \arrow[r] &
        M \arrow[r, "\cdot \varepsilon"] &
        M' \arrow[r] &
        M \arrow[r] &
        0
    \end{tikzcd}
\end{equation}
inducing a $A$-module homomorphism $\theta: M \rightarrow M$, given by $a \mapsto \varphi(a') - a'$, 
where $a'$ is a lift of $a$ along $M' \rightarrow M$.
In fact, $\varphi \leftrightsquigarrow \theta$ is bijective.
Thus the set of automorphisms of $\mathcal{E}'$ as deformation is just $\hom_{A}(M, M)$.
Globalized, get $H^{0}(X, \sheafend_{\mathcal{O}_{X}}(\mathcal{E}))$.
\par
Back to deformation, for $\varphi_{ij} := \varphi_{i}^{-1} \circ \varphi_{j}\big{|}_{U_{ij}}\in Aut(\mathcal{E}'\big{|}_{U_{i} \cap U_{j}}) = H^{0}(U_{i} \cap U_{j}, \sheafend_{\mathcal{O}_{X}}(\mathcal{E}))$.
Assume that $\varphi_{ij}$ corresponds to $\theta_{ij}$.
Then 
\begin{itemize}
    \item $\varphi_{ij} \circ \varphi_{jk} = \varphi_{ik} \leftrightsquigarrow \theta_{ij} + \theta_{jk} = \theta_{ik}$
    \item $\text{modify } \varphi_{i} = \varphi_{i}' \circ \psi_{i} \leftrightsquigarrow \theta_{ij} = \theta_{ij}' + \rho_{i} - \rho_{j}$
\end{itemize}
Conclude that 1st order deformations of $\mathcal{E}$ in $X$ is $H^{1}(X, \sheafend_{\mathcal{O}_{X}}(\mathcal{E}))$.
\begin{remark}
    If generalize vector bundles to coherent sheaves, the answer is $\ext_{\mathcal{O}_{X}}^{1}(\mathcal{E}, \mathcal{E})$ since each element in $\ext_{\mathcal{O}_{X}}^{1}(\mathcal{E}, \mathcal{E})$ gives an extension
    \begin{eqnarray}
        0 \longrightarrow \mathcal{E} \longrightarrow \mathcal{E}' \longrightarrow \mathcal{E} \longrightarrow 0
    \end{eqnarray}
\end{remark}

\subsection{1st order deformation of affine schemes smooth over k}
\label{subsection:Deformation_Theory_1st_order_deformation_of_affine_schemes_smooth__k}

\begin{lemma}
    Let $\spec A \rightarrow \spec k$ be a smooth morphism and $\spec B \rightarrow \spec A$ be any $k$-morphism.
    Given extension of $k$-algebras
    \begin{equation}
        0 \longrightarrow J \longrightarrow B' \longrightarrow B \longrightarrow 0, \text{ where } J^{2} = 0
    \end{equation}
    such that the following diagram commutes
    \begin{equation}
        \begin{tikzcd}
            \spec B \arrow[r] \arrow[d, hookrightarrow] &
            \spec A \arrow[d] \\
            \spec B' \arrow[ur, dashrightarrow] \arrow[r] &
            \spec k
        \end{tikzcd}
    \end{equation}
    Then there exists morphism $\spec B' \rightarrow \spec A$ making the diagram commutative.
    \label{Lemma 1.1}
\end{lemma}
\begin{proof}
    Choose $P = k[x_{1}, \cdots, x_{n}] \twoheadrightarrow A$,
    there is a commutative diagram with exact rows 
    \begin{equation}
        \begin{tikzcd}
            0 \arrow[r] &
            I \arrow[r] \arrow[d, "\rho"] &
            P \arrow[r] \arrow[d, "\varphi"] &
            A \arrow[r] \arrow[d, "\phi"] &
            0 \\
            0 \arrow[r] &
            J \arrow[r] &
            B' \arrow[r] &
            B \arrow[r] &
            0  
        \end{tikzcd}
    \end{equation}
    where $\varphi$ exists since $P$ is free over $k$ and hence projective.
    By diagram chasing, we find that $\varphi$ gives $P$-module structure on $J$ and $\phi$ gives $A$-module structure on both $B$ and $J$, which is independent of choosing of $\varphi$.
    \par
    To get a map $A \rightarrow B'$, want $I \rightarrow J$ to be zero.
    So we need to change $\varphi$.
    For two different $\varphi_{1}$ and $\varphi_{2}$, set $\delta := \varphi_{1} - \varphi_{2}$,
    which is in fact a homomorphism from $P$ to $J$.
    We have that 
    \begin{equation}
        \begin{split}
            \delta(ab) & = \varphi_{1}(a)\varphi_{1}(b) - \varphi_{2}(a)\varphi_{2}(b) \\
            & = \varphi_{1}(a)\varphi_{1}(b) - \varphi_{1}(a)\varphi_{2}(b) + \varphi_{1}(a)\varphi_{2}(b)- \varphi_{2}(a)\varphi_{2}(b) \\
            & = \varphi_{1}(a)\delta(b) - \delta(a)\varphi_{2}(b) \\
            & = a\delta(b) - \delta(a)b
        \end{split}
    \end{equation}
    Thus $\delta$ satisfies Laibniz's Law and hence $\delta\in Der_{k}(P, J)$ is a derivation.
    \par
    As $A$ is smooth over $k$, there is an exact sequence of locally free sheaves 
    \begin{equation}
        0 \longrightarrow I/I^{2} \longrightarrow \Omega_{P/k} \otimes_{P} A \longrightarrow \Omega_{A/k} \longrightarrow 0
    \end{equation}
    Apply $\hom_{A}(\cdot, J)$, get
    \begin{equation}
        \begin{tikzcd}
            \hom_{A}(\Omega_{P/k} \otimes_{P} A, J) \arrow[r, twoheadrightarrow] \arrow[d, equal] &
            \hom_{A}(I/I^{2}, J) \arrow[d, equal] \\
            \hom_{P}(\Omega_{P/k}, J) \arrow[d, equal] &
            \hom_{P}(I, J) \\
            Der_{k}(P, J) &
        \end{tikzcd}
    \end{equation}
    By surjectivity, we can find appropriate $\delta$ such that $I \rightarrow J$ induced by $\varphi + \delta$ is zero map.
\end{proof}
\begin{definition}
    Let $X\in Sch_{k}$.
    A 1st order deformation of $X$ is a fibered diagram
    \begin{equation}
        \begin{tikzcd}
            X \arrow[r, hookrightarrow] \arrow[d] &
            X' \arrow[d, "flat"] \\
            \spec k \arrow[r, hookrightarrow] &
            \spec k[\varepsilon]
        \end{tikzcd}
    \end{equation}
    Two deformations are isomorphic if there is a commutative diagram
    \begin{equation}
        \begin{tikzcd}[sep = small]
            & X \arrow[dr, hookrightarrow] \arrow[dl, hookrightarrow] & \\
            X_{1}' \arrow[rr, "\sim"] \arrow[dr] &&
            X_{2}' \arrow[dl] \\
            & \spec k[\varepsilon] &
        \end{tikzcd}
    \end{equation}
\end{definition}
\begin{remark}
    To describe deformations, in fact we need a strengthened version of Lemma \ref{Lemma 1.1} that for $X \rightarrow Y$ smooth morphism of affine schemes and $B, B'$ same as above,
    if there is a commutative diagram
    \begin{equation}
        \begin{tikzcd}
            \spec B \arrow[r] \arrow[d, hookrightarrow] &
            X \arrow[d] \\
            \spec B' \arrow[ur, dashrightarrow] \arrow[r] &
            Y
        \end{tikzcd}
    \end{equation}
    then there exists $\spec B' \rightarrow X$ making the diagram commutative.
\end{remark}
Claim that deformations are locally trivial.
Assume there is a deformation of $X$ to $X'$.
Locally, $X$ and $X'$ are affine.
Take $X[\varepsilon] = X \times_{k[\varepsilon]} k$.
Then we have a commutative diagram
\begin{equation}
    \begin{tikzcd}
        X \arrow[r, hookrightarrow] \arrow[d, hookrightarrow] &
        X' \arrow[d, "smooth"] \\
        X[\varepsilon] \arrow[ur, dashrightarrow, "u"] \arrow[r] &
        \spec k[\varepsilon]
    \end{tikzcd}
\end{equation}
By strengthened version of Lemma \ref{Lemma 1.1}, there exists $u: X[\varepsilon] \rightarrow X'$ making the diagram commutative.
Assume that $X = \spec A$ and $X' = \spec B$.
Note that $X'$ and $X[\varepsilon]$ are both extension, consider the following commutative diagram
\begin{equation}
    \begin{tikzcd}
        0 \arrow[r] &
        A \arrow[r, "\cdot \varepsilon"] \arrow[d, equal] &
        B \arrow[r] \arrow[d, "u^{\sharp}"] &
        A \arrow[r] \arrow[d, equal] &
        0 \\
        0 \arrow[r] &
        A \arrow[r, "\cdot \varepsilon"] &
        A[\varepsilon] \arrow[r] &
        A \arrow[r] &
        0
    \end{tikzcd}
\end{equation}
By diagram chasing, it is easy to show that $u^{\sharp}$ is isomorphic.
Thus deformations are locally trivial.
\par
Back to general case, now local extension always trivial.
Take affine open covering $\{U_{i}\}$ with local extensions $X_{i}'$.
To glue up a global extension, we need cocycle condition that the following diagram commutes
\begin{equation}
    \begin{tikzcd}
        X_{i}' \arrow[r, "\varphi_{ij}"] \arrow[dr, "\varphi_{ik}" swap] &
        X_{j}' \arrow[d, "\varphi_{jk}"] \\
        & X_{k}'
    \end{tikzcd}
\end{equation}
Thus we should firstly figure out the automorphisms of deformation.
Locally set $X = \spec B$ and $X' = \spec B'$.
We have a commutative diagram with exact rows
\begin{equation}
    \begin{tikzcd}
        0 \arrow[r] &
        I \arrow[r, "\cdot \varepsilon"] \arrow[d, equal] &
        B' \arrow[r] \arrow[d, "\identity" swap] \arrow[d, "\identity + \delta", bend left]&
        B \arrow[r] \arrow[d, equal] &
        0 \\
        0 \arrow[r] &
        I \arrow[r, "\cdot \varepsilon"] &
        B' \arrow[r] &
        B \arrow[r] &
        0
    \end{tikzcd}
\end{equation}
where $\delta\in \hom_{B}(\Omega_{B/k}, B)$ the tangent vector fields.
Thus automorphisms are $H^{0}(X, \mathcal{T}_{X})$
\par
Given deformation of $X$ to $X'$.
Choose affine open covering $\mathcal{U} = \{U_{i}\}$ of $X$.
On each $U_{i}$, choose isomorphism $\varphi_{i}: X'\big{|}_{U_{i}} \rightarrow X[\varepsilon]$.
Then on each $U_{ij}$, we get different isomorphisms
\begin{equation}
    \begin{tikzcd}[row sep = huge]
        X'\big{|}_{U_{ij}} \arrow[rr, "\varphi_{j}" swap, bend right] \arrow[rr, "\varphi_{i}", bend left] &&
        X[\varepsilon]\big{|}_{U_{ij}} 
    \end{tikzcd}
\end{equation}
Their difference $(\delta_{ij})_{i, j}$ satisfying cocycle condition gives an element in $\check{H}^{1}(\mathcal{U}, \mathcal{T}_{X})$.
\par
In conclusion, infinitesimal automorphisms of $X$ is $H^{0}(X, \mathcal{T}_{X})$ and 1st order deformations of $X$ is $H^{1}(X, \mathcal{T}_{X})$.
\begin{remark}
    Given a flat family $\mathfrak{X} \longrightarrow S$ and $k$-point $x\in S$.
    Take fiber $X := \mathfrak{X}_{x}$ smooth over $k$.
    There is a Kodaira-Spencer map from tangent space $T_{x, S} \cong \hom_{X}(\spec k[\varepsilon], S)$ to deformations $H^{1}(X, \mathcal{T}_{X})$.
\end{remark}
\begin{example}
    (1)Let $X = \mathbb{P}^{n}$, then $H^{1}(\mathbb{P}^{n}, \mathcal{T}_{\mathbb{P}^{n}}) = 0$.
    There are no deformations. \\
    (2)Let $C$ be a smooth curve of genus $g$.
    Then $H^{1}(C, \mathcal{T}_{C}) = 3g - 3$ for $g \ge 2$.
\end{example}

\section{Obstruction Theory}
\label{section:Obstruction_Theory}

Let's begin with a bady example.
\begin{example}
    Let $V$ be a vector space of finite dimension and $T: V \rightarrow V$ be a linear map, 
    $W \subseteq V$ $T$-invariant subspace.
    Wonder if there exists $T$-invariant complement.
    Consider the following exact sequence
    \begin{equation}
        0 \longrightarrow W \longrightarrow V \overset{\pi}{\longrightarrow} V/W \longrightarrow 0
    \end{equation}
    Fix $\sigma: V/W \rightarrow V$ such that $\identity_{V/W}$.
    Thus there exists $T$-invariant complement if and only if there exists $\sigma$ such that $\sigma\overline{T} = T\sigma$.
    We say that $\sigma\overline{T} - T\sigma\in \hom_{k}(V/W, W)$ is an obstruction.
    \par
    For two different sections $\sigma$ and $\sigma'$, get difference $\theta = \sigma - \sigma'\in \hom_{k}(V/W, W)$.
    Define $\varphi: \hom_{k}(V/W, W) \rightarrow \hom_{k}(V/W, W)\quad \theta \mapsto \theta\overline{T} - T\theta$.
    Then question change to if there exists $\sigma + \theta$ such that $\sigma + \theta$ is a splitting of $k[T]$-module.
    \begin{equation}
        \begin{split}
            & \iff \sigma\overline{T} - T\sigma\in \im(\varphi) \\
            & \iff [\sigma\overline{T} - T\sigma] = 0 \text{ in } \coker(\varphi)
        \end{split}
    \end{equation}
    which is called a obstruction class.
\end{example}
As its name, obstruction is something stopping us from extension.
For higher order deformation, we globalize the place where we extend to.
Here is an example.
\begin{example}[\textbf{\emph{Extension of maps}}]
    \begin{equation}
        \begin{tikzcd}
            \spec k \arrow[r, "{(0, 0)}"] \arrow[d] &
            V(y^{2} - x^{3}) \subset \mathbb{A}_{k}^{2} \\
            \spec k[\varepsilon] \arrow[ur, "{(at, bt)}" swap] &
        \end{tikzcd}
    \end{equation}
    Each extension is determined by a tangent vector $v = (a, b)$.
    If we moreover extend map to $\spec(k[t]/(t^{3}))$, then
    \begin{equation}
        \begin{tikzcd}
            \spec k \arrow[r, "{(0, 0)}"] \arrow[d] &
            V(y^{2} - x^{3}) \subset \mathbb{A}_{k}^{2} \\
            \spec(k[t]/(t^{3})) \arrow[ur, "{(a_{1}t + a_{2}t^{2}, b_{1}t + b_{2}t^{2})}" swap] &
        \end{tikzcd}
    \end{equation}
    satisfies that $(a_{1}t + a_{2}t^{2})^{3} - (b_{1}t + b_{2}t^{2})^{2} \equiv 0 \pmod {t^{3}}$.
    Thus $b_{1} = 0$ and hence tangent vect should be contained in $x$-axis.
    \par
    Replace $V(y^{2} - X^{3})$ by general $V(g(x, y))$ singular at $(0, 0)$.
    Given extension
    \begin{equation}
        \begin{tikzcd}
            \spec(k[t]/(t^{r})) \arrow[r, "{(x_{r}(t), y_{r}(t))}"] \arrow[d] &
            V(g(x, y)) \subset \mathbb{A}_{k}^{2} \\
            \spec(k[t]/(t^{r + 1})) \arrow[ur, "{(x_{r + 1}(t), y_{r + 1}(t))}" swap] &
        \end{tikzcd}
    \end{equation}
    where $x_{r + 1}(t) = x_{r}(t) + at^{r}$ and $y_{r + 1} = y_{r}(t) + bt^{r}$,
    satisfies that $g(x_{r + 1}(t), y_{r + 1}(t)) \equiv 0 \pmod {t^{r + 1}}$.
    Take Taylor expansion at $(0, 0)$
    \begin{equation}
            g(x_{r + 1}(t) , y_{r + 1}(t)) \equiv g(x_{r}(t), y_{r}(t)) + \frac{\partial g(0, 0)}{\partial x}t^{r} + \frac{\partial g(0, 0)}{\partial y}bt^{r} \pmod {t^{r + 1}}
    \end{equation}
    Hence we need $\frac{\partial g(0, 0)}{\partial x} = \frac{\partial g(0, 0)}{\partial y} = 0$ and $g(x_{r}(t), y_{r}(t)) \equiv 0 \pmod {t^{r + 1}}$.
\end{example}

\subsection{Higher order deformation}
\label{subsection:Obstruction_Theory_higher_order_deformation}

Set $k = \overline{k}$ a algebraically closed field.
Consider $k$-algebras.
\begin{definition}
    An extension of artinian local $k$-algebras with residue field $k$,
    \begin{equation}
        0 \longrightarrow J \longrightarrow A' \longrightarrow A \longrightarrow 0
    \end{equation}
    is called a star extension if $\mathfrak{m}_{A'}J = 0$.
\end{definition}
\begin{remark}
    With $\mathfrak{m}_{A'}J = 0$, $A'$-module structure on $J$ descends to a $k$-vector space structure.
\end{remark}
Usually, our question is that given object over $A$, 
what's the obstruction to lift it to an object over $A'$,
there are how many deformationS and what are automorphisms of extension.

\subsection{Higher order deformation of vector bundles}
\label{subsection:Obstruction_Theory_higher_order_deformation_of_vector_bundles}

\begin{definition}
    Let $X_{0}\in Sch_{k}$ be a $k$-scheme.
    Set $X = X_{0} \times_{k} A$.
    Given $\mathcal{E}$ vector bundle on $X$ and $X'$ extension of $X$ over $A'$.
    An extension of $\mathcal{E}$ is a vector bundle $\mathcal{E}'$ on $X'$ such that $\mathcal{E}'\big{|}_{X} \cong \mathcal{E}$.
\end{definition}
For automorphism, locally consider the commutative diagram with exact rows
\begin{equation}
    \begin{tikzcd}
        0 \arrow[r] &
        \mathcal{E} \otimes_{A} J \arrow[r] \arrow[d, equal] &
        \mathcal{E}' \arrow[r] \arrow[d, "\varphi"] &
        \mathcal{E} \arrow[r] \arrow[d, equal] &
        0 \\
        0 \arrow[r] &
        \mathcal{E} \otimes_{A} J \arrow[r] &
        \mathcal{E}' \arrow[r] &
        \mathcal{E} \arrow[r] &
        0
    \end{tikzcd}
\end{equation}
where $\varphi$ one-to-one corresponds to elements in $\hom_{\mathcal{O}_{X}}(\mathcal{E}, \mathcal{E} \otimes_{A} J)$.
Generalized, the set of automorphisms is $H^{0}(X, \sheafend_{\mathcal{O}_{X}}(\mathcal{E}) \otimes_{A} J)$.
\begin{definition}
    Let $G$ be a group, $S$ a set.
    $S$ is called a $G$-torsor if $S$ is nonempty and $G$ acts on $S$ freely transitively.
\end{definition}
\begin{definition}
    Let $G$ be a group, $S$ a set.
    $S$ is called a $G$-pseudo torsor if $S$ is either a $G$-torsor or empty set.
\end{definition}
For how many deformations, as the argument of 1st order deformation, gluing imformation corresponds to element in $H^{1}(X, \sheafend_{\mathcal{O}_{X}}(\mathcal{E}) \otimes_{A} J)$.
Thus the set of deformations is a $H^{1}(X, \sheafend_{\mathcal{O}_{X}}(\mathcal{E}) \otimes_{A} J)$-pseudo torsor. 
\par
For obstruction, as local extension exists, take affine open covering $\mathcal{U} = \{U_{i}\}$ of $X'$ trivializing $\mathcal{E}$.
Now on each $U_{i}$, we get an extension $\mathcal{E}_{i}'$.
And on each $U_{ij}$, there is isomorphism $\varphi_{ij}: \mathcal{E}_{i}'\big{|}_{U_{ij}} \rightarrow \mathcal{E}_{j}'\big{|}_{U_{ij}}$.
To glue up, we need $\varphi_{ij}$ satisfy cocycle condition.
Set $\rho_{ijk} = \varphi_{ik}^{-1} \circ \varphi_{jk} \circ \varphi_{ij}$, corresponding to $\theta_{ijk}\in H^{0}(U_{ijk}, \sheafend_{\mathcal{O}_{X}}(\mathcal{E}) \otimes_{A} J)$.
Thus we get an element $[\theta]$ in $H^{2}(U_{ij}, \sheafend_{\mathcal{O}_{X}}(\mathcal{E}) \otimes_{A} J)$ independent of choosing of $\mathcal{E}_{i}'$, called the obstruction.
\par
If $[\theta] = [0]$, then there exists $(\delta_{ij})_{i, j}\in \prod_{i, j} \Gamma(U_{ij}, \sheafend_{\mathcal{O}_{X}}(\mathcal{E}) \otimes_{A} J)$ mapping to $[\theta]$.
As $H^{0}(U_{ij}, \sheafend_{\mathcal{O}_{X}}(\mathcal{E}) \otimes_{A} J)$ is the set of automorphisms, each $\delta_{ij}$ gives an automorphism $\psi_{ij}: \mathcal{E}_{j}'\big{|}_{U_{ij}} \rightarrow \mathcal{E}_{j}'\big{|}_{U_{ij}}$.
Consider $\psi_{ij}^{-1} \circ \varphi_{ij}: \mathcal{E}_{i}'\big{|}_{U_{ij}} \rightarrow \mathcal{E}_{j}'\big{|}_{U_{ij}}$.
It is clear that these new maps satisfy cocycle condition so that we can glue up an extension.
\begin{example}
    (1)Let $X_{0}$ be a smooth projective variety, $\mathcal{L}_{0}$ line bundle on $X_{0}$.
    Even though obstruction space $H^{2}(X_{0}, \sheafend_{\mathcal{O}_{X_{0}}}(\mathcal{L}_{0}) \otimes_{k} J) = H^{2}(X_{0}, \mathcal{O}_{X_{0}} \otimes_{k} J)$ needn't be zero,
    higher deformations of $\mathcal{L}$ are unobstructed. \\
    (2)Let $X_{0}$ be a smooth projective variety, $\mathcal{E}_{0}$ vector bundle on $X_{0}$.
    There is a trace map $H^{2}(X_{0}, \sheafend_{\mathcal{O}_{X_{0}}}(\mathcal{E}_{0})) \overset{tr}{\rightarrow} H^{2}(X_{0}, \sheafend_{\mathcal{O}_{X_{0}}}(\det \mathcal{E}_{0}))$ mapping $ob(E)$ to $ob(\det \mathcal{E})$.
\end{example}

\subsection{Higher order deformation of closed subschemes}
\label{subsection:Obstruction_Theory_higher_order_deformation_of_closed_subschemes}

Given $X_{0}, X, X'$ as before.
Let $Y \subset X$ be a closed subscheme and $Y_{0} = Y \times_A k$.
Local extension of $Y$ may not exist in general.
\par
Obstruction is in $H^{1}(Y_{0}, \mathcal{N}_{Y_{0}/X_{0}} \otimes_{k} J)$, 
deformations are a $H^{0}(Y_{0}, \mathcal{N}_{Y_{0}/X_{0}} \otimes_{k} J)$-pseudo torsor and identity map is the only automorphism.
\begin{proposition}
    If $Y$ is a local complete intersection and $X$ is smooth,
    then local extension exists.
    \label{Proposition 2.1}
\end{proposition}
\begin{remark}
    For deformations for local complete intersection, refer to Vistoli's note.
\end{remark}

\subsection{Higher order deformation of abstract smooth schemes}
\label{subsection:Obstruction_Theory_higher_order_deformation_of_abstract_smooth_schemes}

In this case, local extension exists.
\par
Obstruction is in $H^{2}(X_{0}, \mathcal{T}_{X_{0}} \otimes_{k} J)$, 
deformations are a $H^{1}(X_{0}, \mathcal{T}_{X_{0}} \otimes_{k} J)$-pseudo torsor and automorphisms are $H^{0}(X_{0}, \mathcal{T}_{X_{0}} \otimes_{k} J)$.
\begin{definition}
    Let $f: X \rightarrow Y$ be a morphism of schemes.
    We say that $f$ is formally smooth if for any morphism $S = \spec A \hookrightarrow S' = \spec A'$ with ideal $J^{2} = 0$ and commutative diagram
    \begin{equation}
        \begin{tikzcd}
            S \arrow[r] \arrow[d, hookrightarrow] &
            X \arrow[d] \\
            S' \arrow[ur, dashrightarrow] \arrow[r] &
            Y
        \end{tikzcd}
    \end{equation}
    there exists morphism $S' \rightarrow X$ making the diagram commutative.
\end{definition}
There are some facts about formal smoothness.
\begin{proposition}
    (1)smooth $\iff$ of finitely presented and formally smooth. \\
    (2)If $X,Y$ are noetherian, suffices to check for cases that $S'$ is spectrum of some artinian local ring.
    \label{Proposition 2.2}
\end{proposition}
\begin{corollary}
    Let $Y \subset \mathbb{P}^{n}$ be a local complete intersection.
    If $H^{1}(Y, \mathcal{N}_{Y/\mathbb{P}^{n}}) = 0$, then hilbert scheme is smooth at $[Y]$ with fiber of dimension $\dim_{k}H^{0}(Y, \mathcal{N}_{Y/\mathbb{P}^{n}})$.
\end{corollary}

\section{Obstruction Theory of Moduli Space}

\subsection{Obstruction of a local ring}
\label{subsection:Obstruction_Theory_of_Moduli_Space_Obstruction_of_a_local_ring}

Let $S$ be a scheme.
Recall that in a fine moduli space $M$, family over $S$ is equivalent to a morphism $S \rightarrow S$.
\begin{definition}
    Let $M$ be a fine moduli space, $x\in M$ a $k$-point.
    A deformation of object $x$ is a commutative diagram
    \begin{equation}
        \begin{tikzcd}
            \spec A \arrow[r, dashrightarrow] &
            M \\
            \spec k \arrow[u, hookrightarrow, "\mathfrak{m}_{A}"] \arrow[ur, "x" swap] &
        \end{tikzcd}
    \end{equation}
    where $A$ is an artinian local ring with residue field $k$.
\end{definition}
As homomorphism between stalks is local, the deformation diagram factors through $\spec \mathcal{O}_{M, x}$ as following
\begin{equation}
    \begin{tikzcd}
        \spec A \arrow[r, dashrightarrow] &
        \spec \mathcal{O}_{M, x} \arrow[r] &
        M \\
        \spec k \arrow[u, hookrightarrow, "\mathfrak{m}_{A}"] \arrow[ur, "x" swap] &&
    \end{tikzcd}
\end{equation}
Hence deformation of object $x$ is equivalent to deformation of a local ring.
\par
Denote $C := \mathcal{O}_{M, x}$.
For star extension of artinian local rings $0 \rightarrow J \rightarrow A' \rightarrow A \rightarrow 0$,
consider if we can extend $\spec A \rightarrow \spec C$ to $\spec A' \rightarrow \spec C$.
\begin{lemma}
    If $C$ is regular, then for any star extension of $A$ to $A'$, in the following commutative diagram
    \begin{equation}
        \begin{tikzcd}
            \spec A' \arrow[r, dashrightarrow] &
            \spec C \\
            \spec A \arrow[u, hookrightarrow] \arrow[ur] \arrow[r, hookleftarrow] &
            \spec k \arrow[u, hookrightarrow]
        \end{tikzcd}
    \end{equation}
    $\spec A' \rightarrow \spec C$ always exists.
    \label{Lemma 3.1}
\end{lemma}
\begin{remark}
    In this case, we would say there is no obstruction.
\end{remark}
\begin{proof}
    Consider completion $\widehat{C}$ of $C$.
    Since completion of regular local ring is a regular local ring (Atiyah Proposition 11.24), 
    and by Corollary 28.2 in \emph{Commutative Algebra}, Matsumura, $\widehat{C} \cong k[[x_{1}, \cdots, x_{n}]]$ for some $n$.
    \par
    In addition, since $A$ is artinian and $C \rightarrow A$ is local homomorphism,
    $C \rightarrow A$ factors through $C \twoheadrightarrow C/\mathfrak{m}_{C}^{t}$ for large enough $t$.
    Hence the following diagram commutes
    \begin{equation}
        \begin{tikzcd}
            C \arrow[r] \arrow[dr, twoheadrightarrow] &
            A &
            A' \arrow[l, twoheadrightarrow] & \\
            & C/\mathfrak{m}_{C}^{t} \arrow[u] &
            \widehat{C} \arrow[l] \arrow[u, dashrightarrow] \arrow[r, hookleftarrow] &
            k[x_{1}, \cdots, x_{n}] \arrow[ul, dashrightarrow]
        \end{tikzcd}
    \end{equation}
    As $k[x_{1}, \cdots, x_{n}]$ is projective $k$-algebra, there exists $k[x_{1}, \cdots, x_{n}] \rightarrow A'$.
    Moreover, since $A'$ is complete, the map extends to $\widehat{C} \rightarrow A'$.
    As $A'$ is artinian local, factors through some $C/\mathfrak{m}^{r}$ for large enough $r$.
    Take $t = r$ so that we get $C \rightarrow A'$.
\end{proof}
For $C$ not regular, we need to define obstruction theory.
\begin{definition}
    An obstruction theory of $C$ is a vector space and an assignment $\varphi$
    \begin{equation}
        \left.
            \begin{aligned}
                0 \rightarrow J \rightarrow A' \rightarrow A \rightarrow 0 \\
                C \overset{u}{\rightarrow}A \\
                \mathfrak{m}_{A'}J = 0
            \end{aligned}
        \right\}
        \overset{\varphi}{\longmapsto} ob(A', u)\in V \otimes_{k} J
    \end{equation}
    satisfies the following conditions
    \begin{itemize}
        \item $ob(A', u) = 0$ iff $\exists$ lifting of $u$,
        $
        \begin{tikzcd}
            C \arrow[r, "u"] \arrow[dr, dashrightarrow] &
            A \\
            & A' \arrow[u, twoheadrightarrow]
        \end{tikzcd}
        $
        \item Functorial in the sense that $\forall K \subset J$ $k$-vector subspace, in the following commutative diagram
        $$
        \begin{tikzcd}[sep = small]
            0 \arrow[r] &
            J \arrow[r] \arrow[d] &
            A' \arrow[r] \arrow[d] &
            A \arrow[r] \arrow[d] &
            0 \\
            0 \arrow[r] &
            J/K \arrow[r] &
            A'/K \arrow[r] &
            A \arrow[r] &
            0
        \end{tikzcd}
        $$
        $\varphi(u, A') \mapsto \varphi(u, A'/K)$ under $V \otimes J \rightarrow V \otimes J/K$
    \end{itemize}
\end{definition}

\subsection{Canonical obstruction theory}
\label{subsection:Obstruction_Theory_of_Moduli_Space_Canonical_obstruction_theory}

There is also a canonical obstruction theory.
Assume there exists $P \twoheadrightarrow C$, where $P$ is regular local with residue $k$.
Further assume $I := \ker(P \twoheadrightarrow C) \subset \mathfrak{m}_{P}^{2}$.
\begin{equation}
    0 \longrightarrow I \longrightarrow P \longrightarrow C \longrightarrow 0
\end{equation}
Set $V_{C} := (I \otimes_{P} k)^{\vee} = (I/\mathfrak{m}_{P}I)^{\vee}$.
\par
To give assignment $\varphi$, consider the following commutative diagram with exact rows
\begin{equation}
    \begin{tikzcd}
        0 \arrow[r] &
        I \arrow[r] \arrow[d, "f"] &
        P \arrow[r, twoheadrightarrow] \arrow[d, "\widetilde{u}"] &
        C \arrow[r] \arrow[d, "u"] \arrow[dl, dashrightarrow] &
        0 \\
        0 \arrow[r] &
        J \arrow[r] &
        A' \arrow[r, twoheadrightarrow] &
        A \arrow[r] &
        0 
    \end{tikzcd}
\end{equation}
As $P$ is regular local ring, by Lemma \ref{Lemma 3.1}, $\widetilde{u}$ always exist which is a local homomorphism.
As $J^{2} = 0$, $f$ in fact is map $\widetilde{f}: I/\mathfrak{m}_{P}I \rightarrow J$ corresponding to an element $\varphi(A', u)\in (I/\mathfrak{m}_{P})^{\vee} \otimes_{k} J$.
Then $\widetilde{f}$ is zero map if and only if there exists lifting.
\par
For diffreent $\widetilde{u_{1}}, \widetilde{u_{2}}$, set $\theta = \widetilde{u_{1}} - \widetilde{u_{2}}$.
For any $h\in I$, consider $\theta(h) = \sum \theta(\alpha_{i}\beta_{i})$ where $\alpha_{i}, \beta_{i}\in \mathfrak{m}_{P}$.
As $\theta$ is a derivation, $\theta(h) = \sum \alpha_{i}\theta(\beta_{i}) + \theta(\alpha_{i})\beta_{i}$.
While $\theta(\alpha_{i}), \theta(\beta_{i})\in J$, we get $\theta(h) = 0$ so that $f$ is well defined.

\subsection{Geometric point of view}
\label{subsection:Obstruction_Theory_of_Moduli_Space_Geometric_point_of_view}

Let $W$ be a smooth variety over $k$, $E \rightarrow W$ vector bundle.
Assume $s: W \rightarrow E$ is section of $E$.
Set $X = V(s) \subset W$.
For $x\in X$, in the following commutative diagram
\begin{equation}
    \begin{tikzcd}
        \spec A' \arrow[rr, dashrightarrow,"\widetilde{u}"] &&
        W \\
        \spec A \arrow[u, hookrightarrow] \arrow[rr, "u"] \arrow[dr, hookleftarrow] &&
        X \arrow[u, hookrightarrow] \\
        & x \arrow[ur, hookrightarrow] &
    \end{tikzcd}
\end{equation}
$\widetilde{u}$ factors through $X$ if and only if $\widetilde{u}^{\ast}s = 0$ as sections of $\widetilde{u}^{\ast}E$.
Note that there is an exact sequence
\begin{equation}
    0 \longrightarrow \widetilde{J} \longrightarrow \mathcal{O}_{\spec A'} \longrightarrow \mathcal{O}_{\spec A} \longrightarrow 0
\end{equation}
Tensor $\widetilde{u}^{\ast}E$ over $\mathcal{O}_{\spec A'}$, get exact sequence
\begin{equation}
    0 \longrightarrow \widetilde{J} \otimes \widetilde{u}^{\ast}E \longrightarrow \widetilde{u}^{\ast}E \overset{v}{\longrightarrow} \mathcal{O}_{\spec A} \otimes \widetilde{u}^{\ast}E \longrightarrow 0
\end{equation}
As $X = V(s)$, $v \circ \widetilde{u}^{\ast}s$ is zero map.
Hence $\widetilde{u}^{\ast}s$ is in fact in $\Gamma(J \otimes \widetilde{u}^{\ast}E)$.
\begin{lemma}
    On X, we can well define map $TW\big{|}_{X} \overset{\nabla s}{\rightarrow} E\big{|}_{X}$.
    \label{Lemma 3.2}
\end{lemma}
\begin{reason}
    Choose frame $\sigma_{1}, \cdots, \sigma_{k}$ for $E$.
    Write $s = \sum s_{i}\sigma_{i}$.
    Then for $v\in \Gamma(u, Tw)$, $\nabla_{V}(s) = \sum \nabla_{V}(s_{i})\sigma_{i}$.
    Take another frame, for example $g_{1}\sigma_{1}, \cdots, g_{k}\sigma_{k}$.
    Then $s = \sum \frac{s_{i}}{g_{i}}g_{i}\sigma_{i}$ so that
    \begin{equation}
        \begin{split}
            \nabla_{v}(S) & = \sum \nabla_{v}(\frac{s_{i}}{g_{i}})g_{i}\sigma_{i} \\
            & = \sum \nabla_{v}(s_{i})g_{i}^{-1}g_{i}\sigma_{i} + \sum \nabla_{v}(g_{i}^{-1})s_{i}g_{i}\sigma_{i} \\
            & = \sum \nabla_{v}(s_{i})\sigma_{i}
        \end{split}
    \end{equation}
    since $s_{i}$ vanish on $X$.
\end{reason}
\begin{remark}
    Dually, there is a well defined map 
    \begin{equation}
        E^{\vee}\big{|} \longrightarrow \Omega_{W}^{1}\big{|}_{X}\quad \sigma \longmapsto d\langle \sigma, s \rangle
    \end{equation}
    factoring through $I/I^{2}$, where $I = I_{X/W}$.
    And there is a complex
    \begin{equation}
        E^{\vee}\big{|}_{X} \overset{\langle \cdot, s \rangle}{\longrightarrow} I/I^{2} \overset{d}{\longrightarrow} \Omega_{W}^{1}\big{|}_{X} \longrightarrow \Omega_{X}^{1} \longrightarrow 0
    \end{equation}
    When $I \not\subset \mathfrak{m}_{x}^{2}$, $\widetilde{u}^{\ast}s$ in $\Gamma(\widetilde{u}^{\ast}E \otimes J)$ is not well defined.
    In fact, $\widetilde{u}^{\ast}E \otimes J$ is supported on $\spec k(x)$.
    Can directly write $\widetilde{u}^{\ast}E \otimes J$ as a $k(x)$-vector space.
    \par
    For different liftings $\widetilde{u}_{1}, \widetilde{u}_{2}$, $\widetilde{u}_{1} - \widetilde{u}_{2}\in u^{\ast}TW \otimes J$ mapping into $\widetilde{u}^{\ast}E \otimes J$ along $u^{\ast}TW \otimes J \overset{\nabla_{s}\big{|}_{x} \otimes J}{\longrightarrow} \widetilde{u}^{\ast}E \otimes J$.
    So similar as argument in subsection \ref{subsection:Obstruction_Theory_of_Moduli_Space_Canonical_obstruction_theory}, we need $I \subset \mathfrak{m}_{x}^{2}$ to make $\nabla_{S}\big{|}_{x} \otimes J$ a zero map. 
\end{remark}
\begin{example}
    Let $W = \mathbb{A}_{k}^{3}$ with coordinate $x, y, z$, $E = \mathcal{O}_{\mathbb{A}_{k}^{3}}^{\oplus 3}$, $s = (s_{1}, s_{2}, s_{3}) = (xy, yz, xz)$.
    Hence $X = V(s)$ is just union of axes.
    For $TW\big{|}_{X} \cong \mathcal{O}_{X}^{\oplus X}$ and $E\big{|}_{X} \cong \mathcal{O}_{X}^{\oplus 3}$, the map $\nabla_{s}$ has a matrix representation
    $
    \begin{pmatrix}
        y & x & 0 \\
        0 & z & y \\
        z & 0 & x 
    \end{pmatrix}
    $,
    which is of rank $0$ at origin and of rank $2$ elsewhere.
\end{example}

\subsection{Smallestness of canonical obstruction theory}
\label{subsection:Obstruction_Theory_of_Moduli_Space_Smallestness_of_canonical_obstruction_theory}

Now let's show the smallestness of canonical obstruction theory.
Notations are same.
\begin{theorem}
    Given any obstruction theory $(V, \varphi)$.
    There exists canonical injection $V_{C} \hookrightarrow V$ commutting with obstruction class $assignment$.
    In particular, $I$ can be generated by at most $\dim V$ elements.
    \label{Theorem 3.1}
\end{theorem}
\begin{proof}
    Note that $\hom_{k}(V_{C}, V) \cong I/\mathfrak{m}_{P}I \otimes V$.
    Want to find an obstruction class $ob\in V \otimes I/\mathfrak{m}_{P}I$ corresponding to an injection.
    Hence we need a star extension with $J = I/\mathfrak{m}_{P}I$.
    \par
    Firstly consider the exact sequence
    \begin{equation}
        0 \longrightarrow I/\mathfrak{m}_{P}I \longrightarrow P/\mathfrak{m}_{p}I \longrightarrow C \longrightarrow 0
    \end{equation}
    but here both $C$ and $P/\mathfrak{m}_{P}I$ are not necessary artinian local.
    Take quotient of some power $\mathfrak{m}_{P}$, get exact sequence
    \begin{equation}
        0 \longrightarrow I/\mathfrak{m}_{P}I + \mathfrak{m}_{P}^{n} \cap I \cong I + \mathfrak{m}_{P}^{n}/\mathfrak{m}_{P}I + \mathfrak{m}_{P}^{n} \longrightarrow P/\mathfrak{m}_{P}I + \mathfrak{m}_{P}^{n} \longrightarrow C/\mathfrak{m}_{C}^{n} \longrightarrow 0
    \end{equation}
    By Artin-Reez Lemma, for $n$ large enough, $I \cap \mathfrak{m}_{P}^{n} \subseteq \mathfrak{m}_{P}I$ so that $I/\mathfrak{m}_{P}I + \mathfrak{m}_{P}^{n} = I/\mathfrak{m}_{P}I$.
    For each $\alpha\in V_{C}$, there is a commutative diagram with exact rows
    \begin{equation}
        \begin{tikzcd}
            0 \arrow[r] &
            I \arrow[r, hookrightarrow] \arrow[d, twoheadrightarrow] &
            P \arrow[r, twoheadrightarrow] \arrow[d, twoheadrightarrow] &
            C \arrow[r] \arrow[d, twoheadrightarrow, "u"] \arrow[ddl, dashrightarrow, "?" description] \arrow[dl, dashrightarrow, "?" description] &
            0 \\
            0 \arrow[r] &
            I/\mathfrak{m}_{P}I \arrow[r, hookrightarrow] \arrow[d, twoheadrightarrow, "\alpha"] &
            P/\mathfrak{m}_{P}I + \mathfrak{m}_{P}^{n} \arrow[r, twoheadrightarrow] \arrow[d, twoheadrightarrow] &
            C/\mathfrak{m}_{C}^{n} \arrow[r] \arrow[d, equal] &
            0 \\
            0 \arrow[r] &
            k \arrow[r, hookrightarrow] &
            P/\mathfrak{m}_{P}I + \mathfrak{m}_{P}^{n} + \ker(\alpha) \arrow[r, twoheadrightarrow] &
            C/\mathfrak{m}_{C}^{n} \arrow[r] &
            0
        \end{tikzcd}
    \end{equation}
    where all surjections are quotient maps.
    Note that by definition of obstruction theory, $\identity_{V} \otimes \alpha: V \otimes I/\mathfrak{m}_{P}I \rightarrow V \otimes k$ sends $\varphi(u, P/\mathfrak{m}_{P}I) + \mathfrak{m}_{P}^{n}$ to $\varphi(u, P/\mathfrak{m}_{P}I + \mathfrak{m}_{P}^{n} + \ker(\alpha))$.
    \par
    For every $\alpha \neq 0$, by canonical obstruction theory, we know that $u$ cannot by extended to $P/\mathfrak{m}_{P}I + \mathfrak{m}_{P}^{n} + \ker(\alpha)$.
    Hence the corresponding element of $\varphi(u, P/\mathfrak{m}_{P}I + \mathfrak{m}_{P}^{n})$ in $\hom_{k}(V_{C}, V)$ is an injection. 
\end{proof}
\begin{remark}
    Note that $\varphi(u, P/\mathfrak{m}_{P}I + \mathfrak{m}_{P}^{n})$ is independent to choosing of $n$ since for all large enough $n$,
    kernel ideals $J$ of the star extension are all $I/\mathfrak{m}_{P}I$ with identity map of $I/\mathfrak{m}_{P}I$ corresponding to same obstruction class in $V_{C} \otimes I/\mathfrak{m}_{P}I$.
    \par
    In addition, for given obstruction theory $(V, \varphi)$, the functorial property shows that $\varphi(u, P/\mathfrak{m}_{P}I + \mathfrak{m}_{P}^{n})$ is mapping to $\varphi(u, P/\mathfrak{m}_{P}I + \mathfrak{m}_{P}^{n + 1})$ under identity $V \otimes I/\mathfrak{m}_{P}I$.
    Hence the inclusion is canonical.
\end{remark}
\begin{corollary}
    If $C$ has an obstruction theory $(V, \varphi)$, then $\dim C \ge \dim(\mathfrak{m}_{C}/\mathfrak{m}_{C}^{2}) - \dim V$.
    \label{Corollary 3.1}
\end{corollary}
\begin{reason}
    Since $P$ is regular local ring, $\dim P = \dim(\mathfrak{m}_{P}/\mathfrak{m}_{P}^{2}) \ge \dim(\mathfrak{m}_{C}/\mathfrak{m}_{C}^{2})$
    As $\dim C \ge \dim P - \sharp\text{generators of $I$}$, 
    by Theorem \ref{Theorem 3.1}, immediately get $\dim C \ge \dim(\mathfrak{m}_{C}/\mathfrak{m}_{C}^{2}) - \dim V$.
\end{reason}

\subsection{Categories of artinian local rings}
\label{subsection:Obstruction_Theory_of_Moduli_Space_Categories_of_artinian_local_rings}

\begin{example}
    Let $x\in E \subset \mathbb{P}_{k}^{1}$ be a $k$-point, where $E$ is an elliptic curve.
    Note that the stalks of $x$ in $E$ and $\mathbb{P}_{k}^{1}$ can be not isomorphic otherwise their fraction fields are isomorphic,
    inducing a birational map from $\mathbb{P}_{k}^{1}$ to $E$.
    However, the completion of their stalks are isomorphic so that their deformation problems are equivalent.
\end{example}
\begin{definition}
    Let $k = \overline{k}$.
    Denote $\mathcal{C}$ to be the category of artinian local $k$-algebras.
    A functor of artinian local ring is a covariant functor from $\mathcal{C}$ to $Sets$.
\end{definition}
\begin{definition}
    Let $0 \rightarrow J \rightarrow A' \rightarrow A \rightarrow 0$ be a star extension.
    We say that the extension is a small extension if the ideal $J$ is generated by one single element i.e. $J$ is $k$-vector space of dimension $1$.
\end{definition}
\begin{example}
    Let $X_{0}$ be a scheme over $k$.
    Set $\mathcal{F}(A) =$ deformation classes of $X_{0}$ over $A$.
    \begin{equation}
        \begin{tikzcd}
            A_{1} \arrow[r] &
            A_{2} \\
            \spec A_{1} &
            \spec A_{2} \arrow[l]
        \end{tikzcd}
        \overset{\mathcal{F}}{\longrightarrow}
        \begin{tikzcd}
            X_{0} \arrow[rr, dashrightarrow, bend left] \arrow[r] \arrow[d] &
            X_{1} \arrow[d] &
            X_{2} := X_{1} \times_{A_{1}} A_{2} \arrow[d] \arrow[l] \\
            \spec k \arrow[r] \arrow[rr, bend right] &
            \spec A_{2} &
            \spec A_{2} \arrow[l]
        \end{tikzcd}
    \end{equation}
    Called deformation functor of $X_{0}$.
    \label{example:deformation_functor_of_artinian_local_ring}
\end{example}
Let $\widehat{\mathcal{C}}$ be the category of complete local $k$-algebra with residue field $k$.
For $R\in \widehat{\mathcal{C}}$, define contravariant functor $h_{R}: \mathcal{C} \rightarrow Sets$ mapping $A$ to $\hom(R, C) = \mor_{Sch_{k}}(\spec C, \spec R)$.
\begin{definition}
    Let $\mathcal{F}: \mathcal{C} \rightarrow Sets$ be a functor of artinian local ring.
    We say $\mathcal{F}$ is pro-representable if $\mathcal{F} \cong h_{R}$ for some $R\in \widehat{\mathcal{C}}$.
\end{definition} 
\begin{example}
    For $\mathcal{F}$ in Example \ref{example:deformation_functor_of_artinian_local_ring}, if $X_{0} \subset \mathbb{P}_{k}^{N}$ is a closed subscheme of projective space, 
    then $\mathcal{F} \cong h_{R}$ where $R = \widehat{\mathcal{O}_{\hilbertscheme, [X_{0}]}}$.
\end{example}
Note that for a moduli space, we can naturally define a universal family.
But here $R$ needn't be artinian local ring so that we cannot give the universal family by $\identity_{R}\in \hom(R, R)$.
To simulate the universal family of moduli space, we consider $\mathcal{F}(R/\mathfrak{m}^{n})$ for all $n$.
\par
Assume $\mathcal{F} \cong h_{R}$, then for each $n$, $\mathcal{F}(R/\mathfrak{m}^{n}) \cong \hom(R, R/\mathfrak{m}^{n})$.
Hence we can find a $\xi_{n}\in \mathcal{F}(R/\mathfrak{m}^{n})$ corresponding to the quotient map.
There is a sequence
\begin{equation}
    \begin{tikzcd}
        \hom(R, R/\mathfrak{m}) \arrow[d, "\sim"] &
        \hom(R, R/\mathfrak{m}^{2}) \arrow[d, "\sim"] \arrow[l] &
        \hom(R, R/\mathfrak{m}^{3}) \arrow[d, "\sim"] \arrow[l] &
        \cdots \arrow[l] \\
        \mathcal{F}(R/\mathfrak{m}) &
        \mathcal{F}(R/\mathfrak{m}^{2}) \arrow[l] &
        \mathcal{F}(R/\mathfrak{m}^{3}) \arrow[l] &
        \cdots \arrow[l] \\
        \xi_{1} &
        \xi_{2} \arrow[l, mapsto] &
        \xi_{3} \arrow[l, mapsto] &
        \cdots \arrow[l, mapsto]
    \end{tikzcd}
\end{equation}
\par
Given $A\in \mathcal{C}$ and ring homomorphism $f: R \rightarrow A$.
Since $A$ is artinian local, $f$ factors through $g: R/\mathfrak{m}^{n} \rightarrow A$ for some $n$.
Now we can get a family in $\mathcal{F}(A)$ by the following commutative diagram
\begin{equation}
    \begin{tikzcd}[column sep= small]
        \hom(R, R/\mathfrak{m}^{n}) \arrow[rr, "\sim"] \arrow[d, "g \circ"] &&
        \mathcal{F}(R/\mathfrak{m}^{n}) \arrow[d] &
        \xi_{m} \arrow[d, mapsto] \\
        \hom(R, A) \arrow[rr, "\sim"] &&
        \mathcal{F}(A) &
        \mathcal{F}(g)(\xi_{n})
    \end{tikzcd}
\end{equation}
\begin{definition}
    Every functor $\mathcal{F}$ of artinian local ring induces a covariant functor
    \begin{equation}
        \widehat{\mathcal{F}}: \widehat{\mathcal{C}} \longrightarrow Sets\quad R \longmapsto \inverselimit \mathcal{F}(R/\mathfrak{m}^{n})
    \end{equation}
    In particular, element $(\xi_{n})_{n}\in \inverselimit \mathcal{F}(R/\mathfrak{m}^{n})$ is called a formal family.
\end{definition}
For a functor $\mathcal{F}$ of artinian local ring, there are several questions
\begin{itemize}
    \item When pro-representable?
    \item When algebraic?
    \item Existence of global moduli?
\end{itemize}
\begin{example}[\textbf{\emph{No moduli of $\mathbb{P}_{k}^{1}$-bundles}}]
    There does not exist moduli space $M$ such that $\mor(S, M) \cong \{\mathbb{P}_{k}^{1}-\text{bundles over } S\}$.
    \label{example:no_moduli_of_bundles_on_projective_line}
\end{example}
About the first question, there is a fact
\begin{proposition}
    Let $X_{0}$ be a (smooth) projective variety with $\aut{X_{0}} = 0$.
    Then the functor given by deformation classes of $X_{0}$ is pro-representable i.e. there exists $R\in \widehat{\mathcal{C}}$ such that the following commutative diagram commutes
    \begin{equation}
        \begin{tikzcd}
            X_{0} \arrow[r] \arrow[d] &
            X_{1} \arrow[r] \arrow[d] &
            X_{2} \arrow[r] \arrow[d] &
            \cdots \\
            \spec R/\mathfrak{m} \arrow[r] &
            \spec R/\mathfrak{m}^{2} \arrow[r] &
            \spec R/\mathfrak{m}^{3} \arrow[r] &
            \cdots
        \end{tikzcd}
    \end{equation}
    \label{Proposition 3.1}
\end{proposition}
\vspace{-\baselineskip} % 减少一行间距
About the second question, it asks if there is a scheme $X$ such that the following diagram commutes for some $n$
\begin{equation}
    \begin{tikzcd}
        X_{n} \arrow[r] \arrow[d] &
        X \arrow[d] \\
        \spec R/\mathfrak{m}^{n} \arrow[r] &
        \spec R
    \end{tikzcd}
\end{equation}

\section{Formal Family}
\label{section:Formal_Family}

\subsection{Formal family}
\label{subsection:Formal_Family_Formal_family}

Formal moduli is try to study deformations of a point in moduli space.
However, just as Example \ref{example:no_moduli_of_bundles_on_projective_line}, global moduli does not always exist.
So we turn to another much more weaker problem that the existence of formal neighbourhood,
which is also parametrized by something with some kind of "universal property".
\par
Here we begin with a new definition of formal family.
\begin{definition}
    A formal family over $R\in \widehat{\mathcal{C}}$ is a natural transformation from $h_{R}$ to a functor $\mathcal{F}$ of artinian local ring.
\end{definition}
\begin{lemma}
    The two definitions are in fact same since they contain same datum.
    \label{Lemma 4.1}
\end{lemma}
\begin{proof}
    Obviously, given a natural transformation $h_{R} \rightarrow \mathcal{F}$, by definition, we get a sequence $\{\xi_{n}\}$.
    Hence, we only need to show that given an element $(\xi_{n})_{n}\in \inverselimit \mathcal{F}(R/\mathfrak{m}^{n})$,
    we can recover a natural transformation.
    \par
    For any $A\in \mathcal{C}$ and $f\in h_{R}(A)$, since $A$ is artinian local ring, $f$ factors through quotient map $R \twoheadrightarrow R/\mathfrak{m}^{n}$ for some $n$
    \begin{equation}
        f: R \twoheadrightarrow R/\mathfrak{m}^{n} \overset{\widetilde{f}}{\longrightarrow} A
    \end{equation}
    By Yoneda's Lemma, we have that $\mor(h_{R/\mathfrak{m}^{n}}, \mathcal{F}) \cong \mathcal{F}(R/\mathfrak{m}^{n})$.
    Hence $\xi_{n}\in \mathcal{F}(R/\mathfrak{m}^{n})$ corresponds to a natural transformation $\varphi_{n}: h_{R/\mathfrak{m}^{n}} \rightarrow \mathcal{F}$.
    Consider the following diagram
    \begin{equation}
        \begin{tikzcd}[row sep = tiny]
            h_{R}(A) & 
            h_{R/\mathfrak{m}^{n}}(A) \arrow[r, "\varphi_{n}(A)"] \arrow[l, "\circ f"] &
            \mathcal{F}(A) \\
            f &
            \widetilde{f} \arrow[r, mapsto] \arrow[l, mapsto] &
            \mathcal{F}(\widetilde{f})(\xi_{n})
        \end{tikzcd}
    \end{equation}
    It is natural to define a map $h_{R}(A) \rightarrow \mathcal{F}(A)$ by mapping $f$ to $\mathcal{F}(\widetilde{f})(\xi_{n})$.
    Now we remains to show that this map is well defined.
    That is, if $f$ has two factorizations 
    \begin{equation}
        f = R \twoheadrightarrow R/\mathfrak{m}^{n_{1}} \overset{\widetilde{f}_{1}}{\longrightarrow} A = R \twoheadrightarrow R/\mathfrak{m}^{n_{2}} \overset{\widetilde{f}_{2}}{\longrightarrow} A
    \end{equation}
    then $\mathcal{F}(\widetilde{f}_{1})(\xi_{n_{1}})$ should equal to $\mathcal{F}(\widetilde{f}_{2})(\xi_{n_{2}})$.
    \par
    We may assume that $n_{1} \ge n_{2}$.
    Since $(\xi_{n})_{n}\in \inverselimit \mathcal{F}(R/\mathfrak{m}^{n})$, we have that $\xi_{n_{1}}$ is mapping to $\xi_{n_{2}}$ under $\mathcal{F}(R/\mathfrak{m}^{n_{1}}) \rightarrow \mathcal{F}(R/\mathfrak{m}^{n_{2}})$ so that
    \begin{equation}
        \varphi_{n_{2}}: h_{R/\mathfrak{m}^{n_{2}}} \longrightarrow \mathcal{F} = h_{R/\mathfrak{m}^{n_{2}}} \longrightarrow h_{R/\mathfrak{m}^{n_{1}}} \overset{\varphi_{n_{1}}}{\longrightarrow} \mathcal{F}
    \end{equation} 
    In particular, we get
    \begin{equation}
        \begin{tikzcd}[row sep = tiny]
            h_{R/\mathfrak{m}^{n_{2}}}(A) \arrow[r] &
            h_{R/\mathfrak{m}^{n_{1}}}(A) \arrow[r, "\varphi_{n_{1}}(A)"] &
            \mathcal{F}(A) \\
            \widetilde{f}_{2} \arrow[r, mapsto] &
            \widetilde{f}_{1} \arrow[r, mapsto] &
            \mathcal{F}(\widetilde{f}_{1})(\xi_{n_{1}}) = \mathcal{F}(\widetilde{f}_{2})(\xi_{n_{2}})
        \end{tikzcd}
    \end{equation}
    Thus $(\zeta_{n})_{n}$ successfully recover all datum of the original natural transformation, done!
\end{proof}
\begin{remark}
    In addition, there is corollary of Yoneda's Lemma
    \begin{equation}
        \mor(h_{R}, \mathcal{F}) \overset{\sim}{\longrightarrow} \widehat{\mathcal{F}}(R)
    \end{equation}
\end{remark}
\begin{definition}
    Given a formal family $\varphi: h_{R} \rightarrow \mathcal{F}$.
    We say that $\varphi$ is \\
    $\bullet$ versal (or say formally smooth), if $\varphi$ is strongly surjective i.e. for all $A\in \mathcal{C}$,
    $\varphi(A)$ is surjective and for all surjection $B \twoheadrightarrow A$,
    $h_{R}(B) \rightarrow h_{R}(A) \times_{\mathcal{F}(A)} \mathcal{F}(B)$ is surjective. \\
    $\bullet$ miniversal, if $\varphi$ is versal and $\varphi(k[\varepsilon])$ is bijective. \\
    $\bullet$ universal, if $\varphi$ is a natural isomorphism of functors.
\end{definition}
\begin{remark}
    The map $h_{R}(B) \rightarrow h_{R}(A) \times_{\mathcal{F}(A)} \mathcal{F}(B)$ comes from the commutative diagram of natural transformation,
    and the strong surjectivity can be interpreted as a lifting diagram that for all $(\theta, \eta)\in h_{R}(A) \times_{\mathcal{F}(A)} \mathcal{F}(B)$ with commutative diagram
    \begin{equation}
        \begin{tikzcd}
            h_{A} \arrow[r, "\theta"] \arrow[d, hookrightarrow] &
            h_{R} \arrow[d, "\varphi"] \\
            h_{B} \arrow[ur, dashrightarrow, "\delta"] \arrow[r, "\eta"] &
            \mathcal{F}
        \end{tikzcd}
    \end{equation}
    there always exists a lifting $\delta: h_{B} \rightarrow h_{R}$ making the diagram commutative.
    Many of following propositions can also be interpreted as such a lifting diagram.
    \par
    Recall that we have defined that a functor of artinian local ring $\mathcal{F}$ is said to be pro-representable if there exists $R\in \widehat{\mathcal{C}}$ such that $\mathcal{F} \cong h_{R}$.
    Thus $\mathcal{F}$ has universal family is equivalent to $\mathcal{F}$ is pro-representable.
\end{remark}
\begin{lemma}
    Given versal family $\varphi: h_{R} \rightarrow \mathcal{F}$.
    Take $\widetilde{R} = R[[t]]$.
    Then composition $\widetilde{\varphi}: h_{\widetilde{R}} \rightarrow h_{R} \overset{\varphi}{\rightarrow} \mathcal{F}$ is also versal family.
    \label{Lemma 4.2}
\end{lemma}
\begin{proof}
    Note that $h_{\widetilde{R}}(A) \rightarrow h_{R}(A)$ is surjective for all $A\in \mathcal{C}$, 
    we get $\widetilde{\varphi}(A)$ is still surjective as composition of surjections.
    And for commutative diagram 
    \begin{equation}
        \begin{tikzcd}
            h_{A} \arrow[r, "\theta"] \arrow[d, hookrightarrow] &
            h_{\widetilde{R}} \arrow[r, twoheadrightarrow] \arrow[d, "\widetilde{\varphi}"] &
            h_{R} \arrow[dl, "\varphi"] \\
            h_{B} \arrow[ur, dashrightarrow, "\widetilde{\delta}"] \arrow[urr, dashrightarrow, "\delta"] \arrow[r, "\eta"] &
            \mathcal{F}
        \end{tikzcd}
    \end{equation}
    lifting $\delta: h_{B} \rightarrow h_{R}$ always exists.
    Assume that $\delta$ corresponds to ring homomorphism $R \rightarrow B$.
    Then the existence of lifting $\widetilde{\delta}$ making the diagram commutative is equivalent to existence of lifting $R[[t]] \rightarrow B$ in the following commutative diagram
    \begin{equation}
        \begin{tikzcd}
            R \arrow[r, "\delta"] \arrow[d, hookrightarrow] &
            B \arrow[d, twoheadrightarrow] \\
            R[[t]] \arrow[ur, dashrightarrow] \arrow[r, "\theta"] &
            A 
        \end{tikzcd}
    \end{equation}
    In fact, we can construct the map by mapping $t$ to any preimage of $\theta(t)$ in $B$.
\end{proof}
\begin{proposition}
    Given versal family $h_{R} \rightarrow \mathcal{F}$, then \\
    (a)$\sharp \mathcal{F}(k) = 1$ \\
    (b)For all $A' \twoheadrightarrow A$ and $A'' \twoheadrightarrow A$,
    map $\mathcal{F}(A' \times_{A} A'') \rightarrow \mathcal{F}(A') \times_{\mathcal{F}(A)} \mathcal{F}(A'')$ is surjective.
    \par
    Further if $h_{R} \rightarrow \mathcal{F}$ is miniversal, then \\
    (c)$\mathcal{F}(k[\varepsilon]) =: t_{\mathcal{F}}$ is naturally a finite dimensional $k$-vector space, called the tangent space of $\mathcal{F}$. \\
    (d)For all $A\in \mathcal{C}$, $\mathcal{F}(A \times_{k} k[\varepsilon]) \rightarrow \mathcal{F}(A) \times_{\mathcal{F}(k)} \mathcal{F}(k[\varepsilon])$ is bijective. \\
    (e)For small extension $A' \twoheadrightarrow A$ and all $\eta\in \mathcal{F}(A)$,
    $t_{\mathcal{F}} \groupaction \{\eta'\in \mathcal{F}(A') \big{|} \eta' \mapsto \eta\}$ is a transitive group action. 
    \par
    Finally, if $h_{R} \rightarrow \mathcal{F}$ is universal, then \\
    (f)Maps in (b) is bijective \\
    (g)In (e), $\{\eta'\in \mathcal{F}(A') \big{|} \eta' \mapsto \eta\}$ is $t_{\mathcal{F}}$-pseudo torsor.
    \label{Proposition 4.1}
\end{proposition}
\begin{remark}
    If take $\mathcal{F}$ as Example \ref{example:deformation_functor_of_artinian_local_ring} and $A = k[\varepsilon]$ in (c), then we get 
    \begin{equation}
        \begin{split}
            \mathcal{F}(k[\varepsilon]) \times_{\mathcal{F}(k)} \mathcal{F}(k[\varepsilon]) & \overset{\sim}{\longleftarrow} \mathcal{F}(k[\varepsilon] \times_{k} k[\varepsilon]) \\
            & = \mathcal{F}(k[\varepsilon_{1}, \varepsilon_{2}]) \\
            & \longrightarrow \mathcal{F}(k[\varepsilon])
        \end{split}
    \end{equation}
    which is in fact the addition of $H^{1}(X, \mathcal{T}_{X})$.
    \par
    For versal case, $t_{\mathcal{F}}$ need not be a vector space hence we don't have group structure on $t_{\mathcal{F}}$.
    For miniversal case, the group action can be not free i.e. there could exist stabilizer.
    For universal case, the group action is free.
\end{remark}
\begin{proof}
    (a)Since $h_{R}(k) = \mor_{Sch_{k}}(\spec k, \spec R)$ is one point set and $h_{R}(k) \twoheadrightarrow \mathcal{F}(k)$ is surjective,
    $\mathcal{F}(k)$ is also one point set. 
    \par
    (b)Given commutative diagram
    \begin{equation}
        \begin{tikzcd}[sep = small]
            h_{A'} \arrow[rr, "\theta"] \arrow[dr, dashrightarrow, "f"] &&
            \mathcal{F} \\
            & h_{R} \arrow[ur, "\varphi"] & \\
            h_{A} \arrow[uu] \arrow[ur] \arrow[rr] &&
            h_{A''} \arrow[ul, dashrightarrow, "g" swap] \arrow[uu, "\eta"]
        \end{tikzcd}
    \end{equation}
    Since $h_{R}(A) \twoheadrightarrow \mathcal{F}_{A}$ is surjective, $h_{A} \rightarrow \mathcal{F}$ factor through some $h_{A} \rightarrow h_{R}$.
    As $A' \twoheadrightarrow A$ and $A'' \twoheadrightarrow A$ are surjections, by strong surjectivity, 
    there exist liftings $f: h_{A'} \rightarrow h_{R}$ and $g: h_{A''} \rightarrow h_{R}$ making the diagram commutative.
    Hence we get
    \begin{equation}
        \begin{tikzcd}[sep = small]
            R \arrow[drr, "g", bend left] \arrow[dr, dashrightarrow] \arrow[ddr, "f", bend right] && \\
            & A' \times_{A} A'' \arrow[r] \arrow[d] & 
            A'' \arrow[d] \\
            & A' \arrow[r] &
            A \\
        \end{tikzcd}
        \label{equation:proposition_versal_family_b_fibered_product}
    \end{equation}
    By universal property of fibered product, we get there exist $R \rightarrow A' \times_{A} A''$ inducing $h_{A' \times_{A} A''} \rightarrow h_{R}$.
    Taking composition, we get $h_{A' \times_{A} A''} \rightarrow \mathcal{F}$ making the following diagram commutative
    \begin{equation}
        \begin{tikzcd}[sep = small]
            h_{A'} \arrow[rr, "\theta"] \arrow[dr] &&
            \mathcal{F} \\
            & h_{A' \times_{A} A''} \arrow[ur, "\varphi"] & \\
            h_{A} \arrow[uu] \arrow[rr] &&
            h_{A''} \arrow[ul] \arrow[uu, "\eta"]
        \end{tikzcd}
    \end{equation}
    Conclude that $\mathcal{F}(A' \times_{A} A'') \rightarrow \mathcal{F}(A') \times_{\mathcal{F}(A)} \mathcal{F}(A'')$ is surjective.
    \par
    (c)As $h_{R}(k[\varepsilon]) \overset{\sim}{\longrightarrow} \mathcal{F}(k[\varepsilon])$, 
    it suffices to give a $k$-vector space structure on $h_{R}(k[\varepsilon])$ as following
    \begin{equation}
        \begin{split}
            & (r \mapsto a + b_{1}\varepsilon) + (r \mapsto a + b_{2}\varepsilon) \\
            = & r \mapsto a + (b_{1} + b_{2})\varepsilon \\
            & k \cdot (r \mapsto a + b\varepsilon) \\
            = & r \mapsto a + kb\varepsilon
        \end{split}
    \end{equation}
    Easy to check these are well defined.
    \par
    (d)By (b), we already have that $\mathcal{F}(A \times_{k} k[\varepsilon]) \twoheadrightarrow \mathcal{F}(A) \times_{\mathcal{F}(k)} \mathcal{F}(k[\varepsilon])$ is surjective.
    Suffice to show injectivity.
    Consider the following diagram.
    \begin{equation}
        \begin{tikzcd}
            & h_{R} \arrow[dr, "\varphi"] && \\
            h_{A} \arrow[ur] \arrow[rr] \arrow[dr, "\eta"] &&
            \mathcal{F} & \\
            & h_{A \times_{k} k[\varepsilon]} \arrow[uu, "f_{1}" near end, shift left, crossing over] \arrow[uu, "f_{2}" near end, swap, shift right, crossing over] \arrow[ur, "\varphi_{1}", shift left] \arrow[ur, "\varphi_{2}" swap, shift right] &&
            h_{k[\varepsilon]} \arrow[ll, "\pi"] \arrow[ul]
        \end{tikzcd}
    \end{equation}
    If there exist two elements $\varphi_{1},\varphi_{2}\in \mathcal{F}(A \times_{k} k[\varepsilon])$ making the diagram commutative,
    then by strong surjectivity, $h_{R}(A \times_{k} k[\varepsilon]) \twoheadrightarrow h_{R}(A) \times_{\mathcal{F}(A)} \mathcal{F}(A \times_{k} k[\varepsilon])$ is surjective,
    there exist $f_{1},f_{2}\in h_{R}(A \times_{k} k[\varepsilon])$ making the diagram commutative.
    In particular, we get $f_{1} \circ \eta = f_{2} \circ \eta$.
    \par
    Now we have that
    \begin{equation}
        \begin{split}
            \varphi \circ f_{1} \circ \pi & = \varphi_{1} \circ \pi \\
            & = \varphi_{2} \circ \pi \\
            & = \varphi \circ f_{2} \circ \pi
        \end{split}
    \end{equation}
    While $h_{R}(k[\varepsilon]) \overset{\sim}{\rightarrow} \mathcal{F}(k[\varepsilon])$ is bijective, 
    we get $f_{1} \circ \pi = f_{2} \circ \pi$.
    Thus there is a commutative diagram
    \begin{equation}
        \begin{tikzcd}[sep = small]
            R \arrow[drr, bend left] \arrow[dr, "f_{1}", shift left] \arrow[dr, "f_{2}" swap, shift right] \arrow[ddr, bend right]&& \\
            & A \times_{k} k[\varepsilon] \arrow[r, "\pi"] \arrow[d, "\eta"] & 
            k[\varepsilon] \arrow[d] \\
            & A \arrow[r] &
            k \\
        \end{tikzcd}
    \end{equation}
    By universal property of fibered product, we get $f_{1} = f_{2}$.
    Conclude that $\varphi_{1} = \varphi_{2}$ so that $\mathcal{F}(A \times_{k} k[\varepsilon]) \overset{\sim}{\rightarrow} \mathcal{F}(A) \times_{\mathcal{F}(k)} \mathcal{F}(k[\varepsilon])$ is bijective.
    \par
    (e)For small extension $A' \twoheadrightarrow A$, assume the kernel ideal is $(t)$.
    Then there is an isomorphism $A' \times_{A} A' \overset{\sim}{\rightarrow} A' \times_{k} k[\varepsilon]$ defined by
    \begin{equation}
        (x, y) \longmapsto (x, \overline{x} + \overline{\frac{x - y}{t}}\varepsilon)
    \end{equation}
    And we get a diagram
    \begin{equation}
        \begin{tikzcd}
            \mathcal{F}(A') \times_{\mathcal{F}(k)} \mathcal{F}(k[\varepsilon]) \arrow[r, "\psi"] \arrow[d, "\sim"] &
            \mathcal{F}(A') \times_{\mathcal{F}(A)} \mathcal{F}(A') \\
            \mathcal{F}(A' \times_{k} k[\varepsilon]) \arrow[r, "\sim"] &
            \mathcal{F}(A' \times_{A} A') \arrow[u, twoheadrightarrow]
        \end{tikzcd}
    \end{equation}
    where $\psi$ comes from composition.
    Thus we can define a group action that for all $v\in \mathcal{F}(k[\varepsilon])$ and $\eta'\in \mathcal{F}(A)$ mapping to $\eta$,
    $v(\eta')$ is the second entry in $\psi(\eta, v)$ corresponding to the following natural transformation
    \begin{equation}
        h_{A'} \longrightarrow h_{A' \times_{A} A'} \longrightarrow h_{A' \times_{k} k[\varepsilon]} \overset{(\eta, v)}{\longrightarrow} \mathcal{F}
    \end{equation}
    \par
    (f)Similarly consider diagram
    \begin{equation}
        \begin{tikzcd}
            & h_{R} \arrow[dr, "\varphi"] && \\
            h_{A'} \arrow[ur] \arrow[rr] \arrow[dr, "\eta"] &&
            \mathcal{F} & \\
            & h_{A' \times_{A} A''} \arrow[uu, "f_{1}" near end, shift left, crossing over] \arrow[uu, "f_{2}" near end, swap, shift right, crossing over] \arrow[ur, "\varphi_{1}", shift left] \arrow[ur, "\varphi_{2}" swap, shift right] &&
            h_{A''} \arrow[ll, "\pi"] \arrow[ul]
        \end{tikzcd}
    \end{equation}
    If there exist two elements $\varphi_{1},\varphi_{2}\in \mathcal{F}(A' \times_{A} A'')$ making the diagram commutative,
    then $h_{R}(A' \times_{A} A'') \twoheadrightarrow \mathcal{F}(A' \times_{A} A'')$ is surjective,
    there exist $f_{1},f_{2}\in h_{R}(A' \times_{A} A'')$ such that $\varphi \circ f_{i} = \varphi_{i}$.
    \par
    Now we have that
    \begin{equation}
        \begin{split}
            \varphi \circ f_{1} \circ \pi & = \varphi_{1} \circ \pi \\
            & = \varphi_{2} \circ \pi \\
            & = \varphi \circ f_{2} \circ \pi
        \end{split}
        and
        \begin{split}
            \varphi \circ f_{1} \circ \eta & = \varphi_{1} \circ \eta \\
            & = \varphi_{2} \circ \eta \\
            & = \varphi \circ f_{2} \circ \eta
        \end{split}
    \end{equation}
    While $h_{R}(A') \overset{\sim}{\rightarrow} \mathcal{F}(A')$ and $h_{R}(A'') \overset{\sim}{\rightarrow} \mathcal{F}(A'')$ is bijective, 
    we get $f_{1} \circ \pi = f_{2} \circ \pi$ and $f_{1} \circ \eta = f_{2} \circ \eta$.
    Thus there is a commutative diagram
    \begin{equation}
        \begin{tikzcd}[sep = small]
            R \arrow[drr, bend left] \arrow[dr, "f_{1}", shift left] \arrow[dr, "f_{2}" swap, shift right] \arrow[ddr, bend right]&& \\
            & A' \times_{A} A'' \arrow[r, "\pi"] \arrow[d, "\eta"] & 
            A'' \arrow[d] \\
            & A' \arrow[r] &
            A \\
        \end{tikzcd}
    \end{equation}
    By universal property of fibered product, we get $f_{1} = f_{2}$.
    Conclude that $\varphi_{1} = \varphi_{2}$ so that $\mathcal{F}(A' \times_{A} A'') \overset{\sim}{\rightarrow} \mathcal{F}(A') \times_{\mathcal{F}(A)} \mathcal{F}(A'')$ is bijective.
    \par
    (g)By (f), we immediately get the map $\psi$ defined in (e) is bijective hence $\{\eta'\in \mathcal{F}(A') \big{|} \eta' \mapsto \eta\}$ is $t_{\mathcal{F}}$-pseudo torsor.
\end{proof}
\begin{remark}
    For (b), in fact we only need one of the two surjections since we can get $h_{A} \rightarrow h_{R}$ by compositing with $h_{A'} \rightarrow h_{R}$ and $h_{A''} \rightarrow h_{R}$ at first,
    so that the other surjection would similarly give diagram \ref{equation:proposition_versal_family_b_fibered_product}.
\end{remark}

\subsection{Schelessinger's Criterion}
\label{subsection:Formal_Family_Schelessinger's_Criterion}

\begin{theorem}
    Let $\mathcal{F}$ be a functor of artinian local ring.
    Then $\mathcal{F}$ has miniversal family i.e. there exists $R\in \widehat{\mathcal{C}}$ such that $\varphi: h_{R} \rightarrow \mathcal{F}$ is miniversal,
    if and only if the following conditions hold \\
    ($H_{0}$)$\sharp \mathcal{F}(k) = 1$ \\
    ($H_{1}$)For any small extension $A'' \twoheadrightarrow A$ and any morphism $A' \rightarrow A$, 
    the natural map $\mathcal{F}(A' \times_{A} A'') \twoheadrightarrow \mathcal{F}(A') \times_{\mathcal{F}(A)} \mathcal{F}(A'')$ is surjective. \\
    ($H_{2}$)For any morphism $A' \rightarrow A$,
    the natural map $\mathcal{F}(A' \times_{k} k[\varepsilon]) \twoheadrightarrow \mathcal{F}(A') \times_{\mathcal{F}(k)} \mathcal{F}(k[\varepsilon])$ is bijective. \\
    ($H_{3}$)$\mathcal{F}(k[\varepsilon])$ is a finite dimensional $k$-vector space.
    \par
    Further, $\mathcal{F}$ has universal family if and only if $H_{0}$, $H_{1}$, $H_{2}$, $H_{3}$ and following $H_{4}$ hold \\
    ($H_{4}$)For any small extension $A' \twoheadrightarrow A$ and $\eta\in \mathcal{F}(A)$,
    $\{\eta'\in \mathcal{F}(A') \big{|} \eta' \text{ mapping to } \eta\}$ is a $t_{\mathcal{F}}$-pseudo torsor under the group .
    \label{Theorem 4.1} 
\end{theorem}
\begin{example}
    Let $X_{0}$ be a scheme over $k$.
    Consider the deformation functor $\mathcal{F}$ of $X_{0}$ defined in Example \ref{example:deformation_functor_of_artinian_local_ring}.
    If $X$ satisfies either of the two following hypothesis
    \begin{itemize}
        \item $X_{0}$ is projective
        \item $X_{0}$ is affine with isolated singularities
    \end{itemize}
    then $\mathcal{F}$ has miniversal family.
    In fact, $H_{0}$, $H_{1}$ and $H_{2}$ hold for all $k$-scheme $X_{0}$ and the two hypothesis only play a role in $H_{3}$.
    \par
    In particular, let $X_{0}$ be a smooth projective variety.
    There is an obvious question that why we get pseudo torsor condition in previous section but here we only have miniversal family.
    The difference occurs in the equivalence relations.
    When we talk about deformation of $X/A$ to $X'/A'$ for star extension $A' \twoheadrightarrow A$,
    an isomorphism is a commutative diagram
    \begin{equation}
        \begin{tikzcd}[sep = small]
            & X \arrow[dr, hookrightarrow] \arrow[dl, hookrightarrow] & \\
            X_{1}' \arrow[rr, "\sim"] \arrow[dr] &&
            X_{2}' \arrow[dl] \\
            & \spec A' &
        \end{tikzcd}
    \end{equation}
    But here in $\mathcal{F}(A')$, an isomorphism is a commutative diagram
    \begin{equation}
        \begin{tikzcd}[sep = small]
            & X_{0} \arrow[dr, hookrightarrow] \arrow[dl, hookrightarrow] & \\
            X_{1}' \arrow[rr, "\sim"] \arrow[dr] &&
            X_{2}' \arrow[dl] \\
            & \spec A' &
        \end{tikzcd}
    \end{equation}
    Clearly, the second equivalence relation is much weaker causing that isomorphism classes would be less and hence there exist nontrivial stabilizer.
\end{example}
Enlightened by this example, we could give a condition equivalent to universal.
\begin{proposition}
    Let $X_{0}$ be a scheme over $k$, $\mathcal{F}$ the deformation functor of $X_{0}$.
    Assume $X$ satisfies either of the two following hypothesis
    \begin{itemize}
        \item $X_{0}$ is projective
        \item $X_{0}$ is affine with isolated singularities
    \end{itemize}
    Then $\mathcal{F}$ has universal family if and only if for all deformation $X'/A'$,
    the natural map $\aut_{A'}(X'/X_{0}) \rightarrow \aut_{A}(X/X_{0})$ is surjective, where $X = X'\big{|}_{\spec A}$. 
    \label{Proposition 4.2}
\end{proposition}
\begin{proof}
    "$\Rightarrow$": Suppose there is an automorphism $\varphi: X \overset{\sim}{\rightarrow} X$ which cannot be lifted to $\aut_{A'}(X'/X_{0})$.
    Denote $i: X_{0} \hookrightarrow X$ and $j: X \hookrightarrow X'$.
    Then $j: X \hookrightarrow X'$ and $j \circ \varphi: X \hookrightarrow X'$ are not equivalent as deformation of $X$.
    While $\varphi \circ i = i$, they are equivalent in $\mathcal{F}(A')$.
    \par
    If denote equivalence classes under the two different equivalence relations by $[\cdot]_{1}$ and $[\cdot]_{2}$,
    then since $[(X', j)]_{1} \neq [(X', j \circ \varphi)]_{1}$ and the group action is transitive, 
    there exists nonzero $v\in t_{\mathcal{F}}$ such that $v[(X', j)]_{1} = [(X', j \circ \varphi)]_{1}$ but $v[(X', j \circ i)]_{2} = [(X', j \circ i)]_{2}$.
    Thus stabilizer of $[(X', j \circ i)]_{2}$ is nontrivial, contradicting to universal.
    \par
    "$\Leftarrow$": Suffice to show that the two equivalence relations are same.
    Assume $X_{1}'$ and $X_{2}'$ are two deformations of $X$ over $A'$ satisfying that $[X_{1}']_{2} = [X_{2}']_{2}$.
    Then there is an commutative diagram
    \begin{equation}
        \begin{tikzcd}[sep = small]
            & X_{0} \arrow[dr, hookrightarrow] \arrow[dl, hookrightarrow] & \\
            X_{1}' \arrow[rr, "\sim", "\varphi" swap] \arrow[dr] &&
            X_{2}' \arrow[dl] \\
            & \spec A' &
        \end{tikzcd}
    \end{equation}
    Now $\varphi$ restricts to an automorphism $\varphi\big{|}_{X}: X \overset{\sim}{\rightarrow} X$.
    By assumption, there exists $\psi\in \aut_{A'}(X_{1}'/X_{0})$ extending $\varphi\big{|}_{X}^{-1}$.
    Hence we get commutative diagram
    \begin{equation}
        \begin{tikzcd}
            X \arrow[r, "\varphi\big{|}_{X}^{-1}", "\sim" swap] \arrow[d, hookrightarrow, "j_{1}"] &
            X \arrow[r, "\varphi\big{|}_{X}", "\sim" swap] \arrow[d, hookrightarrow, "j_{1}"] &
            X \arrow[d, hookrightarrow, "j_{2}"] \\
            X_{1}' \arrow[r, "\sim", "\psi" swap] \arrow[dr] &
            X_{1}' \arrow[r, "\sim", "\varphi" swap] \arrow[d] &
            X_{2}' \arrow[dl] \\
            & \spec A' &
        \end{tikzcd}
    \end{equation}
    which shows that $[X_{1}']_{1} = [X_{2}']_{1}$, done!
\end{proof}
\begin{corollary}
    Let $X_{0}$ be a projective scheme over $k$ with $H^{0}(X_{0}, \mathcal{T}_{X_{0}}) = 0$, $\mathcal{F}$ the deformation functor of $X_{0}$.
    Then $\mathcal{F}$ has universal family.
    \label{Corollary 4.1}
\end{corollary}
\begin{reason}
    With $H^{0}(X_{0}, \mathcal{T}_{X_{0}}) = 0$, by induction on length of $A$,
    we can prove that for all deformation $X/A$, $\aut_{A}(X/A) = \{\identity\}$.
    Hence by Proposition \ref{Proposition 4.2}, we are done.
\end{reason}
\begin{example}
    For curve $X_{0}$ of genus at least $2$, as $H^{0}(X_{0}, \mathcal{T}_{X_{0}}) = 0$, $\mathcal{F}$ has universal family.
    For curve $X_{0}$ of genus $1$, even though $H^{0}(X_{0}, \mathcal{T}_{X_{0}}) = k \neq 0$, $\mathcal{F}$ still has universal family since
    \begin{equation}
        \aut_{A'}(X'/X_{0}) = A' \twoheadrightarrow \aut_{A}(X/X_{0}) = A
    \end{equation}
    is just the surjection.
\end{example}
\begin{example}
    Consider vector bundles on $\mathbb{P}_{k}^{1}$ of rank $2$, which should of the form $\mathcal{O}(a) \oplus \mathcal{O}(b)$.
    For example, the automorphism group of $\mathcal{O}^{\oplus 2}$ is $GL_{2}(k)$ and the automorphism group of $\mathcal{O}(-1) \oplus \mathcal{O}(1)$ is $G := \{
    \begin{pmatrix}
        \mu & \\
        f & \lambda
    \end{pmatrix}
    \big{|}
    \mu,\lambda\in k^{\times} \text{ and } f\in H^{0}(\mathcal{O}(2))
    \}$.
    Note dimensions are respectively $4$ and $5$.
    So intuitively, $\aut(\mathcal{O}(-1) \oplus \mathcal{O}(1))$ not always lift to $\aut(\mathcal{O}^{\oplus 2})$.
\end{example}

\subsection{Crude deformation functor}
\label{subsection:Formal_Family_Crude_deformation_functor}

\begin{definition}
    Let $X_{0}$ be a $k$-scheme.
    Define the crude deformation functor $\mathcal{F}_{1}$ of $X_{0}$ as following
    \begin{equation}
        A \longmapsto \mathcal{F}_{1}(A) = \{X \overset{flat}{\rightarrow} \big{|} X \times_{A} k \cong X_{0}\}/\sim_{1}
    \end{equation}
    where the equivalence relation is given by the following commutative diagram
    \begin{equation}
        \begin{tikzcd}[sep = small]
            X_{1} \arrow[rr, "\sim"] \arrow[dr] &&
            X_{2} \arrow[dl] \\
            & \spec A &
        \end{tikzcd}
    \end{equation}
\end{definition}
\begin{remark}
    Comparing to deformation functor, here the equivalence relation does not ask the isomorphism to be compatible with closed immersions $X_{0} \hookrightarrow X_{1}$ and $X_{0} \hookrightarrow X_{2}$,
    resulting that $\mathcal{F}_{1}(A)$ is much smaller and $\mathcal{F}_{1}$ is not well behaved in general.
\end{remark}
\begin{lemma}
    Let $X_{0}$ be a $k$-scheme, $\mathcal{F}$ deformation functor of $X_{0}$ and $\mathcal{F}_{1}$ crude deformation functor of $X_{0}$.
    Then there is a strongly surjective natural transformation $\psi: \mathcal{F} \rightarrow \mathcal{F}_{1}$ defined by
    \begin{equation}
        \psi(A): \mathcal{F}(A) \longrightarrow \mathcal{F}_{1}(A)\quad X \longmapsto [X]_{1}
    \end{equation}
    where $[X]_{1}$ denotes the equivalence class of $X$ in $\mathcal{F}_{1}(A)$.
    \label{Lemma 4.3}
\end{lemma}
\begin{proof}
    Obviously, for all $A\in \mathcal{C}$, $\psi(A): \mathcal{F}(A) \rightarrow \mathcal{F}_{1}(A)$ is surjective.
    For any surjection $B \twoheadrightarrow A$ and commutative diagram
    \begin{equation}
        \begin{tikzcd}
            h_{A} \arrow[r, "X"] \arrow[d, hookrightarrow] &
            \mathcal{F} \arrow[d, "\psi"] \\
            h_{B} \arrow[ur, dashrightarrow, "?" description] \arrow[r, "{[Y]_{1}}" swap] &
            \mathcal{F}_{1}
        \end{tikzcd}
    \end{equation}
    where $X\in \mathcal{F}(A)$ and $[Y]_{1}\in \mathcal{F}_{1}(B)$,
    want to show that there exists lifting $h_{B} \rightarrow \mathcal{F}$.
    \par
    As the diagram is commutative, we get $Y \sim_{1} X'$ for some deformation of $X$ over $B$. 
    Thus we can replace $[Y]_{1}$ by $[X']_{1}$.
    Take $h_{B} \rightarrow \mathcal{F}$ corresponding to $X'\in \mathcal{F}(B)$ and clearly this is our desired lifting.
\end{proof}
\begin{proposition}
    Let $X_{0}$ be a $k$-scheme satisfying one of the two hypothesis in Proposition \ref{Proposition 4.2}, $\mathcal{F}_{1}$ crude deformation functor of $X_{0}$.
    Then \\
    (1)$\mathcal{F}_{1}$ has a versal family. \\
    (2)$\mathcal{F}_{1}$ satisfies ($H_{0}$) and ($H_{1}$). \\
    (3)$\mathcal{F}_{1}$ has a miniversal family if and only if for any deformation $X \rightarrow \spec k[\varepsilon]$,
    the natural map $\aut_{k[\varepsilon]}(X) \twoheadrightarrow \aut(X_{0})$ is surjective.
    \label{Proposition 4.3}
\end{proposition}
\begin{proof}
    (1)Consider deformation functor $\mathcal{F}$ of $X_{0}$ and natural transformation $\psi: \mathcal{F} \rightarrow \mathcal{F}_{1}$ defined in previous lemma.
    Since $\mathcal{F}$ has a versal family $\varphi: h_{R} \rightarrow \mathcal{F}$, 
    what we need to show is that composition of strongly surjective natural transformations is still strongly surjective.
    \par
    For all $A\in \mathcal{C}$, $\psi \circ \varphi(A): h_{R} \rightarrow \mathcal{F}_{1}(A)$ is composition of surjections hence surjective.
    For any surjection $B \twoheadrightarrow A$ and commutative diagram
    \begin{equation}
        \begin{tikzcd}
            h_{A} \arrow[r] \arrow[d, hookrightarrow] &
            h_{R} \arrow[r, "\varphi"] & 
            \mathcal{F} \arrow[d, "\psi"] \\
            h_{B} \arrow[ur, dashrightarrow, "{\delta'}"] \arrow[urr, dashrightarrow, "\delta"] \arrow[rr] &&
            \mathcal{F}_{1}
        \end{tikzcd}
    \end{equation}
    As $\psi$ is strongly surjective, there exists lifting $\delta: h_{B} \rightarrow \mathcal{F}$ making the diagram commutative.
    Then again as $\varphi$ is strongly surjective, there exists lifting $\delta': h_{B} \rightarrow h_{R}$ making the diagram commutative, done!
    \par
    (2)Now $\psi \circ \varphi: h_{R} \rightarrow \mathcal{F}_{1}$ is a versal family.
    Then immediately $\mathcal{F}_{1}(k)$ is one-point set and ($H_{0}$) holds.
    For any morphism $f: A' \rightarrow A$ and any small extension $g: A'' \twoheadrightarrow A$, consider the following commutative diagram
    \begin{equation}
        \begin{tikzcd}
            \mathcal{F}(A' \times_{A} A'') \arrow[r, twoheadrightarrow] \arrow[d, twoheadrightarrow] &
            \mathcal{F}(A') \times_{\mathcal{F}(A)} \mathcal{F}(A'') \arrow[d, "{\psi(A') \times_{\psi(A)} \psi(A'')}"] \\
            \mathcal{F}_{1}(A' \times_{A} A'') \arrow[r] &
            \mathcal{F}_{1}(A') \times_{\mathcal{F}_{1}(A)} \mathcal{F}_{1}(A'')
        \end{tikzcd}
    \end{equation}
    Hence we only need to show that $\psi(A') \times_{\psi(A)} \psi(A'')$ is surjective.
    For $([X']_{1}, [X'']_{1})\in \mathcal{F}_{1}(A') \times_{\mathcal{F}_{1}(A)} \mathcal{F}_{1}(A'')$,
    as $\mathcal{F}(A') \twoheadrightarrow \mathcal{F}_{1}(A')$ is surjective, we can lift $[X']_{1}$ to $X'\in \mathcal{F}(A')$.
    Then there is a commutative diagram
    \begin{equation}
        \begin{tikzcd}
            h_{A} \arrow[r, "{X'\big{|}_{\spec A}}"] \arrow[d, hookrightarrow] &
            \mathcal{F} \arrow[d, "\psi"] \\
            h_{A''} \arrow[ur, dashrightarrow, "{Y''}"] \arrow[r, "{[X'']_{1}}" swap] &
            \mathcal{F}_{1}
        \end{tikzcd}
    \end{equation}
    as $\psi$ is strongly surjective, there exists lifting $h_{A''} \rightarrow \mathcal{F}$ corresponding to $Y''\in \mathcal{F}(A'')$.
    Hence $Y''\big{|}_{\spec A} \sim X'\big{|}_{\spec A}$ and $[Y'']_{1} = [X'']_{1}$ so that $(X', Y'') \mapsto ([X']_{1}, [X'']_{1})$.
    Thus $\psi(A') \times_{\psi(A)} \psi(A'')$ surjective and ($H_{1}$) holds.
    \par
    (3)As $\phi$ is miniversal, $\psi \circ \varphi$ is miniversal if and only if $\psi(k[\varepsilon])$ is bijective.
    Assume the right side holds.
    For $X_{1}, X_{2}\in \mathcal{F}(k[\varepsilon])$, if $X_{1} \sim_{1} X_{2}$,
    then there exists isomorphism $\phi$ with commutative diagram
    \begin{equation}
        \begin{tikzcd}[sep = small]
            X_{1} \arrow[rr, "\sim", "\phi" swap] \arrow[dr] &&
            X_{2} \arrow[dl] \\
            & \spec k[\varepsilon] &
        \end{tikzcd}
    \end{equation}
    Restrict $\phi$ to $X_{0}$. 
    As $\aut_{k[\varepsilon]}(X_{1}) \twoheadrightarrow \aut(X_{0})$ is surjective,
    then we can extend $\phi\big{|}_{X_{0}}^{-1}$ to an automorphism $\widetilde{\phi}: X_{1} \overset{\sim}{\rightarrow} X_{1}$.
    \begin{equation}
        \begin{tikzcd}
            X_{0} \arrow[r, "\phi\big{|}_{X_{0}}^{-1}", "\sim" swap] \arrow[d, hookrightarrow, "j_{1}"] &
            X_{0} \arrow[r, "\phi\big{|}_{X_{0}}", "\sim" swap] \arrow[d, hookrightarrow, "j_{1}"] &
            X_{0} \arrow[d, hookrightarrow, "j_{2}"] \\
            X_{1} \arrow[r, "\sim", "\widetilde{\phi}" swap] \arrow[dr] &
            X_{1} \arrow[r, "\sim", "\phi" swap] \arrow[d] &
            X_{2} \arrow[dl] \\
            & \spec k[\varepsilon] &
        \end{tikzcd}
    \end{equation}
    Hence $X_{1} \sim X_{2}$ so that $\psi(k[\varepsilon])$ is bijective.
    \par
    Conversely, assume $\psi(k[\varepsilon])$ is bijective.
    Suppose there exists $\phi: X_{0} \overset{\sim}{\rightarrow} X_{0}\in \aut(X_{0})$ cannot be extended to element in $\aut_{k[\varepsilon]}(X)$ for some $(X, i)\in \mathcal{F}(k[\varepsilon])$,
    where $i: X_{0} \hookrightarrow X$ is the closed immersion. 
    Then $(X, i)$ must be not equivalent to $(X, \phi \circ i)$, 
    while $[(X, i)]_{1}$ and $[(X, \phi \circ i)]_{1}$ are obviously same, contradiciton!
\end{proof}
\begin{example}
    Set $k = \mathbb{C}$.
    Consider maps from scheme $S$ to $[\mathbb{C}^{2}/\mathbb{C}^{\times}]$.
    By definition, a map is equivalent to datum $(\mathcal{L}, x, y)$ where $\mathcal{L}\in \picardscheme(S)$ is line bundle and $x,y\in H^{0}(S, \mathcal{L})$.
    Given $\mathcal{L}_{0}\in \picardscheme(\spec \mathbb{C})$ and $x_{0} = y_{0} = 0$, define deformation functor as 
    \begin{equation*}
        \mathcal{F}(A) = \{\mathcal{L}\in \picardscheme(\spec A), x,y\in H^{0}(\spec A, \mathcal{L}) \big{|} x,y \equiv 0\!\!\!\! \pmod {\mathfrak{m}} \text{ and } \mathcal{L} \otimes_{A} \mathbb{C} \underset{\theta}{\overset{\sim}{\rightarrow}} \mathcal{L}_{0}\}/\sim
    \end{equation*}
    where $(\mathcal{L}_{1}, x_{1}, y_{1}, \theta_{1}) \sim (\mathcal{L}_{2}, x_{2}, y_{2}, \theta_{2})$ if there exists $\varphi: \mathcal{L}_{1} \overset{\sim}{\rightarrow} \mathcal{L}_{2}$ such that $\varphi(x_{1}) = x_{2}$, $\varphi(y_{1}) = y_{2}$ and there is a commutative diagram
    \begin{equation}
        \begin{tikzcd}[sep = small]
            \mathcal{L}_{1} \otimes_{A} \mathbb{C} \arrow[rr, "\varphi \otimes \identity"] \arrow[dr, "\theta_{1}" swap] &&
            \mathcal{L}_{2} \otimes_{A} \mathbb{C} \arrow[dl, "\theta_{2}"] \\
            & \mathcal{L}_{0} &
        \end{tikzcd}
    \end{equation}
    As line bundle on $\spec A$ can be trivialized, we can rewrite $\mathcal{F}(A)$ as $\{(x, y) \big{|} x,y\in \mathfrak{m}_{A}\}/\sim$,
    where $x \sim y$ if there exists $h\in 1 + \mathfrak{m}$ such that $x_{1} = hx_{2}$ and $y_{1} = hy_{2}$.
    \par
    Consider the tangent space of $\mathcal{F}$.
    We have that $\mathcal{F}(\mathbb{C}[\varepsilon]) = \{(a\varepsilon, b\varepsilon) \big{|} a,b\in \mathbb{C}\}/\sim$.
    Suppose $(a_{1}\varepsilon, b_{1}\varepsilon) \sim (a_{2}\varepsilon, b_{2}\varepsilon)$, then there exists $c\in \mathbb{C}$ such that
    \begin{equation}
        \left\{
            \begin{aligned}
                a_{1}\varepsilon = (1 + c\varepsilon)a_{2}\varepsilon \\
                b_{1}\varepsilon = (1 + c\varepsilon)b_{2}\varepsilon 
            \end{aligned}
        \right.
        \Rightarrow
        \left\{
            \begin{aligned}
                a_{1} = a_{2} \\
                b_{1} = b_{2}
            \end{aligned}
        \right.
    \end{equation}
    Hence the equivalence relation is just identity.
    \par
    In fact, $\mathcal{F}$ has miniversal family over $R = \widehat{\mathcal{O}_{\mathbb{C}^{2}, (0, 0)}}$.
    Given $(x, y)\in \mathcal{F}(A)$, we have a spectrum map $\spec A \overset{(x, y)}{\rightarrow} \spec R$.
    Take $A = \mathbb{C}[t]/(t^{3})$, then $\mathcal{F}(A) = \{(a_{1}t + a_{2}t^{2}, b_{1}t + b_{2}t^{2})\}/\sim$.
    Suppose $(a_{1}t + a_{2}t^{2}, b_{1}t + b_{2}t^{2}) \sim (\widetilde{a}_{1}t + \widetilde{a}_{2}t^{2}, \widetilde{b}_{1}t + \widetilde{b}_{2}t^{2})$,
    then there exists $c_{1},c_{2}\in \mathbb{C}$ such that 
    \begin{equation}
        \left\{
            \begin{aligned}
                a_{1}t + a_{2}t^{2} = (1 + c_{1}t + c_{2}t^{2})(\widetilde{a}_{1}t + \widetilde{a}_{2}t^{2})\\
                b_{1}t + b_{2}t^{2} = (1 + c_{1}t + c_{2}t^{2})(\widetilde{b}_{1}t + \widetilde{b}_{2}t^{2})
            \end{aligned}
        \right.
        \Rightarrow
        \left\{
            \begin{aligned}
                a_{1} = \widetilde{a}_{1}\quad & a_{2} = \widetilde{a}_{1}c + \widetilde{a}_{2} \\
                b_{1} = \widetilde{b}_{1}\quad & b_{2} = \widetilde{b}_{1}c + \widetilde{b}_{2}
            \end{aligned}
        \right.
    \end{equation}
    Hence two different spectrum maps from $\spec A$ to $\spec B$ can mapping to same equivalence class in $\mathcal{F}(A)$ so that $h_{R} \rightarrow \mathcal{F}$ is not universal.
\end{example}
\begin{remark}
    In GTM257 written by Hartshorne, Example 18.4.1 gives two example with similar pathology above.
    One shows that crude deformation functor is not necessarily miniversal,
    the other shows that deformation functor is not necessarily universal.
\end{remark}

\section{Algebraization of Formal Moduli}
\label{section:Algebraization_of_Formal_Moduli}

\begin{example}
    Let $X_{0}$ be a smooth projective $k$-scheme, $\mathcal{F}$ deformation functor of $X_{0}$.
    Algebraization of formal moduli is the following process
    \begin{equation*}
        \begin{gathered}
            \text{formal family} \\
            \begin{tikzcd}
                h_{R} \arrow[d] \\
                \mathcal{F}
            \end{tikzcd}
        \end{gathered}
        \rightsquigarrow
        \begin{gathered}
            \text{formal scheme} \\
            \begin{tikzcd}
                \mathfrak{X} \arrow[d] \\
                \spf R
            \end{tikzcd}
        \end{gathered}
        \rightsquigarrow
        \begin{gathered}
            \text{scheme} \\
            \begin{tikzcd}[row sep = 1.53 em]
                \widetilde{X} \arrow[d] \\
                \spec R
            \end{tikzcd}
        \end{gathered}
        \rightsquigarrow
        \begin{gathered}
            \text{family} \\
            \begin{tikzcd}[row sep = 1.95 em]
                X \arrow[d] \\
                S\ni s_{0}
            \end{tikzcd}
        \end{gathered}
    \end{equation*}
    where $\widetilde{X} \rightarrow \spec R$ is flat and of finite type such that $X \times_{\spec R} \spf R \cong \mathfrak{X}$ in category of formal schemes,
    and $S$ is of finite type over $k$ such that $R = \widehat{\mathcal{O}_{S, s_{0}}}$, $X \times_{S} \spec R \cong \widetilde{X}$ and $X \times_{S} s_{0} \cong X_{0}$.
\end{example}
\begin{definition}
    Let $A$ be a noetherian, $I \subseteq A$ ideal, $\widehat{A}$ completion of $A$ under $I$-adic topology.
    Define formal scheme $\spf \widehat{A} = (\mathfrak{X}, \mathcal{O}_{\mathfrak{X}})$ to be a locally ringed space,
    where $\mathfrak{X} = \spec (\widehat{A}/\widehat{I})$ and structure sheaf $\mathcal{O}_{\mathfrak{X}}$ is defined by
    \begin{equation}
        \mathcal{O}_{\mathfrak{X}}(U) = \inverselimit \mathcal{O}_{\spec (\widehat{A}/\widehat{I}^{n})}(U)
    \end{equation}
    This is also called completion of $A$ along $I$.
    Generally, we say a locally ringed space $(\mathfrak{X}, \mathcal{O}_{\mathfrak{X}})$ is a formal scheme if there exists an open covering $U_{i}$ such that $U_{i}$ is completion of noetherian ring $A_{i}$ along ideal $I_{i}$.
\end{definition}
\begin{example}
    (1)Let $A_{0} = k[x, y]$, $I = (x, y)$ and $A = \widehat{A_{0}} = k[[x, y]]$.
    Then underlying topological space of $\spf A$ is a one-point set with structure sheaf given by $A$. \\
    (2)Let $A_{0} = k[x, y]$, $I = (x)$ and $A = \widehat{A_{0}} = k[y][[x]]$.
    Then underlying topological space of $\spf A$ is y-axis and global sections are $\Gamma(\mathfrak{X}, \mathcal{O}_{\mathfrak{X}}) = k[y][[x]]$.
    For $U = D(y)$, $\Gamma(U, \mathcal{O}_{\mathfrak{X}}) = \widehat{k[y][[x]]_{y}} = k[y, y^{-1}][[x]]$.
    Note that $x + y$ is also invertible in $\Gamma(U, \mathcal{O}_{\mathfrak{X}})$ since
    \begin{equation}
        \frac{1}{x + y} = \frac{y^{-1}}{1 + xy^{-1}} = y^{-1}(\sum_{i} (-xy^{-1})^{i})
    \end{equation}
    This coincides with our geometric intuition that $D(y)$ and $D(x + y)$ intersecting with y-axis are both y-axis without origin.
\end{example}
\begin{remark}
    We should be careful with these formal series rings.
    For example, $ k[[x]][y] \subsetneq k[y][[x]]$ since elements in $k[[x]][y]$ would have a bound for degree of $y$,
    while elements in $k[y][[x]]$ is a convergent power series with coeffcients in $k[y]$.
    Thus $\sum_{i} y^{i}x^{i}\in k[y][[x]] \setminus k[[x]][y]$.
    \label{remark:subtle_difference_between_formal_series_rings}
\end{remark}

\subsection{Formal family to formal scheme}
\label{subsection:Algebraization_of_Formal_Moduli_Formal_family_to_formal_scheme}

\begin{definition}
    Let $\{A_{n}\}$ be an inverse system together with homomorphisms $\varphi_{j,i}: A_{j} \rightarrow A_{i}$.
    We say that $\{A_{n}\}$ satisfies Mittag-Leffler condition, (ML) for short, 
    if for all $n$, the decreasing chain 
    \begin{equation}
        \varphi_{n + 1, n}(A_{n + 1}) \supseteq \varphi_{n + 2, n}(A_{n + 2}) \supseteq \cdots 
    \end{equation}
    is stationary.
\end{definition}
\begin{remark}
    By Proposition 9.1 in Algebraic Geometry, Hartshorne, 
    we know that if $0 \rightarrow A_{n} \rightarrow B_{n} \rightarrow C_{n} \rightarrow 0$ is an exact sequence of inverse systems with $\{A_{n}\}$ satisfying (ML),
    then the sequence of inverse limits is also exact
    \begin{equation}
        0 \longrightarrow \inverselimit A_{n} \longrightarrow \inverselimit B_{n} \longrightarrow \inverselimit C_{n} \longrightarrow 0
    \end{equation}
\end{remark}
\begin{proposition}
    Let $(R, \mathfrak{m})$ be a complete local ring with residue field $k$, $X_{0}$ of finite type over $k$.
    Given formal family $\{X_{n}\}$, where $X_{n}$ is deformation of $X_{0}$ over $R/\mathfrak{m}^{n}$.
    Then there exists unique formal scheme $\mathfrak{X}$ with morphism of locally ringed spaces $\mathfrak{X} \rightarrow \spf R$ such that $\mathfrak{X} \times_{\spf R} \spec (R/\mathfrak{m}^{n}) \cong X_{n}$.
    \label{Proposition 5.1}
\end{proposition}
\begin{proof}[\textbf{\emph{Idea}}]
    Given affine open covering $U^{\alpha}$ of $X_{0}$, $B_{n}^{\alpha} := \Gamma(U^{\alpha}, \mathcal{O}_{X_{n}})$ and $B^{\alpha} := \inverselimit B_{n}^{\alpha}$,
    where $B_{n}^{\alpha}$ can be viewed as deformation of $B_{0}^{\alpha}$ over $R/\mathfrak{m}^{n}$.
    We need to check that $B^{\alpha}$ is noetherian and $B^{\alpha}/\mathfrak{m}^{n}B^{\alpha} \cong B_{n}^{\alpha}$.
    \par
    Since $B_{0}^{\alpha}$ is of finite type over $k$, there is a surjection $A_{0} := k[x_{1}, \cdots, x_{N}] \overset{\varphi_{0}}{\twoheadrightarrow} B_{0}^{\alpha}$.
    Similarly, set $A_{i} := R/\mathfrak{m}^{i + 1}[x_{1}, \cdots, x_{N}]$.
    Denote $\psi_{i}: B_{i}^{\alpha} \twoheadrightarrow B_{i - 1}^{\alpha}$.
    Choose one preimage of $\varphi_{0}(x_{i})$ in $B_{1}^{\alpha}$ for each $x_{i}$, 
    define homomorphism $\varphi_{1}: A_{1} \rightarrow B_{1}^{\alpha}$ by mapping $x_{i}$ to corresponding preimage.
    Inductively, we get a commutative diagram
    \begin{equation}
        \begin{tikzcd}
            \cdots \arrow[r, twoheadrightarrow] &
            A_{2} \arrow[r, twoheadrightarrow] \arrow[d, "\varphi_{2}"] &
            A_{1} \arrow[r, twoheadrightarrow] \arrow[d, "\varphi_{1}"] &
            A_{0} \arrow[d, "\varphi_{0}"] \\
            \cdots \arrow[r, twoheadrightarrow] &
            B_{2}^{\alpha} \arrow[r, twoheadrightarrow, "\psi_{2}"] &
            B_{1}^{\alpha} \arrow[r, twoheadrightarrow, "\psi_{1}"] &
            B_{0}^{\alpha} 
        \end{tikzcd}
    \end{equation}
    Consider $A := \inverselimit A_{n} = R\{x_{1}, \cdots, x_{n}\}$ and $I := \inverselimit I_{n}$ where $I_{n} = \ker (\varphi_{i})$,
    want to show that $B^{\alpha} \cong A/I$.
    For all $i$, by flatness of $B_{n}^{\alpha}$ over $R/\mathfrak{m}^{n + 1}$, there is a commutative diagram with exact rows
    \begin{equation}
        \begin{tikzcd}
            0 \arrow[r] & 
            \mathfrak{m}^{n}/\mathfrak{m}^{n + 1} \otimes_{R/\mathfrak{m}^{n + 1}} A_{n} \arrow[r, hookrightarrow] \arrow[d, "\identity \otimes \varphi_{n}"] &
            A_{n} \arrow[r, twoheadrightarrow] \arrow[d, "\varphi_{n}"] &
            A_{n - 1} \arrow[r] \arrow[d, twoheadrightarrow, "\varphi_{n - 1}"] &
            0 \\
            0 \arrow[r] &
            \mathfrak{m}^{n}/\mathfrak{m}^{n + 1} \otimes_{R/\mathfrak{m}^{n + 1}} A_{n} \otimes_{R/\mathfrak{m}^{n + 1}} B_{n}^{\alpha} \arrow[r, hookrightarrow] &
            B_{n}^{\alpha} \arrow[r, twoheadrightarrow, "\psi_{n}"] &
            B_{n - 1}^{\alpha} \arrow[r] &
            0
        \end{tikzcd}
    \end{equation}
    Inductively, assume that we have known $\varphi_{i}$ is surjective for all $i \le n - 1$, want to show that $\varphi_{n}$ is surjective.
    In fact, there is a commutative diagram
    \begin{equation}
        \begin{tikzcd}
            \mathfrak{m}^{n}/\mathfrak{m}^{n + 1} \otimes_{R/\mathfrak{m}^{n + 1}} A_{n} \arrow[r, "\sim"] \arrow[d, "\identity \otimes \varphi_{n}"] &
            \mathfrak{m}^{n} \otimes_{R } A_{0} \arrow[d, "\identity \otimes \varphi_{0}"] \\
            \mathfrak{m}^{n}/\mathfrak{m}^{n + 1} \otimes_{R/\mathfrak{m}^{n + 1}} A_{n} \otimes_{R/\mathfrak{m}^{n + 1}} B_{n}^{\alpha} \arrow[r, "\sim"] &
            \mathfrak{m}^{n} \otimes_{R} B_{0}^{\alpha}
        \end{tikzcd}
    \end{equation}
    Hence $\identity \otimes \varphi_{n} = \identity \otimes \varphi_{0}$ is surjective.
    And by 5 Lemma, we get $\varphi_{n}$ is a surjection.
    Now we have a commutative diagram with exact rows and columns
    \begin{equation}
        \begin{tikzcd}
            && 0 \arrow[d] &
            0 \arrow[d] & \\
            && \mathfrak{m}^{n} \otimes_{R} A_{0} \arrow[r, twoheadrightarrow, "\identity \otimes \varphi_{0}"] \arrow[d, hookrightarrow] &
            \mathfrak{m}^{n} \otimes_{R} B_{0}^{\alpha} \arrow[d, hookrightarrow] & \\
            0 \arrow[r] &
            I_{n} \arrow[r, hookrightarrow] \arrow[d, "\phi_{n}"] &
            A_{n} \arrow[r, twoheadrightarrow, "\varphi_{n}"] \arrow[d, twoheadrightarrow] &
            B_{n}^{\alpha} \arrow[r] \arrow[d, twoheadrightarrow, "\psi_{n}"] &
            0 \\
            0 \arrow[r] &
            I_{n - 1} \arrow[r, hookrightarrow] &
            A_{n - 1} \arrow[r, hookrightarrow, "\varphi_{n - 1}"] &
            B_{n - 1}^{\alpha} \arrow[r] &
            0
        \end{tikzcd}
    \end{equation}
    By Snake Lemma, the following sequence is exact
    \begin{equation}
        \begin{tikzcd}
            \mathfrak{m}^{n} \otimes_{R} A_{0} \arrow[r, twoheadrightarrow, "\identity \otimes \varphi_{0}"] &
            \mathfrak{m}^{n} \otimes_{R} B_{0}^{\alpha} \arrow[r] &
            \coker \phi_{n} \arrow[r] &
            0 
        \end{tikzcd}
    \end{equation}
    which implies that $\phi_{n}$ is surjective for all $n$.
    Thus inverse system $\{I_{n}\}$ satisfies (ML) and hence we get exact sequence
    \begin{equation}
        0 \longrightarrow I \longrightarrow A \longrightarrow B^{\alpha} \longrightarrow 0
    \end{equation}
    so that $B^{\alpha}$ is quotient ring of $A$ hence northerian too.
    By diagram chasing, we also get $B^{\alpha}/\mathfrak{m}^{n}B^{\alpha} \overset{\sim}{\rightarrow} B_{n}^{\alpha}$.
\end{proof}

\subsection{Formal scheme to scheme}
\label{subsection:Algebraization_of_Formal_Moduli_Formal_scheme_to_scheme}

\begin{definition}
    Let $\mathfrak{X} \rightarrow \spf R$ be the noetherian formal scheme given by previous subsection.
    We sat that $\mathfrak{X} \rightarrow \spf R$ is effective if there exists scheme $\widetilde{X}$ flat and of finite type over $\spec R$ such that completion of $\widetilde{X}$ along fiber at $\mathfrak{m}$ is $\mathfrak{X}$.
\end{definition}
\begin{remark}
    This is not always possible. 
    Intuitively, if we take formal completion open covering of $\mathfrak{X}$, denoted by $\{U_{i} = \widehat{X_{i}}\}$,
    then $U_{i}$ and $U_{j}$ are compatible at $U_{i} \cap U_{j}$.
    \par
    However, note that underlying topological spaces of $X_{i}$ and $X_{j}$ are bigger than $\widehat{X_{i}}$ and $\widehat{X_{j}}$,
    it is not necessary for $X_{i}$ and $X_{j}$ to be compatible at $X_{i} \cap X_{j}$ so that we cannot always glue up $X_{i}$ to get our desired $\widetilde{X}$.
\end{remark}
\begin{theorem}[\textbf{\emph{Grothendieck}}]
    Let $X \rightarrow \spec R$ be a proper morphism with $R\in \widehat{\mathcal{C}}$.
    Take $\widehat{X}$ to be the completion of $X$ along the fiber at $\mathfrak{m}$.
    Then $Coh(X) \rightarrow Coh(\widehat{X})$ mapping $\mathcal{F}$ to $\widehat{\mathcal{F}}$ gives equivalence of the two categories.
    \label{Theorem 5.1}
\end{theorem}
\begin{remark}
    This theorem tells us that for deformation of coherent sheaf is always effective.
    In addition, properness is necessary.
    For example, if take $R = k[[x]]$ and $X = \spec k[[x]][y]$, then $\widehat{X} = \spf k[y][[x]]$.
    While 
    \begin{equation}
        \hom_{\mathcal{O}_{X}}(\mathcal{O}_{X}, \mathcal{O}_{X}) = k[[x]][y]\quad \hom_{\mathcal{O}_{\widehat{X}}}(\mathcal{O}_{\widehat{X}}, \mathcal{O}_{\widehat{X}}) = k[y][[x]]
    \end{equation}
    are not same as Remark \ref{remark:subtle_difference_between_formal_series_rings},
    which implies $\mor_{Coh(X)}(\mathcal{O}_{X}, \mathcal{O}_{X}) \neq \mor_{Coh(\widehat{X})}(\mathcal{O}_{\widehat{X}}, \mathcal{O}_{\widehat{X}})$ so that the two categories are not equivalent.
\end{remark}
\begin{theorem}[\textbf{\emph{Grothendieck}}]
    Let $\mathfrak{X} \rightarrow \spf R$ be a proper morphism between formal schemes with $R\in \widehat{\mathcal{C}}$, $X_{0} = \mathfrak{X} \otimes_{\spf R} \spec k$.
    Assume there exists line bundle $\mathcal{L}$ on $\mathfrak{X}$ such that $\mathcal{L}_{0} := \mathcal{L}\big{|}_{X_{0}}$ is ample.
    Then $\mathfrak{X} \rightarrow \spec R$ is effective i.e. there exists of finite type morphism $\widetilde{X} \rightarrow \spec R$ and line bundle $L$ on $\widetilde{X}$ such that $\widetilde{X} \times_{\spec R} \spf R \cong \mathfrak{X}$ and $\widehat{L} \cong \mathcal{L}$.
    \label{Theorem 5.2}
\end{theorem}
\begin{reason}
    Since $\mathcal{L}_{0}$ is ample, we can embed $X$ into a projective space $\mathbb{P}_{k}^{N}$, 
    then the problem of deformation of scheme $X_{0}$ is equivalent to the problem of deformation of closed subscheme $X_{0} \subseteq \mathbb{P}_{k}^{N}$.
    While the deformation of subscheme is equivalent to deformation of ideal sheaf, 
    hence by Theorem \ref{Theorem 5.1}, we get $\mathfrak{X} \rightarrow \spf R$ is effective.
\end{reason}
\begin{example}
    Let $X_{0}$ be a projective $k$-scheme.
    Assume that deformation of line bundles on $X_{0}$ is unobstructed (for example if $H^{2}(X_{0}, \mathcal{O}_{X_{0}}) = 0$).
    Then given ample line bundle $\mathcal{L}_{0}$ on $X_{0}$, 
    we can extend it to compatible $\mathcal{L}_{n}$ to $X_{n}$.
    Taking inevrse limit, we get our desired $\mathcal{L}$ so that in this case, $\mathfrak{X} \rightarrow \spf R$ is effective.
\end{example}
There is also an noneffective example.
\begin{example}
    Let $X_{0} \subseteq \mathbb{P}_{k}^{3}$ be a smooth quartic surface with $\characteristic k = 0$.
    Intuitively, since $H^{2}(X_{0}, \mathcal{O}_{X_{0}}) = k$, deformation of line bundle on $X_{0}$ is not always unobstructed.
    \par
    In addition, computing by Hodge theory, 
    we would see that $H^{0}(X_{0}, \mathcal{N}_{X_{0}/\mathbb{P}_{k}^{3}}) \rightarrow H^{1}(\mathcal{T}_{X_{0}})$ is not surjective,
    and hence not all deformations of $X_{0}$ comes from deformations of $X_{0}$ as subscheme of $\mathbb{P}_{k}^{3}$.
    Thus there exists deformation of $X_{0}$ where $\mathcal{L}_{0} := \mathcal{O}_{\mathbb{P}_{k}^{3}}\big{|}_{X_{0}}$ does not deform to.
    \par
    There is a fact that for all line bundles $\mathcal{L}\in \picardscheme(X_{0})$,
    there exists some 1st order deformation $X'$ of $X_{0}$ over $\spec k[\varepsilon]$ such that $\mathcal{L}_{0}$ does not lift to $X'$.
    While $H^{0}(X_{0}, \mathcal{T}_{X_{0}})= 0$, $H^{2}(X_{0}, \mathcal{T}_{X_{0}}) = 0$ and $H^{1}(X_{0}, \mathcal{T}_{X_{0}}) = k^{20}$ is finite dimensional,
    we get that deformation of $X_{0}$ over $R$ is universal, where $R$ is formal series ring over $k$ with $20$ variants.
    \par
    Suppose that $\mathfrak{X} \rightarrow \spf R$ is effective.
    Assume that $\widetilde{X} \rightarrow \spec R$ is our desired morphism.
    As $\widetilde{X} \rightarrow \spec R$ is flat and proper with geometric fibers smooth, 
    we get $\widetilde{X} \rightarrow \spec R$ itself smooth so that $\widetilde{X}$ is regular.
    \par
    Take affine open subset $U \subseteq X$ meeting $X_{0}$.
    Assume that $D$ is a prime Cartier divisor on $U$ and $D \cap X_{0} \neq \varnothing$.
    Since $\widetilde{X}$ is regular, $\overline{D}$ is a Cartier divisor and $\overline{D} \cap X_{0}$ is an effective Cartier divisor.
    \par
    Take $\mathcal{L}_{0} := \mathcal{O}_{X_{0}}(\overline{D} \cap X_{0})$ which can be lifted to $\widehat{\mathcal{L}_{0}} := \mathcal{O}_{X}(\overline{D})$.
    While by the fact, there exists 1st deformation $X_{1}$ of $X$ such that $\mathcal{L}_{0}$ does not lift to $X_{1}$.
    But by universal property, $\mathcal{L}_{0}$ can be lifted to $\mathcal{L}_{1} := \widehat{\mathcal{L}_{0}}\big{|}_{X_{1}}$ on $X_{1}$, contradiciton!
\end{example}
In the following, we are going to give a proof of the fact we used in the previous example.
Let $X$ be a smooth $k$-scheme, $\mathcal{L}$ line bundle on $X$.
Want to study deformations of pair $(X, \mathcal{L})$.
\begin{definition}
    Let $X$ be a smooth $k$-scheme, $\mathcal{L}$ line bundle on $X$.
    Consider commutative diagram
    \begin{equation}
        \begin{tikzcd}[sep = small]
            X \arrow[rr, "\identity"] \arrow[dr, "\Delta"] \arrow[dd, "\identity"] &&
            X \arrow[dd] \\
            & X \times_{k} X \arrow[ur, "q"  swap] \arrow[dl, "p"] & \\
            X \arrow[rr] &&
            k
        \end{tikzcd}
    \end{equation}
    Define bundle of 1st jets to be $J^{1}(\mathcal{L}) := q_{\ast}(p^{\ast}\mathcal{L} \otimes_{\mathcal{O}_{X \times X}} \mathcal{O}_{X \times X}/\mathcal{I}_{\Delta_{X}}^{2})$.
\end{definition}
\begin{remark}
    Since $X$ is smooth, we have exact sequence
    \begin{equation}
        0 \longrightarrow \Omega_{X}^{1} \otimes_{\mathcal{O}_{X}} \mathcal{L} \overset{\iota}{\longrightarrow} J^{1}(\mathcal{L}) \longrightarrow \mathcal{L} \longrightarrow 0
        \label{equation:exact_sequence_of_bundle_of_1st_jets}
    \end{equation}
    And for all $x\in X$, $J^{1}(\mathcal{L})\big{|}_{x} \cong \mathcal{L}/(\mathfrak{m}_{x})^{2} \cong k^{\dim X + 1}$.
\end{remark}
There exists a canonical section $\mathcal{L} \overset{\sigma}{\rightarrow} J^{1}(\mathcal{L})$ which is not $\mathcal{O}_{X}$-linear.
For $\alpha\in H^{0}(U, \mathcal{L})$ and $f\in H^{0}(U, \mathcal{O}_{X})$, $\sigma(\alpha) := q_{\ast}p^{\ast}(\alpha)$ and
\begin{equation}
    \begin{split}
        \sigma(f\alpha) & = q_{\ast}(f\alpha \otimes 1) \\
        & = q_{\ast}(\alpha \otimes (f \otimes 1)) \\
        & = q_{\ast}(\alpha \otimes (df + 1 \otimes f)) \\
        & = \iota(\alpha \otimes df) + \sigma(\alpha)f
    \end{split}
\end{equation}
satisfying the Leibniz's Law.
\begin{remark}
    Intuitively, the definition of $J^{1}(\mathcal{L})$ is similar to giving a bimodule structure on $\mathcal{O}_{X \times X}/\mathcal{I}_{\Delta_{X}^{2}}$,
    where tensoring with $p_{1}^{\ast}\mathcal{L}$ gives a left module structure and pushing forward gives a right module structure.
\end{remark}
\begin{definition}
    Let $X \rightarrow S$ be a morphism of locally ringed spaces, $\mathcal{M}$ an $\mathcal{O}_{X}$-module.
    A connection of $\mathcal{M}$ over $S$ is an $\mathcal{O}_{S}$-module homomorphism $\nabla: \mathcal{M} \rightarrow \mathcal{M} \otimes \Omega_{X/S}^{1}$ such that $\nabla(am) = a\nabla(m) + m \otimes da$.
\end{definition}
Hence there is a one-to-one correspondence
\vspace{-8pt} % 减少一行间距
\usetikzlibrary{decorations.pathmorphing} %波浪线引用
\begin{equation*}
    \begin{tikzcd}[column sep = tiny]
        \{\mathcal{O}_{X} \text{-linear sections } \mathcal{L} \rightarrow J^{1}(\mathcal{L})\} \arrow[d, leftrightsquigarrow]  &
        \ni & 
        \varphi \arrow[d, mapsto] &
        \sigma - \iota \circ \nabla \\
        \{\nabla: \mathcal{L} \rightarrow \Omega_{X}^{1} \otimes_{\mathcal{O}_{X}} \mathcal{L} \text{ connections}\} &
        \ni & 
        \sigma - \varphi &
        \nabla \arrow[u, mapsto]
    \end{tikzcd}
\end{equation*}
Note that existence of $\mathcal{O}_{X}$-linear section is equivalent to that exact sequence \ref{equation:exact_sequence_of_bundle_of_1st_jets} splits.
Hence the extension class in $\ext_{\mathcal{O}_{X}}^{1}(\mathcal{L}, \Omega_{X}^{1} \otimes_{\mathcal{O}_{X}} \mathcal{L}) \cong H^{1}(X, \Omega_{X}^{1})$,
called the Atiyah class, is the obstruction to existence of connection.
\begin{remark}
    If consider $H^{1}(X, \Omega_{X}^{1}) \rightarrow H^{2}(X, \mathbb{C}) = H^{2}(X, \mathcal{O}_{X}) \oplus H^{1}(X, \Omega_{X}^{1}) \oplus H^{0}(X, \Omega_{X}^{2})$,
    then in fact the Atiyah class is mapping to the 1st Chern class of $X$.
    Hence if the 1st Chern class of $X$ is not zero, then there is no connection.
\end{remark}
\begin{definition}
    Let $X$ be a smooth scheme, $\mathcal{L}$ line bundle on $X$.
    Define the sheaf of principal parts $\mathcal{P}_{\mathcal{L}}$ as an extension
    \begin{equation}
        0 \longrightarrow \mathcal{O}_{X} \longrightarrow \mathcal{P}_{\mathcal{L}} \longrightarrow \mathcal{T}_{X} \longrightarrow 0
    \end{equation}
    defined by the cohomology class $c(\mathcal{L})\in H^{1}(X, \Omega_{X}^{1}) \cong \ext_{\mathcal{O}_{X}}^{1}(\mathcal{T}_{X}, \mathcal{O}_{X})$,
    where $c(\mathcal{L})$ is the image of $[\mathcal{L}]$ under map $H^{1}(X, \mathcal{O}_{X}^{\ast}) \rightarrow H^{1}(X, \Omega_{X}^{1})$ induced by $d\log: \mathcal{O}_{X}^{\ast} \rightarrow \Omega_{X}^{1}$ sending $f$ to $\frac{df}{d}$.
\end{definition}
\begin{remark}
    One way to see that this definition given by Hartshorne is same to our definition of bundle of 1st jets is that the Atiyah class is in fact just $c(\mathcal{L})$.
\end{remark}
Apply functor $\hom_{\mathcal{O}_{X}}(\cdot, \mathcal{L})$ to exact sequence \ref{equation:exact_sequence_of_bundle_of_1st_jets} and we get a long exact sequence
\begin{equation}
    0 \rightarrow \Gamma(X, \mathcal{O}_{X}) \rightarrow \hom_{\mathcal{O}_{X}}(J^{1}(\mathcal{L}), \mathcal{L}) \rightarrow \Gamma(X, \mathcal{T}_{X}) \rightarrow H^{1}(X, \mathcal{O}_{X}) \rightarrow \cdots
\end{equation}
In fact, $\ext_{\mathcal{O}_{X}}^{i}(J^{1}(\mathcal{L}), \mathcal{L}) = H^{i}(X, \mathcal{P}_{\mathcal{L}})$.
And intuitively, automorphism groups of deformations of $(X, \mathcal{L})$ is just $H^{0}(X, \mathcal{P}_{\mathcal{L}} \otimes J)$,
deformations are classified by $H^{1}(X, \mathcal{P}_{\mathcal{L}} \otimes J)$ and the obstruction is in $H^{2}(X, \mathcal{P}_{\mathcal{L}} \otimes J)$.
\par
Want to prove that for all $\mathcal{L} \neq \mathcal{O}_{X}$, 
there exists deformation $X'$ of $X$ such that deformation of $\mathcal{L}$ to $X'$ is obstructed i.e. $H^{1}(X, \mathcal{T}_{X}) \overset{c(\mathcal{L})}{\rightarrow} H^{2}(X, \mathcal{O}_{X})$ is not a zero map,
where $c(\mathcal{L})\in H^{1}(X, \Omega_{X}^{1})$ induces the map by contraction $\mathcal{T}_{X} \otimes \Omega_{X}^{1} \rightarrow \mathcal{O}_{X}$.
\par
By Serre duality, map $c(\mathcal{L})$ above is dual to $\mathbb{C} \cong H^{0}(X, \Omega_{X}^{2}) \overset{\widetilde{c}}{\rightarrow}H^{1}(X, \Omega_{X}^{1} \otimes \Omega_{X}^{2}) \cong H^{1}(X, \Omega_{X}^{1})$ sending $1$ to $c(\mathcal{L})$.
When $\characteristic k = 0$, we have that $H^{1}(X, \mathcal{O}_{X}) = 0$ so that $H^{1}(X, \mathcal{O}_{X}^{\ast}) \rightarrow H^{1}(X, \Omega_{X}^{1})$ is injective.
Hence nontrivial line bundle gives nonzero $c(\mathcal{L})$ and $\widetilde{c}$ is not zero map.
Thus $c(\mathcal{L})$ as dual of $\widetilde{c}$ is not either zero map.

\subsection{Scheme to universal family}
\label{subsection:Algebraization_of_Formal_Moduli_Scheme_to_universal_family}

\begin{theorem}[\textbf{\emph{M.Artin}}]
    Let $X_{0}$ be a projective smooth scheme over $k$.
    Assume that $X_{0}$ admits an effective formal versal deformation $\widetilde{X}$ over complete local $k$-algebra $R$.
    Then there exist flat morphism $X \rightarrow S$ and $s_{0}\in S$ such that $S$ is of finite type over $k$, $R = \widehat{\mathcal{O}_{S, s_{0}}}$, $X \times_{S} \spec R \cong \widetilde{X}$ and $X \times_{S} s_{0} \cong X_{0}$.
\end{theorem}
\begin{example}
    Let $X_{0}$ be smooth projective curve of genus $g \ge 2$.
    We have that $H^{0}(X_{0}, \mathcal{T}_{X_{0}}) = 0$, $H^{1}(X_{0}, \mathcal{T}_{X_{0}}) = k^{3g - 3}$, $H^{2}(X_{0}, \mathcal{T}_{X_{0}}) = 0$ and $H^{2}(X_{0}, \mathcal{O}_{X_{0}}) = 0$.
    Then the first three conditions guarantee that deformation functor of $X_{0}$ has a universal family, 
    and the last condition implies that there is no obstruction to deformation of line bundle.
    Hence deformation of $X_{0}$ is algebraic.
\end{example}
\begin{remark}
    The algebraization $(X, S, s_{0})$ given by Noether's Theorem is only unique locally around $s_{0}$ in the etale topology,
    meaning that if $(X', S', s_{0}')$ is another such pair, then there exists $S''$ with a point $s_{0}''$ and etale morphisms $S' \rightarrow S$ and $S'' \rightarrow S'$ sending $s_{0}''$ to $s_{0}$ and $s_{0}'$ respectively such that $X \times_{s} S'' \cong X' \times_{S'} S''$.
    \par
    In fact, if given $X, Y$ of finite type over $k$ and $x\in X$, $y\in Y$ are $k$-points with $\widehat{\mathcal{O}_{X, x}} \cong \widehat{\mathcal{O}_{Y, y}}$, then
    \begin{equation}
        \begin{tikzcd}[sep = tiny]
            & z\in Z \arrow[dr, "etale"] \arrow[dl, "etale" swap] & \\
            x\in X && y\in Y
        \end{tikzcd}
    \end{equation}
    give same etale cover.
    However, we should be careful about that the isomorphism induced by the same etale cover would not in general be the original one.
\end{remark}
\begin{example}
    Let $X = V(y^{2} - x^{3} - x^{2})$ and $Y = V(uv)$, $x = y = (0, 0)$.
    Then $\widehat{\mathcal{O}_{X, x_{0}}} = \widehat{\mathcal{O}_{Y, y_{0}}}$ and we have a diagram
    \begin{equation}
        \begin{tikzcd}[sep = tiny]
            & k[x, y, t]/(y^{2} - x^{3} - x^{2}, t^{2} - 1 - x) & \\
            k[x, y]/(y^{2} - x^{3} - x^{2}) \arrow[ur] && 
            k[u, v]/(uv) \arrow[ul]
        \end{tikzcd}
    \end{equation}
    where $u$ is mapping to $y - xt$ and $v$ is mapping to $y + xt$.
\end{example}
The remark about local uniqueness up to etale cover inspires us to take isomorphisms into consideration when describing the universal property of algebraization.
Let $X_{0}$ be a smooth projective curve of genus $\ge 2$.
Given another $(X', S', s_{0}')$ such that $S'$ is of finite type over $k$ and $X' \times_{S'} s_{0}' \cong X_{0}$, 
define Isom scheme as following.
\par
Consider projections
\begin{equation}
    \begin{tikzcd}[sep = tiny]
        & S \times S' \arrow[dr, "q"] \arrow[dl, "p" swap] & \\
        S &&
        S'
    \end{tikzcd}
\end{equation}
For $T \rightarrow S \times S'$, we can also define a functor $\isomorphismfunctor(p^{\ast}X, q^{\ast}X')$
\begin{equation*}
    T \mapsto \{(\alpha, \beta, \gamma) \big{|} (\alpha, \beta)\in S \times S'(T) \text{ and } \gamma: \alpha^{\ast}X \overset{\sim}{\rightarrow} \beta^{\ast}X'\}
\end{equation*}
In fact, $\isomorphismfunctor(p^{\ast}X, q^{\ast}X')$ is representable by some scheme $\isomorphismscheme(p^{\ast}X, q^{\ast}X')$.
Note that for any field $K$, $K$-valued points of $\isomorphismscheme(p^{\ast}X, q^{\ast}X')$ would correspond to $\isomorphismfunctor(p^{\ast}X, q^{\ast}X')(\spec K)$.
Hence the set of closed points of $\isomorphismscheme(p^{\ast}X, q^{\ast}X')$ is $\{(y, \theta) \big{|} x\in S \times S' \text{ closed point}, p(y) = s, q(y) = s' \text{ and } \theta: X_{s} \overset{\sim}{\rightarrow} X'_{s'}\}$.
\par
Take $\theta_{0}$ to be the given isomorphism $X_{s_{0}} \overset{\sim}{\rightarrow} X'_{s_{0}'}$.
There is a one-to-one correspondence for all $A\in \mathscr{C}$
\begin{equation*}
    \{\spec A \rightarrow (S', s_{0}')\} \leftrightsquigarrow \{\spec A \rightarrow (\isomorphismscheme(p^{\ast}X, q^{\ast}X'), \theta_{0})\}
\end{equation*}
For each $\alpha': \spec A \rightarrow S'$, consider pull back $\spec A \times_{S'} k(s_{0}')$ which is a deformation of $X_{0}$ over $A$.
By universal property of $\widetilde{X}$, we get there exists $\alpha: \spec A \rightarrow S$ such that $\spec A \times_{S} k(s_{0}) \cong \spec A \times_{S'} k(s_{0}')$.
Denote the isomorphism by $\gamma$ and we get an element $(\alpha, \alpha', \delta)\in \isomorphismfunctor(p^{\ast}X, q^{\ast}X')(\spec A)$.
By property of universal family, we see that this corresponds to $\spec A \rightarrow (\isomorphismscheme(p^{\ast}X, q^{\ast}X'), \theta_{0})$.
\par
Now with the correspondence, as the deformation functor of $X_{0}$ has universal family, $\isomorphismscheme(p^{\ast}X, q^{\ast}X') \rightarrow S'$ is etale near $\theta_{0} \mapsto s_{0}'$.
When $S'$ also has this universal property, then we also get $\isomorphismscheme(p^{\ast}X, q^{\ast}X') \rightarrow X'$ is etale near $\theta_{0} \mapsto s_{0}$ so that the Isom scheme gives our desired locally etale cover.

\section{Derived Category}
\label{section:Derived_Category}

\subsection{Derived category}
\label{subsection:Derived_Category_Derived_category}

\begin{definition}
    Let $\mathscr{A}$ be an abelian category.
    Denote $\mathcal{C}(\mathscr{A})$ to be the set of cochain complexes in $\mathscr{A}$.
    For $A,B\in \mathcal{C}(\mathscr{A})$, a morphism $f: A \rightarrow B$ is called a quasi-isomorphism if $H^{\bullet}(f): H^{\bullet}(A) \overset{\sim}{\rightarrow} H^{\bullet}(B)$ are isomorphic.
\end{definition}
\begin{definition}
    Let $\mathscr{A}$ be an abelian category.
    Define derived category of $\mathscr{A}$ to be $\mathcal{D}(\mathscr{A}) := \{\text{cochain complexes in } \mathscr{A}\}/\text{quasi-isomorphisms}$.
\end{definition}
\begin{remark}
    If $\mathscr{A}$ has enough injective objects, then for each $A\in \mathscr{A}$,
    we can take an injective resolution $0 \rightarrow A \rightarrow I^{\bullet}$ and this would give a quasi-isomorphism
    \begin{equation}
        \begin{tikzcd}
            0 \arrow[r] & 
            A \arrow[r] \arrow[d, hookrightarrow] &
            0 & \\
            0 \arrow[r] &
            I^{0} \arrow[r] &
            I^{1} \arrow[r] & 
            \cdots
        \end{tikzcd}
    \end{equation}
    In addition, recall that for additive functor $\mathcal{F}$, we can define $R^{i}\mathcal{F}(A) := H^{i}(\mathcal{F}(I^{\bullet}))$.
    But taking cohomology would give up a lot of imformation, so we prefer to remember the whole complex.
    \par
    For $A^{\bullet}\in \mathcal{C}^{\ge 0}(\mathscr{A})$, there is a Cartan-Eilenburg resolution
    \begin{equation}
        \begin{tikzcd}
            & 0 \arrow[d] &
            0 \arrow[d] &
            0 \arrow[d] & \\
            0 \arrow[r] & 
            A^{0} \arrow[r] \arrow[d, hookrightarrow] &
            A^{1} \arrow[r] \arrow[d, hookrightarrow] &
            A^{2} \arrow[r] \arrow[d, hookrightarrow] &
            \cdots \\
            0 \arrow[r] &
            I^{0,0} \arrow[r] \arrow[d] &
            I^{1,0} \arrow[r] \arrow[d] &
            I^{2,0} \arrow[r] \arrow[d] &
            \cdots \\
            0 \arrow[r] &
            I^{0,1} \arrow[r] \arrow[d] &
            I^{1,1} \arrow[r] \arrow[d] &
            I^{2,1} \arrow[r] \arrow[d] &
            \cdots \\
            & \cdots &
            \cdots &
            \cdots &
        \end{tikzcd}
    \end{equation}
    where for all $i$, $0 \rightarrow A^{i} \rightarrow I^{i, \ast}$ is injective resolution of $A^{i}$.
    We have that $A^{\bullet} \rightarrow Tot(I^{\bullet, \ast})$ is a quasi-isomorphism and hyper cohomology is defined as $R^{i}\mathcal{F}(A^{\bullet}) := H^{i}(\mathcal{F}(Tot(I^{\bullet, \ast})))$.
    \label{remark:derived_functor_when_there_are_enough_injective_objects}
\end{remark}
\begin{example}
    Let $X$ be a smooth projective variety over $\mathbb{C}$, $X^{\analytification}$ analytification of $X$.
    There is a resolution of $\mathbb{C}$
    \begin{equation}
        0 \rightarrow \mathbb{C}_{X^{\analytification}} \rightarrow \mathcal{O}_{X^{\analytification}} \rightarrow \Omega_{X^{\analytification}}^{1} \rightarrow \Omega_{X^{\analytification}}^{2} \rightarrow \cdots
    \end{equation}
    Even though $\Omega_{X^{\analytification}}^{\bullet}$ is not a injective resolution,
    we can apply hyper cohomology to compute cohomology of $\mathbb{C}$.
    In fact, spectral sequence $E_{1}^{p, q} := H^{q}(X, \Omega_{X}^{p})$ converges to $H^{1}(X, \mathbb{C})$.
    \label{example:compute_cohomology_of_complex_field_with_hyper_cohomology}
\end{example}
Intuitively, we can think that $ob(\mathcal{D}(\mathscr{A})) = ob(\mathcal{C}(\mathscr{A}))$,
while quasi-isomorphisms in $\mathcal{C}(\mathscr{A})$ are identified as isomorphisms in $\mathcal{D}(\mathscr{A})$.
\begin{lemma}
    Let $\mathscr{A}$ be an abelian category and $\mathscr{B}$ be any category, $\mathcal{F}: \mathcal{C}(\mathscr{A}) \rightarrow \mathscr{B}$ functor.
    Assume that for all $f$ quasi-isomorphism, $\mathcal{F}(f)$ is isomorphism in $\mathscr{B}$.
    Then $\mathcal{F}$ factor through unique $\mathcal{G}: \mathcal{D}(\mathscr{A}) \rightarrow \mathscr{B}$.
    \begin{equation}
        \begin{tikzcd}
            \mathcal{C}(\mathscr{A}) \arrow[r] \arrow[dr, "\mathcal{F}"] &
            \mathcal{D}(\mathscr{A}) \arrow[d, dashrightarrow, "\exists! \mathcal{G}"] \\
            & \mathscr{B}
        \end{tikzcd}
    \end{equation}
    \label{Lemma 6.1}
\end{lemma}
\begin{example}
    Let $0 \rightarrow A^{\bullet} \overset{u}{\rightarrow} B^{\bullet} \overset{v}{\rightarrow} C^{\bullet} \rightarrow 0$ be a short exact sequence in $\mathcal{C}(\mathscr{A})$.
    Then we get connecting maps $H^{i}(C^{\bullet}) \rightarrow H^{i + 1}(A^{\bullet})$.
    And there really exists a morphism $C^{\bullet} \rightarrow A^{\bullet}[1]$ in $\mathcal{D}(\mathscr{A})$ given as following
    \begin{equation}
        \begin{tikzcd}
            B^{\bullet} \arrow[r] \arrow[dr] &
            \cone(u) \arrow[r] \arrow[d] & 
            A^{\bullet} \\
            & C^{\bullet} \arrow[ur, dashrightarrow] &
        \end{tikzcd}
    \end{equation}
    where $\cone(u) = B^{\bullet} \oplus A^{\bullet}[1]$ and $\cone(u) \rightarrow C^{\bullet}$ sending $(b, a)$ to $v(b)$ is a quasi-isomorphism.
    \label{example:short_exact_sequence_of_complexes_inducing_a_triangle}
\end{example}
Given a morphism $E \rightarrow F$ in $\mathcal{D}(\mathscr{A})$, by definition, it should be of the following form
\begin{equation*}
    \begin{tikzcd}[sep = tiny]
        E && F \\
        & \bullet \arrow[ur, "chain" swap] \arrow[ul, "qis"]
    \end{tikzcd}
    \text{ or }
    \begin{tikzcd}[sep = tiny]
        E \arrow[dr, "chain" swap] && F \arrow[dl, "qis"] \\
        & \bullet
    \end{tikzcd}
\end{equation*}
so that it is hard to explicitly write composition of morphisms in $\mathcal{D}(\mathscr{A})$ as some chain maps.
To solve this problem, we'd like to adapt a better way to define derived category.
\begin{definition}
    Let $\mathscr{A}$ be an abelian category.
    Define homotopy category $\mathcal{K}(\mathscr{A})$ of $\mathscr{A}$ with same objects as $\mathcal{C}(\mathscr{A})$ and morphism $\hom_{\mathcal{K}(\mathscr{A})}(A^{\bullet}, B^{\bullet}) = \hom_{\mathcal{C}(\mathscr{A})}(A^{\bullet}, B^{\bullet})/\text{homotopy}$.
\end{definition}
\begin{definition}
    A triangle in an additive category with shift functors $A \mapsto A[n]$ is a sequence
    \begin{equation}
        A \longrightarrow B \longrightarrow C \longrightarrow A[1]
    \end{equation}
    inducing a complex
    \begin{equation}
        \cdots \longrightarrow A \longrightarrow B \longrightarrow C \longrightarrow A[1] \longrightarrow B[1] \longrightarrow \cdots
    \end{equation}
    A morphism of of triangles is a commutative diagram
    \begin{equation}
        \begin{tikzcd}
            A \arrow[r] \arrow[d] &
            B \arrow[r] \arrow[d] &
            C \arrow[r] \arrow[d] &
            A[1] \arrow[d] \\
            A' \arrow[r] &
            B' \arrow[r] &
            C' \arrow[r] &
            A'[1]
        \end{tikzcd}
    \end{equation}
    A standard triangle is a triangle of the form
    \begin{equation}
        A \overset{f}{\longrightarrow} B \longrightarrow \cone(f) \longrightarrow A[1]
    \end{equation}
    An exact triangle is a triangle isomorphic to a standard triangle.
\end{definition}
\begin{lemma}
    Let $\mathscr{A}$ be an abelian category, $\mathcal{K}(\mathscr{A})$ homotopy category of $\mathscr{A}$.
    Then for all chain map $f: A^{\bullet} \rightarrow B^{\bullet}$, the following sequence is a triangle in $\mathcal{K}(\mathscr{A})$
    \begin{equation}
        A^{\bullet} \overset{f}{\longrightarrow} B^{\bullet} \overset{i}{\longrightarrow} \cone(f) \overset{p}{\longrightarrow} A^{\bullet}[1]
    \end{equation}
    \label{Lemma 6.2}
\end{lemma}
\vspace{-\baselineskip} % 减少一行间距
\vspace{-\baselineskip} % 减少一行间距
\begin{proof}
    Firstly, it is clear that $p \circ i$ is zero map hence null-homotopic.
    Suffices to show $i \circ f \simeq 0$ and $f[1] \circ p \simeq 0$.
    Take $D: A^{n} \rightarrow \cone(f)^{n - 1}$ sending $a$ to $(0, a)$.
    Then $\partial \circ D + D \circ \partial(a) = (f(a), -\partial a) + (0, \partial a) = i \circ f(a)$ and hence $i \circ f$ is null-homotopic.
    \par
    Take $\widetilde{D}: \cone(f)^{n} \rightarrow B^{\bullet}[1]$ sending $(b, a)$ to $b$.
    Then $\partial \circ \widetilde{D} + \widetilde{D} \circ \partial(b, a) = - \partial b + f(a) + \partial b = f[1] \circ p(b, a)$ and hence $f[1] \circ p$ is null-homotopic.
\end{proof}
\begin{lemma}
    Let $\mathscr{A}$ be an abelian category, $\mathcal{K}(\mathscr{A})$ homotopy category of $\mathscr{A}$.
    Assume that $A^{\bullet} \overset{f}{\rightarrow} B^{\bullet} \overset{i}{\rightarrow} C^{\bullet} \overset{p}{\rightarrow} A^{\bullet}[1]$ is an exact triangle in $\mathcal{K}(\mathscr{A})$.
    Then for any $X^{\bullet}$, there are two long exact sequences
    \begin{equation*}
        \rightarrow \hom_{\mathcal{K}(\mathscr{A})}(X^{\bullet}, A^{\bullet}) \rightarrow \hom_{\mathcal{K}(\mathscr{A})}(X^{\bullet}, B^{\bullet}) \rightarrow \hom_{\mathcal{K}(\mathscr{A})}(X^{\bullet}, C^{\bullet}) \rightarrow \hom_{\mathcal{K}(\mathscr{A})}(X^{\bullet}, A^{\bullet}[1]) \rightarrow
    \end{equation*}
    and
    \begin{equation*}
        \leftarrow \hom_{\mathcal{K}(\mathscr{A})}(A^{\bullet}, X^{\bullet}) \leftarrow \hom_{\mathcal{K}(\mathscr{A})}(B^{\bullet}, X^{\bullet}) \leftarrow \hom_{\mathcal{K}(\mathscr{A})}(C^{\bullet}, X^{\bullet}) \leftarrow \hom_{\mathcal{K}(\mathscr{A})}(A^{\bullet}[1], X^{\bullet}) \leftarrow
    \end{equation*}
    \label{Lemma 6.3}
\end{lemma}
\vspace{-\baselineskip} % 减少一行间距
\vspace{-\baselineskip} % 减少一行间距
\begin{proof}
    Here we only show the exactness of the first sequence at $\hom_{\mathcal{K}(\mathscr{A})}(X^{\bullet}, B^{\bullet})$.
    In fact, we suffice to prove for standard triangles.
    Assume $C = \cone(f)$.
    By Lemma \ref{Lemma 6.2}, the first sequence is obviously a complex.
    For all $\alpha\in \ker (\hom_{\mathcal{K}(\mathscr{A})}(X^{\bullet}, B^{\bullet}) \rightarrow \hom_{\mathcal{K}(\mathscr{A})}(X^{\bullet}, C^{\bullet}))$,
    $g \circ \alpha$ is null-homotopic so that there exists $D: X^{n} \rightarrow C^{n - 1}$ such that $g \circ \alpha = \partial \circ D + D \circ \partial$.
    \par
    Assume that $D = (D', D'')$ i.e. $D(x) = (D'(x), D''(x))$.
    Then we have $(\alpha(x), 0) = (\partial D'(x) + f(D''(x)), -\partial D''(x)) + (D'(\partial x), D''(\partial x))$.
    Hence
    \begin{equation}
        \left\{
            \begin{aligned}
                & \alpha(x) = f(D'(x)) + \partial D'(x) + D'(\partial x) \\
                & \partial D''(x) = D''(\partial x)
            \end{aligned}
        \right.
    \end{equation}
    It is natural to take $\beta: X^{\bullet} \rightarrow A^{\bullet}$ sending $x$ to $D''(x)$.
    And we get $\alpha - f \circ \beta = D' \circ \partial + \partial \circ D'$ so that $\alpha\in \im (\hom_{\mathcal{K}(\mathscr{A})}(X^{\bullet}, A^{\bullet}) \rightarrow \hom_{\mathcal{K}(\mathscr{A})}(X^{\bullet}, B^{\bullet}))$.
\end{proof}
\begin{definition}
    Let $\mathscr{C}$ be a category, $Q$ a class of morphisms in $\mathscr{C}$ is a left ore system if it satisfies the following conditions \\
    (1)$Q$ is multiplicative i.e. $Q \circ Q \subseteq Q$ and $\identity_{C}\in Q$ for all object $C\in \mathscr{C}$. \\
    (2)Every pair of morphism $A' \overset{q}{\leftarrow} A \overset{f}{\rightarrow} B$ with $q\in Q$ can be completed to a commutative diagram
    \begin{equation}
        \begin{tikzcd}
            A \arrow[r, "f"] \arrow[d, "q"] &
            B \arrow[d, "r"] \\
            A' \arrow[r, "g"] &
            B'
        \end{tikzcd}
    \end{equation}
    with $r\in Q$. \\
    (3)If $f \circ q = 0$ with $q\in Q$, then there exists $r\in Q$ such that $r \circ f = 0$.
    \par
    Dually, we can also define right ore system.
\end{definition}
\begin{lemma}
    Let $\mathscr{A}$ be an abelian category.
    Then for all $f$ having homotopic inverse, $f$ is a quasi-isomorphism.
    \label{Lemma 6.4}
\end{lemma}
\begin{remark}
    This lemma tells us that quasi-isomorphism is well defined in $\mathcal{K}(\mathscr{A})$.
\end{remark}
\begin{lemma}
    Let $\mathscr{A}$ be an abelian category, $\mathcal{K}(\mathscr{A})$ homotopy category of $\mathscr{A}$.
    Then the set of quasi-isomorphisms is both left and right ore system of $\mathcal{K}(\mathscr{A})$.
    \label{Lemma 6.5}
\end{lemma}
\begin{proof}
    Here we only show that the set of quasi-isomorphisms is a left ore system.
    Given $E' \overset{q}{\leftarrow} E \overset{g}{\rightarrow} F$, take $F' = \cone((q, g))$.
    Then there is a diagram
    \begin{equation}
        \begin{tikzcd}
            E \arrow[r, "g"] \arrow[d, "q"] &
            F \arrow[d, "r"] \\
            E' \arrow[r, "-h"] &
            F'
        \end{tikzcd}
    \end{equation}
    where $r$ and $g$ are natural inclusions.
    Want to check that $r$ is quasi-isomorphism and the diagram is commutative in the sense up to homotopy.
    \par
    For $[f]\in H^{n}(F)$, $[f]$ is mapping to $[(0, f, 0)]$.
    If $[(0, f, 0)] = [0]$, there exists $(e_{0}', f_{0}, e_{0})$ such that $(0, f, 0) = \partial (e_{0}', f_{0}, e_{0})$.
    Hence
    \begin{equation}
        \left\{
            \begin{aligned}
                & \partial e_{0}' + q(e_{0}) = 0 \\
                & \partial f + g(e_{0}) = f \\
                & \partial e_{0} = 0
            \end{aligned}
        \right.
    \end{equation}
    so that $[q(e_{0})] = 0$.
    Since $q$ is quasi-isomorphism, get $[e_{0}] = 0$ and hence $[f] = 0$.
    Thus $r^{\ast}$ is injective.
    For all $[(e', f, e)]\in H^{n}(F')$, we have that 
    \begin{equation}
        \left\{
            \begin{aligned}
                & \partial e' + q(e) = 0 \\
                & \partial f + g(e) = 0 \\
                & \partial e = 0
            \end{aligned}
        \right.
    \end{equation}
    so that $[q(e)] = 0$.
    Since $q$ is quasi-isomorphism, get $[e] = 0$ and there exists $e_{0}$ such that $\partial e_{0} = e$.
    Hence $[e' + q(e_{0})]\in H^{n}(E')$ and as $q$ is quasi-isomorphism, there exists $e_{1}$ such that $[e' + q(e_{0})] = [q(e_{1})]$.
    Thus there exists $e_{0}'$ such that $\partial e_{0}' = e' + q(e_{0}) - q(e_{1})$ and $\partial (e_{0}', 0, e_{1} - e_{0}) = (e', g(e_{1}) - g(e_{0}), e)$.
    Conclude that $[(e', f, e)] = [(0, f + g(e_{0}) - g(e_{1}), 0)]$ and $r^{\ast}$ is surjective.
    \par
    Want to show that $r \circ g \simeq -h \circ q$.
    For all $e$, $r \circ g + h \circ q(e) = (q(e), g(e), 0)$.
    Take $D: E^{n} \rightarrow {F'}^{n - 1}$ sending $e$ to $(0, 0, e)$.
    Then $D \circ \partial + \partial \circ D(e) = (0, 0, \partial e) + (q(e), g(e), -\partial e) = (q(e), g(e), 0)$.
    Thus $r \circ g \simeq -h \circ q$.
    \par
    Given $E' \overset{q}{\rightarrow} E \overset{g}{\rightarrow} F$ with composition $g \circ q$ null-homotopic.
    Consider $E \overset{i}{\rightarrow} \cone (q)$, by Lemma \ref{Lemma 6.3}, $g$ factors through some $h: \cone (q) \rightarrow F$.
    Take $F' = \cone (h)$ and $j: F \rightarrow F'$.
    Claim that $j$ is quasi-isomorphism and $j \circ g$ is null-homotopic.
    \par
    For all $[f]\in H^{n}(F)$, $[f]$ is mapping to $[(f, 0, 0)]$.
    If $[(f, 0, 0)] = 0$, there exists $(f_{0}, e_{0}, e_{0}')$ such that $\partial (f_{0}, e_{0}, e_{0}') = (f, 0, 0)$.
    Hence
    \begin{equation}
        \left\{
            \begin{aligned}
                & \partial f_{0} + h(e_{0}, e_{0}') = f \\
                & \partial e_{0} + q(e_{0}') = 0 \\
                & \partial e_{0}' = 0
            \end{aligned}
        \right.
    \end{equation}
    so that $[q(e_{0}')] = 0$.
    Since $q$ is quasi-isomorphism, $[e_{0}'] = 0$ so that there exists $e_{1}'$ such that $\partial e_{1}' = e_{0}'$.
    As $\partial (e_{0} + q(e_{1}')) = 0$, there exists $e_{2}'$ and $e_{1}$ such that $\partial e_{1} = e_{0} + q(e_{1}') - q(e_{2}')$.
    Thus $\partial (e_{1}, e_{2}' - e_{1}') = (e_{0}, e_{0}')$ so that $[f] = 0$ and hence $j^{\ast}$ is injective.
    \par
    For all $[(f, e, e')]\in H^{n}(F')$, we have that
    \begin{equation}
        \left\{
            \begin{aligned}
                & \partial f + h(e, e') = 0 \\
                & \partial e + q(e') = 0 \\
                & \partial e' = 0
            \end{aligned}
        \right.
    \end{equation}
    so that $[q(e')] = 0$.
    Since $q$ is quasi-isomorphism, $[e'] = 0$ so that there exists $e_{0}'$ such that $\partial e_{0}' = e'$.
    As $\partial (e + q(e_{0}')) = 0$, there exists $e_{1}'$ and $e_{0}$ such that $\partial e_{0} = e + q(e_{0}') - q(e_{1}')$.
    Thus $\partial (0, e_{0}, e_{1}' - e_{0}') = (h(e_{0}, e_{1}' - e_{0}'), -e, -e')$ so that $[(f, e, e')] = [(f + h(e_{0}, e_{1}' - e_{0}'), 0, 0)]$.
    Conclude that $j^{\ast}$ is surjective.
    \par
    Want to show that $j \circ g$ is null-homotopic.
    For all $e$, $j \circ g(e) = (g(e), 0, 0)$.
    Take $D: E^{n} \rightarrow F'^{n - 1}$ sending $e$ to $(0, e, 0)$.
    Then $D \circ \partial + \partial \circ D(e) = (0, \partial e, 0) + (h(e, 0), -\partial e, 0) = (h(e, 0), 0, 0)$.
    Note that $g(e) = h(e, 0)$, done!
\end{proof}
\begin{remark}
    Now with this lemma, we have a more concise way to interpret composition of morphisms in $\mathcal{D}(\mathscr{A})$ as following
    \begin{equation}
        \begin{tikzcd}
            E_{1} && 
            E_{2} &&
            E_{3} \\
            & \bullet \arrow[ur, "chain" swap] \arrow[ul, "qis"] &&
            \bullet \arrow[ur, "chain" swap] \arrow[ul, "qis"] & \\
            && \bullet \arrow[ur, "chain" swap] \arrow[ul, "qis"] &&
        \end{tikzcd}
    \end{equation}
    And for chain map $f: E \rightarrow F$, $f = 0$ in $\mathcal{D}(\mathscr{A})$ if and only if there exists quasi-isomorphism $q: E' \rightarrow E$ such that $f \circ q \simeq 0$.
\end{remark}
In addition, the following corollary tells us there is a more easy understanding way to say two morphisms in derived category are same.
\begin{corollary}
    Let $\mathscr{A}$ be an abelian category, $A^{\bullet},B^{\bullet}\in \mathcal{C}(\mathscr{A})$.
    Given two different interpretations of same morphism in $\hom_{\mathcal{D}(\mathscr{A})}(B^{\bullet}, A^{\bullet})$
    \begin{equation}
        \begin{tikzcd}
            & C^{\bullet} \arrow[ddr, "f_{1}"] \arrow[ddl, "g_{1}" swap] & \\
            & D^{\bullet} \arrow[dr, "f_{2}" swap] \arrow[dl, "g_{2}"] & \\
            B^{\bullet} &&
            A^{\bullet}
        \end{tikzcd}
    \end{equation}
    where $g_{1}$ and $g_{2}$ are quasi-isomorphisms,
    there exists a quasi-isomorphism $h: C^{\bullet} \rightarrow D^{\bullet}$ such that commutative diagram up to homotopy
    \begin{equation}
        \begin{tikzcd}
            & C^{\bullet} \arrow[ddr, "f_{1}"] \arrow[d, "h"] \arrow[ddl, "g_{1}" swap] & \\
            & D^{\bullet} \arrow[dr, "f_{2}" swap] \arrow[dl, "g_{2}"] & \\
            B^{\bullet} &&
            A^{\bullet}
        \end{tikzcd}
    \end{equation}
    \vspace{-\baselineskip} % 减少一行间距
    \label{Corollary 6.1}
\end{corollary}
\begin{proof}
    By Lemma \ref{Lemma 6.5}, if we take $E^{\bullet} = \cone((f_{2}, g_{2}))$,
    there is a commutative diagram up to homotopy
    \begin{equation}
        \begin{tikzcd}
            & C^{\bullet} \arrow[dr, "f_{1}"] \arrow[dl, "g_{1}" swap] & \\
            B^{\bullet} \arrow[dr] &&
            A^{\bullet} \arrow[dl] \\
            & E^{\bullet} &
        \end{tikzcd}
    \end{equation}
    Then we have chain homotopy $D = D_{1} \oplus D_{2} \oplus D_{3}: C^{\bullet} \rightarrow E^{\bullet}$.
    By definition, we get
    \begin{equation}
        \left\{
            \begin{aligned}
                & f_{1} = D_{1} \circ \partial + \partial \circ D_{1} + f_{2} \circ D_{3} \\ 
                & g_{2} = D_{2} \circ \partial + \partial \circ D_{2} + g_{2} \circ D_{3} \\
                & D_{3} \circ \partial - \partial \circ D_{3} = 0
            \end{aligned}
        \right.
    \end{equation}
    Hence $D_{3}$ is a chain map and just our desired $h$.
    Since $g_{2} \circ h \simeq g_{1}$, we get ${g_{2}}_{\sharp} \circ h_{\sharp} = {g_{1}}_{\sharp}$ as cohomology group homomorphism.
    Note that $g_{1}$ and $g_{2}$ are quasi-isomorphisms, we conclude that $h$ is also a quasi-isomorphism.
\end{proof}
\begin{remark}
    In Corollary \ref{Corollary 6.1}, what we have is only existence.
    Even though by symmetry, there would also exist chain map $h': D^{\bullet} \rightarrow C^{\bullet}$ having similar property.
    The two chain maps in general would not be homotopic inverse of each other. 
\end{remark}
\begin{example}
    Recall in Example \ref{example:short_exact_sequence_of_complexes_inducing_a_triangle}, 
    for a short exact sequence of complexes $0 \rightarrow A^{\bullet} \overset{u}{\rightarrow} B^{\bullet} \overset{v}{\rightarrow} C^{\bullet} \rightarrow 0$,
    there is a triangle up to quasi-isomorphism
    \begin{equation}
        \begin{tikzcd}
            A^{\bullet} \arrow[r, "u"] &
            B^{\bullet} \arrow[r, "v"] \arrow[dr] &
            C^{\bullet} \arrow[r] &
            A^{\bullet}[1] \\
            && \cone(u) \arrow[u, "qis"] \arrow[ur] &
        \end{tikzcd}
    \end{equation}
    If we view $A\in \mathscr{A}$ as a complex $A[0] = [0 \rightarrow A \rightarrow 0]$ where $A$ is at degree $0$,
    then for a short sequence $0 \rightarrow A \overset{u}{\rightarrow} B \overset{v}{\rightarrow} C \rightarrow 0$ in $\mathscr{A}$,
    similarly we have an exact triangle up to quasi-isomorphism
    \begin{equation}
        \begin{tikzcd}
            A[0] \arrow[r, "u"] &
            B[0] \arrow[r, "v"] \arrow[dr] &
            C[0] \arrow[r] &
            A[1] \\
            && \cone(u) \arrow[u, "qis"] \arrow[ur] &
        \end{tikzcd}
    \end{equation}
    In particular, $\cone(u) = [0 \rightarrow A \rightarrow B \rightarrow 0]$ where $B$ is at degree $0$,
    and the map $\cone(u) \rightarrow A[1]$ is zero map in $\mathcal{D}(\mathscr{A})$ if and only if it is zero map in $\mathcal{K}(\mathscr{A})$ if and only if the given short exact sequence splits.
\end{example}
I would like to give a lemma generalizing the last sentence in this example.
\begin{lemma}
    Let $\mathscr{A}$ be an abelian category, morphism $\pi: A \rightarrow B$ in $\mathscr{A}$.
    Then the following chain map is zero in $\mathcal{D}(\mathscr{A})$ if and only if it is zero in $\mathcal{K}(\mathscr{A})$.
    \begin{equation}
        \begin{tikzcd}
            {[A} \arrow[r, "\pi"] \arrow[d, "f"] &
            {B]} \arrow[d] \\
            {[C} \arrow[r] &
            {0]}
        \end{tikzcd}
        \label{equation:chain_map_in_lemma_6.6}
    \end{equation}
    In addition, dually, the following chain map is zero in $\mathcal{D}(\mathscr{A})$ if and only if it is zero in $\mathcal{K}(\mathscr{A})$.
    \begin{equation}
        \begin{tikzcd}
            {[0} \arrow[r] \arrow[d] &
            {C]} \arrow[d, "f"] \\
            {[A} \arrow[r, "\pi"] &
            {B]}
        \end{tikzcd}
    \end{equation}
    \label{Lemma 6.6}
\end{lemma}
\vspace{-\baselineskip} % 减少一行间距
\begin{proof}
    Suffice to show the $\Rightarrow$ arrow.
    Assume chain map \ref{equation:chain_map_in_lemma_6.6} is zero in $\mathcal{D}(\mathscr{A})$,
    then there exists quasi-isomorphism $X^{\bullet} \overset{\varphi}{\rightarrow} [A \overset{\pi}{\rightarrow} B]$ such that their composition is null-homotopic.
    By canonical truncation, we may assume that for all $i \ge 1$, $X^{i} = 0$.
    Hence there exists morphism $D: X_{0} \rightarrow A$ such that the following diagram commutes
    \begin{equation}
        \begin{tikzcd}
            X^{-1} \arrow[r, "\partial_{X}"] \arrow[d, "\varphi_{-1}"] &
            X^{0} \arrow[r] \arrow[d, "\varphi_{0}"] \arrow[ddl, "D" near end] &
            0 \\
            A \arrow[r, "\pi" near start] \arrow[d, "f" swap] &
            B \arrow[r] \arrow[d] & 
            0 \\
            A \arrow[r] &
            0
        \end{tikzcd}
    \end{equation}
    Note that $\varphi$ is a quasi-isomorphism, for all $b\in B$,
    there exists some $x\in X^{0}$ such that $\overline{\varphi_{0}(x)} = \overline{b}$ in $B/A$.
    Hence there exists $a$ such that $\pi(a) = \varphi_{0}(x) - b\in A$.
    Define $s(b) = D(x) - f(a)$.
    \par
    First to check that $s$ is well defined.
    Assume that $x_{1},x_{2}\in X^{0}$ satisfies that $\overline{x_{1}} = \overline{b} = \overline{x_{2}}$.
    Then $x_{1} - x_{2}\in \im \partial_{X}$ so that there exists $x'\in X^{-1}$ mapping to $x_{1} - x_{2}$.
    Suppose $\pi(a_{1}) = \varphi_{0}(x_{1}) - b$ and $\pi(a_{2}) = \varphi_{0}(x_{2}) - b$.
    Note that $\pi(a_{1} - a_{2}) = \varphi_{0}(x_{1} - x_{2}) = \varphi_{0} \circ \partial_{X}(x') = \pi \circ \varphi_{-1}(x')$.
    We get $a_{1} - a_{2} - \varphi_{-1}(x')\in \ker \pi$.
    \par
    As $\varphi$ is a quasi-isomorphism, there exists $x''\in \ker \partial_{X}$ such that $\varphi_{-1}(x'') = a_{1} - a_{2} - \varphi_{-1}(x')$.
    Thus $D(x_{1}) - f(a_{1}) - (D(x_{2}) - f(a_{2})) = D(x_{1} - x_{2}) - f(a_{1} - a_{2}) = f \circ \varphi_{-1}(x') - f(a_{1} - a_{2}) = f \circ \varphi_{-1}(-x'') = D \circ \partial_{X}(-x'') = 0$ so that $s$ is well defined.
    And $s(a) = 0 - f(-a) = f(a)$.
    Conclude that chain map \ref{equation:chain_map_in_lemma_6.6} is null-homotopic.
\end{proof}
\begin{remark}
    Note that when $\pi$ is injective, the cohomology group homomorphism induced by chain map \ref{equation:chain_map_in_lemma_6.6} is always a zero map,
    this is also an example that nonzero morphism in $\mathcal{K}(\mathscr{A})$ induces trivial cohomology group homomorphism.
\end{remark}
\begin{lemma}
    Let $\mathscr{A}$ be an abelian category, $A\in \mathscr{A}$.
    Then for any $B^{\bullet}\in \mathcal{C}^{\le 0}(\mathscr{A})$, we have that $\hom_{\mathcal{D}(\mathscr{A})}(B^{\bullet}, A[0]) = \hom_{\mathcal{C}(\mathscr{A})}(B^{\bullet}, A[0])$.
    Similarly, for any $B^{\bullet}\in \mathcal{C}^{\ge 0}(\mathscr{A})$, we have that $\hom_{\mathcal{D}(\mathscr{A})}(A[0], B^{\bullet}) = \hom_{\mathcal{C}(\mathscr{A})}(A[0], B^{\bullet})$.
    \label{Lemma 6.7}
\end{lemma}
\begin{proof}
    We only prove for the first case, given a morphism in $\hom_{\mathcal{D}(\mathscr{A})}(B^{\bullet}, A[0])$,
    we can interpret it as $B^{\bullet} \overset{qis}{\leftarrow} C^{\bullet} \overset{f}{\rightarrow} A[0]$.
    By truncation $\tau^{\ge 0}$, we may assume that $C^{\bullet}\in \mathcal{C}^{\le 0}(\mathscr{A})$
    Consider the commutative diagram
    \begin{equation}
        \begin{tikzcd}
            C^{-1} \arrow[r, "\partial_{C^{\bullet}}"] \arrow[d] &
            C^{0} \arrow[r] \arrow[d, "f"] &
            0 \\
            0 \arrow[r] &
            A \arrow[r] &
            0
        \end{tikzcd}
    \end{equation}
    As $f \circ \partial_{C^{\bullet}} = 0$, $f$ factors through $\coker \partial_{C^{\bullet}} = H^{0}(C^{\bullet})$.
    Note that $C^{\bullet} \rightarrow B^{\bullet}$ is a quasi-isomorphism, we get $\coker \partial_{B^{\bullet}} = H^{0}(B^{\bullet}) = H^{0}(C^{\bullet})$.
    Hence $f$ induces an element in $\hom_{\mathcal{C}(\mathscr{A})}(B^{\bullet}, A[0])$.
    \par
    Want to show that this map is well defined.
    Given two different interpretations of same morphism in $\hom_{\mathcal{D}(\mathscr{A})}(B^{\bullet}, A[0])$,
    by Corollary \ref{Corollary 6.1}, we have a commutative diagram up to homotopy
    \begin{equation}
        \begin{tikzcd}
            & C^{\bullet} \arrow[ddr, "f_{1}"] \arrow[d, "h"] \arrow[ddl, "g_{1}" swap] & \\
            & D^{\bullet} \arrow[dr, "f_{2}" swap] \arrow[dl, "g_{2}"] & \\
            B^{\bullet} &&
            A[0]
        \end{tikzcd}
    \end{equation}
    where $h$, $g_{1}$ and $g_{2}$ are quasi-isomorphisms.
    Since $\hom_{\mathcal{K}(\mathscr{A})}(C^{\bullet}, A[0]) = \hom_{\mathcal{C}(\mathscr{A})}(C^{\bullet}, A[0])$,
    we immediately get that $f_{1} = f_{2} \circ h$.
    Hence by definition, we have a commutative diagram
    \begin{equation}
        \begin{tikzcd}
            C^{0} \arrow[r, "h"] &
            D^{0} \arrow[r, "f_{2}"] &
            A \\
            \ker \partial_{C^{\bullet}}^{0} \arrow[u, hookrightarrow] \arrow[r, "h"] &
            \ker \partial_{D^{\bullet}}^{0} \arrow[u, hookrightarrow] \arrow[ur] & \\
            C^{-1} \arrow[u, "\partial_{C^{\bullet}}^{-1}"] \arrow[r, "h"] &
            D^{-1} \arrow[u, "\partial_{D^{\bullet}}^{-1}"] &
        \end{tikzcd}
    \end{equation}
    By universal property of cokernel, we get $H^{0}(C^{\bullet}) \overset{h_{\sharp}}{\rightarrow} H^{0}(D^{\bullet}) \rightarrow A$.
    Again by commutativity of the diagram, we get cohomology group homomorphism ${g_{1}}_{\sharp} = {g_{2}}_{\sharp} \circ h_{\sharp}$.
    Hence the map is well defined.
    \par
    Obviously, the map is surjective.
    Remains to show that the map is injective.
    Assume that $B^{\bullet} \overset{qis}{\leftarrow} C^{\bullet} \overset{f}{\rightarrow} A[0]$ is mapping to zero map.
    Then $H^{0}(C^{\bullet}) \rightarrow A$ is zero map so that $f$ is zero map.
    Conclude that the map is bijective.
\end{proof}
\begin{corollary}
    Let $\mathscr{A}$ be an abelian category, $A,B\in \mathscr{A}$.
    Then $\hom_{\mathcal{D}(\mathscr{A})}(A[0], B[0]) = \hom_{\mathscr{A}}(A, B)$.
    \label{Corollary 6.2}
\end{corollary}
\begin{remark}
    This immediately comes from the previous lemma, 
    implying that $\mathcal{D}(\mathscr{A})$ is good since it preserves what it should preserve.
\end{remark}
We end this subsection with an important proposition, which deeply characterizes quasi-isomorphisms.
\begin{proposition}
    Let $f: A^{\bullet} \rightarrow B^{\bullet}$ be a chain map in $\mathcal{C}(\mathscr{A})$.
    Then $f$ is quasi-isomorphism if and only if $\cone(f)$ is acyclic.
    \label{Proposition 6.1}
\end{proposition}
\begin{proof}
    "$\Rightarrow$": Assume $[(b, a)]\in H^{n}(\cone(f))$, then $\partial (b, a) = (\partial b + f(a), -\partial a) = 0$.
    Hence $[f(a)] = 0$ in $H^{n + 1}(B^{\bullet})$.
    Since $f$ is quasi-isomorphism, $[a] = 0$ in $H^{n + 1}(A^{\bullet})$ so there exists $a'\in A^{n}$ mapping to $a$ under boundary map.
    Take $(0, -a')\in \cone(f)^{n - 1}$, we get
    \begin{equation}
        (b, a) - \partial (0, -a') = (b, a) - (-f(a'), a) = (b + f(a'), 0)
    \end{equation}
    Note that $\partial (b + f(a')) = \partial b + f(a) = 0$, we get $[b + f(a')]\in H^{n}(B^{\bullet})$.
    Again since $f$ is quasi-isomorphism, there exists $[a'']\in H^{n}(A^{\bullet})$ such that $[f(a'')] = [b + f(a')]$.
    Hence there exists $b'\in B^{n - 1}$ such that $\partial b' = b + f(a') - f(a'')$.
    Take $(b', a'' - a')$, we get
    \begin{equation}
        \partial (b', a'' - a') = (\partial b' + f(a'') - f(a'), \partial a' - \partial a'') = (b, a)
    \end{equation}
    Hence $[(b, a)] = 0$ in $H^{n}(\cone(f))$ and conclude that $\cone(f)$ is acyclic.
    \par
    "$\Leftarrow$": If $[f(a)] = 0$ in $H^{n}(B^{\bullet})$ for some $[a]\in H^{n}(A^{\bullet})$, then there exists $b\in B^{n - 1}$ such that $\partial b = f(a)$.
    Then $\partial (b, -a) = 0$.
    As $\cone(f)$ is acyclic, there exists $(b', a')\in \cone(f)^{n - 2}$ such that $\partial (b', a') = (b, -a)$.
    Then we get $\partial a' = a$ so that $[a] = 0$ and $f^{\ast}$ is injective.
    \par
    For all $[b]\in H^{n}(B^{\bullet})$, we get $[(b, 0)]\in H^{n}(\cone(f))$.
    As $\cone(f)$ is acyclic, there exists $(b', a)\in \cone(f)^{n - 1}$ such that $\partial (b', a) = (b, 0)$.
    Hence $[a]\in H^{n}(A^{\bullet})$ and $[f(a)] = [b]$.
    Conclude that $f$ is quasi-isomorphism.
\end{proof}
\begin{remark}
    With this proposition, somehow we can much more simplify proof of Lemma \ref{Lemma 6.5}.
\end{remark}

\subsection{Derived functor}
\label{subsection:Derived_Category_Derived_functor}

\begin{definition}
    Let $\mathscr{A}$ be an abelian category.
    A cochain complex $A^{\bullet}\in \mathcal{C}(\mathscr{A})$ is said to be strictly bounded-below (resp. above) if there exists some $n_{0}$ such that for all $i < n_{0}$ (resp. $i > n_{0}$), $A^{i} = 0$.
    Denote $\mathcal{C}^{+}(\mathscr{A})$ (resp. $\mathcal{C}^{-}(\mathscr{A})$) to be the full subcategory of $\mathcal{C}(\mathscr{A})$ of strictly bounded-below (resp. above) complexes.
    \par
    A cochain complex $A^{\bullet}\in \mathcal{C}(\mathscr{A})$ is said to be bounded-below (resp. above) if there exists some $n_{0}$ such that for all $i < n_{0}$ (resp. $i > n_{0}$), $H^{i}(A^{\bullet}) = 0$.
    Denote $\mathcal{D}^{+}(\mathscr{A})$ (resp. $\mathcal{D}^{-}(\mathscr{A})$) to be the set of equivalence classes of bounded-below (resp. above) complexes.
\end{definition}
\begin{remark}
    Clearly, we have that $\mathcal{C}^{+}(\mathscr{A}) = \cup_{n} \mathcal{C}^{\ge n}(\mathscr{A})$ and $\mathcal{C}^{-}(\mathscr{A}) = \cup_{n} \mathcal{C}^{\le n}(\mathscr{A})$.
\end{remark}
\begin{definition}
    Let $\mathscr{A}$ be an abelian category.
    Define canonical truncation $\tau^{\ge n}$ (resp. $\tau^{\le n}$) to be a functor from $\mathcal{C}(\mathscr{A})$ to $\mathcal{C}(\mathscr{A})$
    \begin{equation*}
        A^{\bullet} \longmapsto [0 \rightarrow A^{n}/\im \partial_{n - 1} \rightarrow A^{n + 1} \rightarrow \cdots]
    \end{equation*}
    and respectively
    \begin{equation*}
        A^{\bullet} \longmapsto [\cdots \rightarrow A^{n - 1} \rightarrow \ker \partial_{n} \rightarrow 0]
    \end{equation*}
\end{definition}
\begin{remark}
    There are also notions of naive truncations (or say stupid truncations) which sending $A^{\bullet}$ to $[0 \rightarrow A^{n} \rightarrow A^{n + 1} \rightarrow \cdots]$ or $[\cdots \rightarrow A^{n - 1} \rightarrow A^{n} \rightarrow 0]$.
\end{remark}
\begin{proposition}
    Let $\mathscr{A}$ be an abelian category, $A\in \mathscr{A}$ and $B^{\bullet}\in \mathcal{C}(\mathscr{A})$.
    Then for all $i$, we have that $\hom_{\mathcal{D}(\mathscr{A})}(B^{\bullet}, A[i]) = \hom_{\mathcal{D}(\mathscr{A})}(B^{\ge -i}, A[i])$ and $\hom_{\mathcal{D}(\mathscr{A})}(A[i], B^{\bullet}) = \hom_{\mathcal{D}(\mathscr{A})}(A[i], B^{\le -i})$.
    \label{Proposition 6.2}
\end{proposition}
\begin{proof}
    Here we only prove for the first case.
    We may assume that $i = 0$.
    The natural map $B^{\bullet} \rightarrow B^{\ge 0}$ would induces a homomorphism $\varphi: \hom_{\mathcal{D}(\mathscr{A})}(B^{\ge 0}, A[0]) \rightarrow \hom_{\mathcal{D}(\mathscr{A})}(B^{\bullet}, A[0])$.
    Want to show that $\varphi$ is bijective.
    \par
    For injectivity, suppose that $B^{\ge 0} \rightarrow C^{\bullet} \overset{qis}{\leftarrow} A[0]$ is mapping to zero under $\varphi$.
    Then there is a commutative diagram
    \begin{equation}
        \begin{tikzcd}
            B^{\bullet} \arrow[d] &
            D^{\bullet} \arrow[dd, "0", bend left] \arrow[d] \arrow[l, "qis"] \\
            B^{\ge 0} \arrow[d] &
            D^{\ge 0} \arrow[l, "qis"] \\
            C^{\bullet} &
            A[0] \arrow[l, "qis"]
        \end{tikzcd}
    \end{equation}
    Hence there exists chain homotopy $\psi: D^{\bullet} \rightarrow C^{\bullet}$ such that $\psi \circ \partial + \partial \circ \psi = D^{\bullet} \overset{qis}{\rightarrow} B^{\bullet} \rightarrow B^{\ge 0} \rightarrow C^{\bullet}$.
    Note that $A[0] \rightarrow C^{\bullet}$ is a quasi-isomorphism,
    for all $j < 0$, $C^{\bullet}$ is exact at degree $j$.
    Hence natural map $C^{\bullet} \rightarrow C^{\ge 0}$ is a quasi-isomorphism and we get another interpretation of the morphism $B^{\ge 0} \rightarrow A[0]$.
    Consider the following diagram
    \begin{equation}
        \begin{tikzcd}
            \cdots \arrow[r] & 
            D^{-1} \arrow[r] \arrow[d, "0"] &
            D^{0} \arrow[r] \arrow[d] \arrow[dl, "\psi_{0}"] &
            D^{1} \arrow[r] \arrow[d] \arrow[dl, "\psi_{1}"] &
            \cdots \\
            \cdots \arrow[r] &
            C^{-1} \arrow[r] \arrow[d] &
            C^{0} \arrow[r] \arrow[d, twoheadrightarrow] &
            C^{1} \arrow[r] \arrow[d, equal] &
            \cdots \\
            \cdots \arrow[r] &
            0 \arrow[r] &
            C^{0}/\im \partial \arrow[r] &
            C^{1} \arrow[r] &
            \cdots
        \end{tikzcd}
    \end{equation}
    As the zero map $D^{\bullet} \rightarrow A[0]$ factors through $D^{\ge 0}$,
    we get
    \begin{equation}
        \begin{tikzcd}
            \cdots \arrow[r] & 
            0 \arrow[r] \arrow[d] &
            D^{0}/ \im \partial \arrow[r] \arrow[d] \arrow[dl] &
            D^{1} \arrow[r] \arrow[d] \arrow[dl, "\widetilde{\psi}_{1}"] &
            \cdots \\
            \cdots \arrow[r] &
            0 \arrow[r] &
            C^{0}/\im \partial \arrow[r] &
            C^{1} \arrow[r] &
            \cdots
        \end{tikzcd}
    \end{equation}
    Check that $\widetilde{\psi}$ is a chain homotopy between $D^{\ge 0} \rightarrow C^{\ge 0}$ and zero map.
    For $j \ge 2$, $\widetilde{\psi}_{j} = \psi_{j}$ hence ok.
    For $j = 1$, $\partial_{C^{\ge 0}} \circ \widetilde{\psi}_{1} = \partial_{C^{\bullet}} \circ \psi_{0}$ hence ok.
    For $j = 0$, as $\im (\partial \circ \psi) \subseteq \im \partial$, 
    for all $\overline{a}\in D^{0}/\im \partial$, $\widetilde{\psi}_{1} \circ \partial(\overline{a}) = \overline{\psi_{1}(a)}$ is just the the image of $[a]$ under the chain map hence ok.
    Conclude that the morphism $B^{\bullet} \rightarrow A[0]$ is zero in $\hom_{\mathcal{D}(\mathscr{A})}(B^{\ge 0}, A[0])$.
    \par
    For surjectivity, for any morphism $B^{\bullet} \overset{qis}{\leftarrow} D^{\bullet} \rightarrow A[0]$,
    since $D^{\bullet} \rightarrow A[0]$ would factor through $D^{\ge 0}$, we get commutative diagram
    \begin{equation}
        \begin{tikzcd}
            B^{\bullet} \arrow[d] &
            D^{\bullet} \arrow[dr] \arrow[d] \arrow[l, "qis"] & \\
            B^{\ge 0} \arrow[dr] &
            D^{\ge 0} \arrow[r] \arrow[l, "qis"] &
            A[0] \arrow[dl, "qis"] \\
            & C^{\bullet} &
        \end{tikzcd}
    \end{equation}
    where $C^{\bullet}$ is given by Lemma \ref{Lemma 6.5}.
    By commutativity, we get $B^{\bullet} \overset{qis}{\leftarrow} D^{\bullet} \rightarrow A[0]$ is the image of $B^{\ge 0} \overset{qis}{\leftarrow} D^{\ge 0} \rightarrow A[0]$ under $\varphi$, done!
\end{proof}
\begin{lemma}
    Let $\mathscr{A}$ be an abelian category.
    Then $\mathcal{D}^{+}(\mathscr{A})$ (resp. $\mathcal{D}^{-}(\mathscr{A})$) can be identified with the localization of $\mathcal{C}^{+}(\mathscr{A})$ (resp. $\mathcal{C}^{-}(\mathscr{A})$) by quasi-isomorphisms.
    \label{Lemma 6.8}
\end{lemma}
\begin{reason}
    Linguistically, for all $A^{\bullet}\in \mathcal{D}^{+}(\mathscr{A})$, 
    we only need to find a quasi-isomorphism $A^{\bullet} \rightarrow B^{\bullet}$ with $B^{bullet}$ strictly bounded-below.
    Since $A^{\bullet}\in \mathcal{D}^{+}(\mathscr{A})$, there exists some $n_{0}$ such that for all $i < n_{0}$, $H^{i}(A^{\bullet}) = 0$.
    It is natural to consider canonical truncation $\tau^{\ge n_{0} - 1}$.
    There is a natural chain map
    \begin{equation}
        \begin{tikzcd}
            \cdots \arrow[r] &
            A^{n_{0} - 2} \arrow[r] \arrow[d] &
            A^{n_{0} - 1} \arrow[r] \arrow[d] &
            A^{n_{0}} \arrow[r] \arrow[d] &
            \cdots \\
            \cdots \arrow[r] &
            0 \arrow[r] &
            A^{n_{0} - 1}/\im \partial_{n_{0} - 2} \arrow[r] &
            A^{n_{0}} \arrow[r] &
            \cdots
        \end{tikzcd}
    \end{equation}
    Clearly, this is a quasi-isomorphism and $\tau^{\ge n_{0} - 1}(A^{\bullet})$ is strictly bounded-below.
\end{reason}
Recall that in Remark \ref{remark:derived_functor_when_there_are_enough_injective_objects},
when $\mathscr{A}$ has enough injective objects,
we have defined derived functors both for objects in $\mathscr{A}$ and complexes in $\mathcal{C}^{+}(\mathscr{A})$.
And same idea works for strictly bounded-below complex in $\mathcal{C}^{+}(\mathscr{A})$. 
But what's about the case that $\mathscr{A}$ does not have enough injective objects?
In the following, we would give a sufficient condition for existence of derived functor.
\begin{definition}
    Let $\mathscr{A}, \mathscr{B}$ be abelian categories, $\mathcal{F}: \mathscr{A} \rightarrow \mathscr{B}$ an additive functor, $\mathscr{A}' \subseteq \mathscr{A}$ full additive subcategory.
    We say that $\mathscr{A}'$ satisfies $\mathcal{F}$-condition if \\
    (1)for all $A\in \mathscr{A}$, there is an injection $A \hookrightarrow A'$ with $A'\in \mathscr{A}'$. \\
    (2)for any short exact sequence $0 \rightarrow A \rightarrow B \rightarrow C \rightarrow 0$ in $\mathscr{A}$,
    if $A$ and $B$ are both in $\mathscr{A}'$, then $C$ is also in $\mathscr{A}'$. \\
    (3)$\mathcal{F}$ is an exact functor on $\mathscr{A}'$.
\end{definition}
Now given $\mathcal{F}$ and $\mathscr{A}' \subseteq \mathscr{A}$ satisfying $\mathcal{F}$-condition,
by definition, for all $A\in \mathscr{A}$, we can find a $\mathcal{F}$-resolution $J^{\bullet}$ of $A$.
Similar to the case when $\mathscr{A}$ has enough injective objects, we can just define $R^{i}\mathcal{F}(A) = H^{i}(\mathcal{F}(J^{\bullet}))$.
\par
For complexes, by same strategy of construction of Cartan-Eilenburg resolution, there exists resolution $\mathcal{J}^{\bullet, \ast}$ for all $A^{\bullet}\in \mathcal{C}^{+}(\mathscr{A})$. 
Hence we can also define hyper cohomology $R^{i}\mathcal{F}(A^{\bullet}) = H^{i}(\totalcomplex (\mathcal{J}^{\bullet, \ast}))$.
And in fact, we would get a quasi-isomorphism $A^{\bullet} \rightarrow \totalcomplex (\mathcal{J}^{\bullet, \ast})$.
To define derived functor on derived category, consider the following diagram
\begin{equation}
    \begin{tikzcd}
        \mathcal{K}^{+}(\mathscr{A}) \arrow[r, "\mathcal{F}"] \arrow[d] &
        \mathcal{K}^{+}(\mathscr{B}) \arrow[d] \\
        \mathcal{D}^{+}(\mathscr{A}) \arrow[r, dashrightarrow, "R\mathcal{F}"] &
        \mathcal{D}^{+}(\mathscr{B})
    \end{tikzcd}
\end{equation}
where $\mathcal{K}^{+}(\mathscr{A})$ (resp. $\mathcal{K}^{-}(\mathscr{A})$) is the homotopy category of strictly bounded-below (resp. above) complexes.
And it is natural to define $R\mathcal{F}(A^{\bullet}) = \mathcal{F}(\totalcomplex (\mathcal{J})^{\bullet, \ast})$.
To show this definition is well defined and really a functor, we need following lemmas. 
\begin{lemma}
    Let $A^{\bullet} \rightarrow B^{\bullet}$ be a chain map in $\mathcal{C}^{+}(\mathscr{A})$.
    By argument above, there are two quasi-isomorphisms $A^{\bullet} \rightarrow A'^{\bullet}$ and $B^{\bullet} \rightarrow B'^{\bullet}$ with $A'^{\bullet},B'^{\bullet}\in \mathcal{C}(\mathscr{A}')$.
    Then there exists morphism $A'^{\bullet} \rightarrow B'^{\bullet}$ in $\mathcal{D}(\mathscr{A}')$ making the following diagram commutes
    \begin{equation}
        \begin{tikzcd}
            A^{\bullet} \arrow[r, "qis"] \arrow[d] &
            A'^{\bullet} \arrow[d] \\
            B^{\bullet} \arrow[r, "qis"] &
            B'^{\bullet}
        \end{tikzcd}
    \end{equation}
    Moreover, the morphism is independent to the choice of quasi-isomorphisms.
    \label{Lemma 6.9}
\end{lemma}
\begin{proof}
    Applying Lemma \ref{Lemma 6.5} (1), there exists $C^{\bullet}\in \mathcal{C}(\mathscr{A})$ such that the following diagram commutes up to homotopy
    \begin{equation}
        \begin{tikzcd}
            A^{\bullet} \arrow[r, "qis"] \arrow[d] &
            A'^{\bullet} \arrow[d] \\
            B^{\bullet} \arrow[r, "qis"] \arrow[d, "qis"] &
            C^{\bullet} \\
            B'^{\bullet} &
        \end{tikzcd}
    \end{equation}
    Note that $C^{\bullet}$ is also bounded-below, by Cartan-Eilenburg resolution, we can replace $C^{\bullet}$ by $C'^{\bullet}\in \mathcal{C}(\mathscr{A}')$ with $C^{\bullet} \overset{qis}{\rightarrow} C'^{\bullet}$.
    Again apply Lemma \ref{Lemma 6.5} (1), there exists $D^{\bullet}\in \mathcal{C}(\mathscr{A}')$ such that the following diagram commutes up to homotopy
    \begin{equation}
        \begin{tikzcd}
            A^{\bullet} \arrow[r, "qis"] \arrow[d] &
            A'^{\bullet} \arrow[d] \\
            B^{\bullet} \arrow[r, "qis"] \arrow[d, "qis"] &
            C^{\bullet} \arrow[d] \\
            B'^{\bullet} \arrow[r, "qis"] &
            D^{\bullet}
        \end{tikzcd}
    \end{equation}
    Hence we get a morphism $A'^{\bullet} \rightarrow B'^{\bullet}$ in $\mathcal{D}(\mathscr{A}')$.
    Independentness is clear.
\end{proof}
\begin{lemma}
    Let $A^{\bullet} \rightarrow B^{\bullet}$ be a quasi-isomorphism in $\mathcal{C}(\mathscr{A}')$.
    Then $\mathcal{F}(A^{\bullet}) \rightarrow \mathcal{F}(B^{\bullet})$ is also a quasi-isomorphism.
    \label{Lemma 6.10}
\end{lemma}
\begin{reason}
    As $\mathcal{F}$ is exact on $\mathscr{A}'$, it preserves kernel, image and hence cohomology group.
\end{reason}
\begin{remark}
    This lemma tells us we can well define $\mathcal{F}$ on $\mathcal{D}(\mathscr{A}')$.
\end{remark}
\begin{lemma}
    $R\mathcal{F}$ is well defined.
    \label{Lemma 6.11}
\end{lemma}
\begin{proof}
    Assume that there are two quasi-isomorphisms $f: A^{\bullet} \rightarrow A'^{\bullet}$ and $g: A^{\bullet} \rightarrow A''^{\bullet}$ with $A'^{\bullet},A''^{\bullet}\in \mathcal{C}(\mathscr{A}')$.
    Similar to proof of Lemma \ref{Lemma 6.5} (1), taking $A'''^{\bullet} = \cone((f, g))$, we get a commutative diagram
    \begin{equation}
        \begin{tikzcd}
            A^{\bullet} \arrow[r, "g"] \arrow[d, "f"] &
            A''^{\bullet} \arrow[d, "\beta"] \\
            A'^{\bullet} \arrow[r, "-\alpha"] &
            A'''^{\bullet} 
        \end{tikzcd}
    \end{equation}
    where $\alpha$ and $\beta$ are natural inclusions.
    By same argument as proof of Lemma \ref{Lemma 6.5} (1), $\alpha$ and $\beta$ are both quasi-isomorphism.
    By Lemma \ref{Lemma 6.10}, $\mathcal{F}(A'^{\bullet}) = \mathcal{F}(A'''^{\bullet}) = \mathcal{F}(A''^{\bullet})$ in $\mathcal{D}(\mathscr{A})$,
\end{proof}
Recall that between triangulated categories, a triangulated functor (or say exact functor) is a functor respecting the triangulated structure i.e it commutes with shift functor and preserves exact triangles.
Then for any additive functor $\mathcal{F}: \mathscr{A} \rightarrow \mathscr{B}$, the induced functor $\mathcal{F}: \mathcal{K}(\mathscr{A}) \rightarrow \mathcal{K}(\mathscr{B})$ is triangulated.
\par
With lemmas above, we can define a natural transformation
\begin{equation}
    \begin{tikzcd}
        \mathcal{K}^{+}(\mathscr{A}) \arrow[r, "\mathcal{F}"] \arrow[d] &
        \mathcal{K}^{+}(\mathscr{B}) \arrow[d] \arrow[dl, Rightarrow, "\varepsilon" swap] \\
        \mathcal{D}^{+}(\mathscr{A}) \arrow[r, "R\mathcal{F}"] &
        \mathcal{D}^{+}(\mathscr{B})
    \end{tikzcd}
\end{equation}
For all $A^{\bullet}\in \mathcal{K}^{+}(\mathscr{A})$, take quasi-isomorphism $A^{\bullet} \rightarrow A'^{\bullet}$ with $A'^{\bullet}\in \mathcal{C}(\mathscr{A}')$.
The morphism $\varepsilon(A^{\bullet})$ in $\mathcal{D}(\mathscr{A})$ is just the morphism in $\mathcal{K}(\mathscr{A})$ induced by $A^{\bullet} \overset{qis}{\rightarrow} A'^{\bullet}$.
And for all morphism $f: A^{\bullet} \rightarrow B^{\bullet}$ in $\mathcal{C}^{+}(\mathscr{A})$, there is a commutative diagram 
\begin{equation}
    \begin{tikzcd}
        \mathcal{F}(A^{\bullet}) \arrow[r, "\varepsilon(A^{\bullet})"] \arrow[d, "\mathcal{F}f"] &
        R\mathcal{F}(A^{\bullet}) = \mathcal{F}(A'^{\bullet}) \arrow[d, "\mathcal{F}\widetilde{f}"] \\
        \mathcal{F}(B^{\bullet}) \arrow[r, "\varepsilon(B^{\bullet})"] &
        R\mathcal{F}(B^{\bullet}) = \mathcal{F}(B'^{\bullet})
    \end{tikzcd}
\end{equation}
where $\widetilde{f}$ is the morphism in $\mathcal{D}(\mathscr{A}')$ given by Lemma \ref{Lemma 6.9}.
\par
In addition, $R\mathcal{F}$ is an exact functor and $(R\mathcal{F}, \varepsilon)$ has universal property as following.
Given exact functor $\mathcal{G}: \mathcal{D}^{+}(\mathscr{A}) \rightarrow \mathcal{D}^{+}(\mathscr{B})$ and a natural transformation
\begin{equation}
    \begin{tikzcd}
        \mathcal{K}^{+}(\mathscr{A}) \arrow[r, "\mathcal{F}"] \arrow[d] &
        \mathcal{K}^{+}(\mathscr{B}) \arrow[d] \arrow[dl, Rightarrow, "{\varepsilon'}" swap] \\
        \mathcal{D}^{+}(\mathscr{A}) \arrow[r, "\mathcal{G}"] &
        \mathcal{D}^{+}(\mathscr{B})
    \end{tikzcd}
\end{equation}
there exists unique natural transformation $\pi: R\mathcal{F} \rightarrow \mathcal{G}$ such that $\varepsilon' = \pi \circ \varepsilon$.
\begin{example}
    Let $X$ be a smooth projective complex manifold.
    As Example \ref{example:compute_cohomology_of_complex_field_with_hyper_cohomology}, there is a resolution of $\mathbb{C}$
    \begin{equation}
        0 \rightarrow \mathbb{C}_{X} \rightarrow \mathcal{O}_{X} = \Omega_{X}^{0} \rightarrow \Omega_{X}^{1} \rightarrow \Omega_{X}^{2} \rightarrow \cdots
    \end{equation}
    Hence $\mathcal{C}_{X}[0] \rightarrow \Omega_{X}^{\bullet}$ is a quasi-isomorphism and $R\Gamma(X, \mathbb{C}_{X}[0]) = R\Gamma(X, \Omega_{X}^{\bullet})$.
    \par
    Algebraically, the same complex is not still exact.
    However, by Serre's GAGA, we claim that $R\Gamma(X^{\operatorname{alg}}, \Omega_{X^{\operatorname{alg}}}^{\bullet}) = R\Gamma(X^{\analytification}, \Omega_{X^{\analytification}}^{\bullet})$.
\end{example}

\subsection{Total hom functor}
\label{subsection:Derived_Category_Total_hom_functor}

\begin{definition}
    Let $\mathscr{A}$ be an abelian category, $A^{\bullet},B^{\bullet}\in \mathcal{C}(\mathscr{A})$.
    Define total hom complex $\hom^{\bullet}(A^{\bullet}, B^{\bullet})$, where $\hom^{p}(A^{\bullet}, B^{\bullet}) = \prod_{n} \hom_{\mathscr{A}}(A^{n}, B^{n + p})$.
    In fact, this is the total complex of the following diagram
    \begin{equation}
        \begin{tikzcd}
            & \cdots \arrow[d] &
            \cdots \arrow[d] & \\ 
            \cdots \arrow[r] &
            \hom_{\mathscr{A}}(A^{n}, B^{n + p}) \arrow[r, "\circ \partial_{A}^{n - 1}"] \arrow[d, "\partial_{B}^{n + p} \circ"] &
            \hom_{\mathscr{A}}(A^{n - 1}, B^{n + p}) \arrow[r] \arrow[d, "\partial_{B}^{n + p} \circ"] &
            \cdots \\
            \cdots \arrow[r] &
            \hom_{\mathscr{A}}(A^{n}, B^{n + p + 1}) \arrow[r, "\circ \partial_{A}^{n - 1}"] \arrow[d] &
            \hom_{\mathscr{A}}(A^{n - 1}, B^{n + p + 1}) \arrow[r] \arrow[d] &
            \cdots \\
            & \cdots &
            \cdots &
        \end{tikzcd}
    \end{equation}
    with boundary map given by $f \mapsto \partial_{B}^{n + p} \circ f - (-1)^{p}f \circ \partial_{A}^{n - 1}$.
\end{definition}
Fixed $A^{\bullet}$, we would get a covariant functor $\hom^{\bullet}(A^{\bullet}, \cdot)$, called total hom functor.
While total hom functor does not comes from some functor $\mathcal{F}: \mathscr{A} \rightarrow \abeliangroups$,
to define derived functor of it, we need generalize our argument in subsection \ref{subsection:Derived_Category_Derived_functor}.
\begin{definition}
    Let $\mathscr{A}, \mathscr{B}$ be abelian categories, $\mathcal{F}: \mathcal{K}(\mathscr{A}) \rightarrow \mathcal{K}(\mathscr{B})$ a triangulated functor, $\mathcal{K}' \subseteq \mathcal{K}(\mathscr{A})$ full triangulated subcategory.
    We say that $\mathcal{K}'$ satisfies $\mathcal{F}$-condition if \\
    (1)for all $A^{\bullet}\in \mathcal{K}(\mathscr{A})$, there is a quasi-isomorphism $A^{\bullet} \rightarrow A'^{\bullet}$ with $A'^{\bullet}\in \mathcal{K}'$. \\
    (2)for all $A^{\bullet}\in \mathcal{K}'$ acyclic, $\mathcal{F}A^{\bullet}$ is also acyclic.
\end{definition}
\begin{remark}
    Here, we directly ask quasi-isomorphism exist.
    Hence we do not need the strictly bounded-below condition, which guarantees that we can take resolution in the past.
\end{remark}
Now given $\mathcal{F}$ and $\mathcal{K}' \subseteq \mathcal{K}(\mathscr{A})$ satisfying $\mathcal{F}$-condition,
by definition, for all $A\in \mathcal{K}(\mathscr{A})$, there is a quasi-isomorphism $A^{\bullet} \rightarrow A'^{\bullet}$ with $A'^{\bullet}\in \mathcal{K}'$.
To define derived functor on derived category, consider the following diagram
\begin{equation}
    \begin{tikzcd}
        \mathcal{K}(\mathscr{A}) \arrow[r, "\mathcal{F}"] \arrow[d] &
        \mathcal{K}(\mathscr{B}) \arrow[d] \\
        \mathcal{D}(\mathscr{A}) \arrow[r, dashrightarrow, "R\mathcal{F}"] &
        \mathcal{D}(\mathscr{B})
    \end{tikzcd}
\end{equation}
And it is natural to define $R\mathcal{F}(A^{\bullet}) = \mathcal{F}(A'^{\bullet})$.
To show this definition is well defined and really a functor, we need following generalized lemmas,
and most of the proofs are same as their original version and we only provide proof of Lemma \ref{Lemma 6.13} here.
\begin{lemma}
    Let $A^{\bullet} \rightarrow B^{\bullet}$ be a chain map in $\mathcal{C}(\mathscr{A})$.
    By definition, there are two quasi-isomorphisms $A^{\bullet} \rightarrow A'^{\bullet}$ and $B^{\bullet} \rightarrow B'^{\bullet}$ with $A'^{\bullet},B'^{\bullet}\in \mathcal{K}'$.
    Then there exists morphism $A'^{\bullet} \rightarrow B'^{\bullet}$ in $\mathcal{D}(\mathcal{K}')$ making the following diagram commutes
    \begin{equation}
        \begin{tikzcd}
            A^{\bullet} \arrow[r, "qis"] \arrow[d] &
            A'^{\bullet} \arrow[d] \\
            B^{\bullet} \arrow[r, "qis"] &
            B'^{\bullet}
        \end{tikzcd}
    \end{equation}
    where $\mathcal{D}(\mathcal{K}')$ is the localization of $\mathcal{K}'$ over quasi-isomorphisms.
    Moreover, the morphism is independent to the choice of quasi-isomorphisms.
    \label{Lemma 6.12}
\end{lemma}
\begin{lemma}
    Let $A^{\bullet} \rightarrow B^{\bullet}$ be a quasi-isomorphism in $\mathcal{K}'$.
    Then $\mathcal{F}(A^{\bullet}) \rightarrow \mathcal{F}(B^{\bullet})$ is also a quasi-isomorphism.
    \label{Lemma 6.13}
\end{lemma}
\begin{proof}
    Firstly, by Proposition \ref{Proposition 6.1}, $\cone(f)$ is acyclic.
    By definition, we know $\mathcal{F}\cone(f)$ is still acyclic.
    Consider exact triangle
    \begin{equation}
        A^{\bullet} \overset{f}{\longrightarrow} B^{\bullet} \longrightarrow \cone(f) \longrightarrow A^{\bullet}[1] 
    \end{equation}
    As $\mathcal{F}$ is a triangulated functor, applying $\mathcal{F}$, we get an exact triangle
    \begin{equation}
        \mathcal{F}A^{\bullet} \overset{\mathcal{F}f}{\longrightarrow} \mathcal{F}B^{\bullet} \longrightarrow \mathcal{F}\cone(f) \longrightarrow \mathcal{F}A^{\bullet}[1] 
    \end{equation}
    By axiom ($TR3$) of triangulated category, there is a commutative diagram in $\mathcal{K}'$
    \begin{equation}
        \begin{tikzcd}
            \mathcal{F}A^{\bullet} \arrow[r, "\mathcal{F}f"] \arrow[d, equal] &
            \mathcal{F}B^{\bullet} \arrow[r] \arrow[d, equal] &
            \cone(\mathcal{F}f) \arrow[r] \arrow[d, "\varphi"] &
            \mathcal{F}A^{\bullet}[1] \arrow[d, equal] \\
            \mathcal{F}A^{\bullet} \arrow[r, "\mathcal{F}f"] &
            \mathcal{F}B^{\bullet} \arrow[r] &
            \mathcal{F}\cone(f) \arrow[r] &
            \mathcal{F}A^{\bullet}[1]
        \end{tikzcd}
    \end{equation}
    By 5 Lemma, $\varphi$ is an isomorphism and hence $\cone(\mathcal{F}f)$ is acyclic.
    Again by Proposition \ref{Proposition 6.1}, we get $\mathcal{F}f$ is quasi-isomorphism.
\end{proof}
\begin{lemma}
    $R\mathcal{F}$ is well defined.
    \label{Lemma 6.14}
\end{lemma}
Assume that $\mathscr{A}$ has enough injective objects.
Take $\mathcal{K}'$ to be the homotopy category of bound-below complexes of injective objects.
To define derived functor of total hom functor,
it remains to show that $\mathcal{K}'$ satisfies $\hom^{\bullet}(A^{\bullet}, \cdot)$-condition for all $A^{\bullet}$.
Here we only prove for the acyclic condition.
\begin{proposition}
    For all complexes $A^{\bullet},B^{\bullet}\in \mathcal{C}(\mathscr{A})$,
    we have that $H^{n}(\hom^{\bullet}(A^{\bullet}, B^{\bullet})) = \hom_{\mathcal{K}(\mathscr{A})}(A^{\bullet}, B^{\bullet}[n])$. 
    \label{Proposition 6.3}
\end{proposition}
\begin{proof}
    For chain map $f: A^{\bullet} \rightarrow B^{\bullet}$ in $\ker (\partial_{\hom^{\bullet}}^{n})$,
    we have that $-(-1)^{n}f_{i} \circ \partial_{A^{\bullet}}^{i - 1} + \partial_{B^{\bullet}}^{i + n - 1} \circ f_{i - 1} = 0$ so that $f_{i} \circ \partial_{A^{\bullet}}^{i - 1} = \partial_{B^{\bullet}[n]}^{i - 1} \circ f_{i - 1}$.
    Hence each $f$ defines a chain map $\widetilde{f}: A^{\bullet} \rightarrow B^{\bullet}[n]$.
    \par
    For $[f] = [g]$ in $H^{n}(\hom^{\bullet}(A^{\bullet}, B^{\bullet}))$,
    we get that $f - g = \partial h$ for some $h\in \hom^{n - 1}(A^{\bullet}, B^{\bullet})$.
    While 
    \begin{equation}
        \begin{split}
            \partial h & = (-(-1)^{n - 1}h_{i} \circ \partial_{A^{\bullet}}^{i - 1} + \partial_{B^{\bullet}}^{i + n - 1} \circ g_{i - 1}) \\
            & = ((-1)^{n}(g_{i} \circ \partial_{A^{\bullet}}^{i - 1} + \partial_{B^{\bullet}}^{i + n - 1} \circ g_{i - 1}))
        \end{split}
    \end{equation}
    so that $\widetilde{f}$ is homotopic to $\widetilde{g}$.
    Thus we get a well defined map from $H^{n}(\hom^{\bullet}(A^{\bullet}, B^{\bullet}))$ to $\hom_{\mathcal{K}(\mathscr{A})}(A^{\bullet}, B^{\bullet}[n])$.
    Clearly, the map is bijective.
\end{proof}
Recall that in homological algebra, we say a complex $A^{\bullet}$ is contractible if $\identity_{A^{\bullet}}$ is null-homotopic and all acyclic bounded-below complex of injective objects is contractible.
We have the following lemma.
\begin{lemma}
    Let $I^{\bullet}\in \mathcal{K}'$ be acyclic bounded-below complex of injective objects.
    Then for all complex $A^{\bullet}$, $\hom^{\bullet}(A^{\bullet}, I^{\bullet})$ is still acyclic.
    \label{Lemma 6.15}
\end{lemma}
\begin{proof}
    By Proposition \ref{Proposition 6.3}, this is equivalent to $\hom_{\mathcal{K}(\mathscr{A})}(A^{\bullet}, I^{\bullet}[n]) = \{0\}$ for all $n$.
    Hence, we suffice to show that any chain map $f: A^{\bullet} \rightarrow I^{\bullet}$ is null-homotopic, 
    which immediately comes from the fact that $I^{\bullet}$ is contractible.
\end{proof}
Hence $\mathcal{K}'$ satisfies $\hom^{\bullet}(A^{\bullet}, \cdot)$-condition for all $A^{\bullet}$.
And we can define derived functor, denoted by $R\hom(A^{\bullet}, \cdot)$.
\begin{lemma}
    Let $I^{\bullet}\in \mathcal{K}'$ be bounded-below complex of injective objects.
    Then for all complex $A^{\bullet}$, we have that $\hom_{\mathcal{D}(\mathscr{A})}(A^{\bullet}, I^{\bullet}) = \hom_{\mathcal{K}(\mathscr{A})}(A^{\bullet}, I^{\bullet})$.
    \label{Lemma 6.16}
\end{lemma}
\begin{proof}
    Only need to show that for any morphism $A^{\bullet} \rightarrow B^{\bullet} \overset{qis}{\leftarrow} I^{\bullet}$,
    there exists chain map $A^{\bullet} \rightarrow I^{\bullet}$ making the diagram commutative up to homotopy
    \begin{equation}
        \begin{tikzcd}
            & B^{\bullet} & \\
            A^{\bullet} \arrow[ur] \arrow[rr] &&
            I^{\bullet} \arrow[ul, "qis" swap]
        \end{tikzcd}
    \end{equation}
    While here we prove a stronger fact that there exists chain map $B^{\bullet} \rightarrow I^{\bullet}$,
    whose composition with the quasi-isomorphism is homotopy to $\identity_{I^{\bullet}}$.
    \par
    In fact, since $q: I^{\bullet} \rightarrow B^{\bullet} \rightarrow B'^{\bullet}$ is quasi-isomorphism.
    By Proposition \ref{Proposition 6.1}, $\cone(q)\in \mathcal{K}'$ is acyclic and hence contractible.
    Hence just as proof of Lemma \ref{Lemma 6.15}, the natural map $\cone(q) \rightarrow I^{\bullet}[1]$ is null-homotopic.
    Consider the chain homotopy $D_{n} = (\alpha_{n}, \beta_{n}): \cone(q)^{n} \rightarrow I^{n}$.
    Then 
    \begin{equation}
        \left\{
            \begin{aligned}
                & \partial_{I^{\bullet}} \circ \alpha_{n} = \alpha_{n + 1} \circ \partial_{\cone(q)} \\
                & \identity_{I^{\bullet}} = \alpha_{n + 1} \circ q - \partial_{I^{\bullet}} \circ \beta_{n} - \beta_{n + 1} \circ \partial_{I^{\bullet}}
            \end{aligned}
        \right.
    \end{equation}
    Hence $\alpha$ is a chain map and $\alpha \circ q$ is homotopy to $\identity_{I^{\bullet}}$.
\end{proof}
\begin{lemma}
    For all quasi-isomorphism $E \rightarrow E'$ in $\mathcal{C}(\mathscr{A})$,
    the induced map $\hom^{\bullet}(E', F) \rightarrow \hom^{\bullet}(E, F)$ is also a quasi-isomorphism for all $F\in \mathcal{K}'$.
    \label{Lemma 6.17}
\end{lemma}
\begin{proof}
    By Lemma \ref{Lemma 6.16}, we immediately get $H^{n}(\hom^{\bullet}(E', F)) = \hom_{\mathcal{D}(\mathscr{A})}(E', F[n])$ and $H^{n}(\hom^{\bullet}(E, F)) = \hom_{\mathcal{D}(\mathscr{A})}(E, F[n])$.
    Hence 
    \begin{equation}
        \hom_{\mathcal{D}(\mathscr{A})}(E', F[n]) \overset{\circ (E' \rightarrow E)}{\longrightarrow} \hom_{\mathcal{D}(\mathscr{A})}(E, F[n])
    \end{equation}
    is an isomorphism.
    Conclude that $\hom^{\bullet}(E', F) \rightarrow \hom^{\bullet}(E, F)$ is a quasi-isomorphism.
\end{proof}
\begin{remark}
    With the previous two lemmas, when consider a total hom functor, we can directly take $A^{\bullet}\in \mathcal{D}(\mathscr{A})$.
    In addition, if we denote $\ext^{n}(A^{\bullet}, B^{\bullet}) = H^{n}(R\hom(A^{\bullet}, B^{\bullet}))$, then
    \begin{equation}
        \begin{split}
            \ext^{n}(A^{\bullet}, B^{\bullet}) & = H^{n}(R\hom(A^{\bullet}, B^{\bullet})) \\
            & = H^{n}(\hom^{\bullet}(A^{\bullet}, B'^{\bullet})) \\
            & = \hom_{\mathcal{K}(\mathscr{A})}(A^{\bullet}, B'^{\bullet}[n]) \\
            & = \hom_{\mathcal{D}(\mathscr{A})}(A^{\bullet}, B'^{\bullet}[n]) \\
            & = \hom_{\mathcal{D}(\mathscr{A})}(A^{\bullet}, B^{\bullet}[n])
        \end{split}
    \end{equation}
    so that we can directly write out extension classes without annoying $A'^{\bullet}$ or $B'^{\bullet}$.
\end{remark}
Let $\mathscr{A} = \module(\mathcal{O}_{X})$.
For short, we denote homotopy category and derived category by $\mathcal{K}(X)$ and $\mathcal{D}(X)$ respectively.
Consider total tensor product $\mathcal{F}^{\bullet} \otimes_{\mathcal{O}_{X}}^{n} \mathcal{G}^{\bullet} = \oplus_{p + q = n} \mathcal{F}^{p} \otimes_{\mathcal{O}_{X}} \mathcal{G}^{q}$.
Replacing injective objects by flat $\mathcal{O}_{X}$-modules, 
we can also define derived functor of total tensor functor.
In particular, we have derived pull back $Lf^{\ast}$.

\section{Cotangent Complex}
\label{section:Cotangent_Complex}

\subsection{Some properties of cotangent complex}
\label{subsection:Cotangent_Complex_Some_properties_of_cotangent_complex}

Let $f: X \rightarrow Y$ be a morphism of scheme.
Denote the cotangent complex of $f$ by $\mathbb{L}_{f}$, or when there is no confusion by $\mathbb{L}_{X/Y}$.
At most of time, we would view it as an element in $\mathcal{D}^{\le 0}(X)$.
Firstly, let's state some properties about cotangent complex.
\begin{proposition}
    Given commutative diagram
    \begin{equation}
        \begin{tikzcd}[sep = tiny]
            X \arrow[rr, "f"] \arrow[dr] &&
            Y \arrow[dl] \\
            & S &
        \end{tikzcd}
    \end{equation}
    there is an exact triangle in $\mathcal{D}(X)$
    \begin{equation}
        Lf^{\ast}\mathbb{L}_{Y/S} \longrightarrow \mathbb{L}_{X/S} \longrightarrow \mathbb{L}_{f} \longrightarrow Lf^{\ast}\mathbb{L}_{Y/S}[1]
    \end{equation}
    \vspace{-1.5\baselineskip} % 减少一行间距
    \label{Proposition 7.1}
\end{proposition}
\begin{remark}
    In fact, $H^{0}(\mathbb{L}_{X/Y}) = \Omega_{X/Y}^{1}$ and when taking sheaf cohomology, we would get a long exact sequence
    \begin{equation}
        \cdots \longrightarrow H^{-1}(\mathbb{L}_{f}) \longrightarrow f^{\ast}\Omega_{Y/S}^{1} \longrightarrow \Omega_{X/S}^{1} \longrightarrow \Omega_{X/Y}^{1} \longrightarrow 0
    \end{equation}
    which is a generalization of classic short exact sequence of Kahler differentials.
    Hence to some extent, $H^{-1}(\mathbb{L}_{f})$ measures the singularities of $f$.
\end{remark}
\begin{lemma}
    Let $X \rightarrow Y$ be an etale morphism.
    Then $\mathbb{L}_{X/Y} = 0$.
    \label{Lemma 7.1}
\end{lemma}
\begin{lemma}
    Let $X \rightarrow Y$ be a smooth morphism.
    Then $\mathbb{L}_{X/Y} = \Omega_{X/Y}^{1}[0]$.
    \label{Lemma 7.2}
\end{lemma}
\begin{lemma}
    Let $X \rightarrow Y$ be a morphism of schemes.
    Assume $X \rightarrow Y$ factors $W$ such that $X \hookrightarrow W$ is a regular embedding with ideal sheaf $\mathcal{I}$ and $W \rightarrow Y$ is smooth.
    Then $\mathbb{L}_{X/Y} = [\mathcal{I}/\mathcal{I}^2 \rightarrow \Omega_{W/Y}^{1}\big{|}_{X}]$.
    \label{Lemma 7.3}
\end{lemma}
\begin{definition}
    Let $X \rightarrow Y$ be a morphism of schemes.
    Define naive cotangent complex of $f$ to be $\mathbb{L}_{X/Y}^{\ge -1} = \tau^{\ge -1}\mathbb{L}_{X/Y}$.
    In some references, this would also be denoted as $N\mathbb{L}_{X/Y}$.
\end{definition}
\begin{lemma}
    Let $X \rightarrow Y$ be a morphism of schemes.
    Assume $X \rightarrow Y$ factors $W$ such that $X \hookrightarrow W$ is a closed embedding with ideal sheaf $\mathcal{I}$ and $W \rightarrow Y$ is smooth.
    Then $\mathbb{L}_{X/Y}^{\ge -1} = [\mathcal{I}/\mathcal{I}^2 \rightarrow \Omega_{W/Y}^{1}\big{|}_{X}]$.
    \label{Lemma 7.4}
\end{lemma}
\begin{remark}
    In affine case, we can always find such a $W$.
    Let $f: A \rightarrow B$ be a ring homomorphism.
    We can define naive cotangent complex of $f$ to be $N\mathbb{L}_{B/A} = [I/I^{2} \rightarrow \Omega_{P/A} \otimes_{P} B]$,
    where $P = A[B]$ and $I$ is kernel of $P \rightarrow B$.
\end{remark}
\begin{proposition}
    Let $X \hookrightarrow Y$ be a closed immersion with ideal sheaf $\mathcal{I}$.
    Then $\mathbb{L}_{X/Y}^{\ge -1} = \mathcal{I}/\mathcal{I}^{2}[1]$.
    \label{Proposition 7.2}
\end{proposition}
\begin{reason}
    Consider $X \hookrightarrow Y \overset{\identity_{Y}}{\rightarrow} Y$.
    By Lemma \ref{Lemma 7.4}, we get $\mathbb{L}_{X/Y}^{\ge -1} = [\mathcal{I}/\mathcal{I}^2 \rightarrow \Omega_{Y/Y}^{1}\big{|}_{X}] = \mathcal{I}/\mathcal{I}^{2}[1]$.
\end{reason}

\subsection{Deformation of map}
\label{subsection:Cotangent_Complex_Deformation_of_map}

Consider deformation problem of map over $K$-scheme.
Here $K$ is a ring, not necessarily a field.
Given closed immersion $T \hookrightarrow T'$ with ideal sheaf $\mathcal{J}$ of square $0$.
Assume there is a morphism $f: T \rightarrow X$.
Question if there exists extension $T' \rightarrow X$ making the diagram commutative
\begin{equation}
    \begin{tikzcd}[sep = tiny]
        T' \arrow[rr, dashrightarrow] &&
        X \\
        & T \arrow[ul] \arrow[ur, "f" swap] &
    \end{tikzcd}
\end{equation}
In fact, we have that
\begin{itemize}
    \item no automorphism 
    \item extensions are pseudo torsor under $\ext_{T}^{0}(Lf^{\ast}\mathbb{L}_{X}, \mathcal{J})$ 
    \item obstruction lies in $\ext_{T}^{1}(Lf^{\ast}\mathbb{L}_{X}, \mathcal{J})$.
\end{itemize}
For extensions of map, assume that there are two extensions
\begin{equation}
    \begin{tikzcd}
        T' \arrow[rr, "g", shift left] \arrow[rr, "h" swap, shift right] &&
        X \\
        & T \arrow[ul] \arrow[ur, "f" swap] &
    \end{tikzcd}
\end{equation}
then locally we have that
\begin{equation}
    \begin{tikzcd}
        A' \arrow[dr, twoheadrightarrow] &&
        B \arrow[dl, "f^{\sharp}"] \arrow[ll, "g^{\sharp}" swap, shift right] \arrow[ll, "h^{\sharp}", shift left] \\
        & A &
    \end{tikzcd}
\end{equation}
By commutativity, $g^{\sharp} - h^{\sharp}$ defines a $K$-module homomorphism $B \rightarrow J$.
Thus
\begin{equation}
    \begin{split}
        \{\text{extensions}\} & \leftrightsquigarrow \derivation_{K}(B, J) \\
        & \leftrightsquigarrow \hom_{B}(\Omega_{B}, J)
    \end{split}
\end{equation}
where $\hom_{B}(\Omega_{B}, J) = \hom_{B}(\Omega_{B}, \hom_{A}(A, J)) = \hom_{A}(\Omega_{B} \otimes_{B} A, J)$.
Hence globally, extensions are classified by $\hom_{T}(f^{\ast}\Omega_{X}^{1}, \mathcal{J})$.
\begin{lemma}
    In fact, $\hom_{T}(f^{\ast}\Omega_{X}^{1}, \mathcal{J}) = \ext_{T}^{0}(Lf^{\ast}\mathbb{L}_{X}^{\ge -1}, \mathcal{J}) = \ext_{T}^{0}(Lf^{\ast}\mathbb{L}_{X}, \mathcal{J})$.
    \label{Lemma 7.5}
\end{lemma}
\begin{proof}
    We firstly show the second equality.
    Consider flat resolution of cotangent complex $\mathbb{L}_{X}$
    \begin{equation}
        \begin{tikzcd}
            & \cdots \arrow[d] &
            \cdots \arrow[d] & \\
            \cdots \arrow[r] &
            \mathcal{F}^{-1, -2} \arrow[r] \arrow[d] &
            \mathcal{F}^{0, -2} \arrow[r] \arrow[d] &
            0 \\
            \cdots \arrow[r] &
            \mathcal{F}^{-1, -1} \arrow[r] \arrow[d] &
            \mathcal{F}^{0, -1} \arrow[r] \arrow[d] &
            0 \\
            \cdots \arrow[r] &
            \mathcal{\mathcal{F}}^{-1, 0} \arrow[r] \arrow[d] &
            \mathcal{\mathcal{F}}^{0, 0} \arrow[r] \arrow[d] &
            0 \\
            \cdots \arrow[r] &
            \mathbb{L}_{X}^{-1} \arrow[r] &
            \mathbb{L}_{X}^{0} \arrow[r] &
            0 \\
        \end{tikzcd}
    \end{equation}
    Take total complex, we get quasi-isomorphism $\totalcomplex (\mathcal{F}^{\bullet, \ast}) \rightarrow \mathbb{L}_{X}$.
    Note that by truncation $\tau^{\ge -1}$, $\totalcomplex (\mathcal{F}^{\ge -1, \ast}) \rightarrow \mathbb{L}_{X}^{\ge -1}$ also a quasi-isomorphism,
    we have that $\partial_{Lf^{\ast}\mathbb{L}_{X}}^{-1} = \partial_{Lf^{\ast}\mathbb{L}_{X}^{\ge -1}}^{-1}$.
    While by Lemma \ref{Lemma 6.7}, 
    \begin{equation}
        \hom_{\mathcal{D}(T)}(Lf^{\ast}\mathbb{L}_{X}, \mathcal{J}[0]) = \hom_{\mathcal{C}(T)}(Lf^{\ast}\mathbb{L}_{X}, \mathcal{J}[0]) = \hom_{T}(\coker \partial_{Lf^{\ast}\mathbb{L}_{X}}^{-1}, \mathcal{J})
    \end{equation}
    and
    \begin{equation}
        \hom_{\mathcal{D}(T)}(Lf^{\ast}\mathbb{L}_{X}^{\ge -1}, \mathcal{J}[0]) = \hom_{\mathcal{C}(T)}(Lf^{\ast}\mathbb{L}_{X}^{\ge -1}, \mathcal{J}[0]) = \hom_{T}(\coker \partial_{Lf^{\ast}\mathbb{L}_{X}^{\ge -1}}^{-1}, \mathcal{J})
    \end{equation}
    Hence the second equality holds.
    For the first equality, as $f^{\ast}$ is right exact functor,
    we get that $\coker \partial_{Lf^{\ast}\mathbb{L}_{X}^{\ge -1}}^{-1} = H^{0}(\totalcomplex (\mathcal{F}^{\ge -1, \ast})) = H^{0}(\mathbb{L}_{X}^{\ge -1}) = \Omega_{X}$, done!
\end{proof}
\begin{remark}
    Similarly, we also have $\ext_{T}^{1}(Lf^{\ast}\mathbb{L}_{X}^{\ge -1}, \mathcal{J}) = \ext_{T}^{1}(Lf^{\ast}\mathbb{L}_{X}, \mathcal{J})$.
\end{remark}
For obstruction, by Proposition \ref{Proposition 7.1}, we have a diagram in $\mathcal{D}(T)$
\begin{equation}
    \begin{tikzcd}
        Lf^{\ast}\mathbb{L}_{X} \arrow[r] &
        \mathbb{L}_{T} \arrow[r] \arrow[d, equal] &
        \mathbb{L}_{f} \arrow[r] &
        Lf^{\ast}\mathbb{L}_{X}[1] \\
        Li^{\ast}\mathbb{L}_{T'} \arrow[r] &
        \mathbb{L}_{T} \arrow[r] &
        \mathbb{L}_{T/T'} \arrow[r] &
        Li^{\ast}\mathbb{L}_{T'}[1]
    \end{tikzcd}
\end{equation}
Hence we get a morphism $\varphi: Lf^{\ast}\mathbb{L}_{X} \rightarrow \mathbb{L}_{T/T'}^{\ge -1}$.
Note that by Proposition \ref{Proposition 7.2}, $\mathbb{L}_{T/T'}^{\ge -1} = \mathcal{J}[1]$, 
the morphism $\varphi$ is an element in $\hom_{\mathcal{D}(T)}(Lf^{\ast}\mathbb{L}_{X}, \mathcal{J}[1]) = \ext_{T}^{1}(Lf^{\ast}\mathbb{L}_{X}, J)$.
In fact $\varphi$ is just our desired obstruction.
\par
More explicitly, assume that $X$ is a closed subscheme of some $W$ smooth over $\spec K$ and there exists morphism $g: T' \rightarrow W$ making the following diagram commutative
\begin{equation}
    \begin{tikzcd}
        T \arrow[r, "f"] \arrow[d, hookrightarrow, "i"] &
        X \arrow[d, hookrightarrow, "j"] \\
        T' \arrow[r, "g"] &
        W
    \end{tikzcd}
\end{equation}
Denote the ideal sheaf of $j$ by $\mathcal{I}$, locally we have a commutative diagram
\begin{equation}
    \begin{tikzcd}
        A/J &
        B/I \arrow[l] \\
        A \arrow[u, twoheadrightarrow] &
        B \arrow[l] \arrow[u, twoheadrightarrow]
    \end{tikzcd}
\end{equation}
Hence there is natural map $I/I^{2} \rightarrow J/J^{2} = J$.
And $B \rightarrow A$ factors through $B/I$ if and only if $I/I^{2} \rightarrow J$ is a zero map.
Globally we get $g^{\ast}: \mathcal{I}/\mathcal{I}^{2}\big{|}_{T} \rightarrow \mathcal{J}$.
Suffices to show that $g^{\ast}$ can be modified to zero map if and only if $\varphi = 0$.
Note that there is an exact sequence
\begin{equation}
    \mathcal{I}/\mathcal{I}^{2}\big{|}_{T} \longrightarrow \Omega_{W}^{1}\big{|}_{T} \longrightarrow \Omega_{X}^{1}\big{|}_{T} \longrightarrow 0
\end{equation}
By Lemma \ref{Lemma 7.4}, $\mathbb{L}_{X}^{\ge -1} = [\mathcal{I}/\mathcal{I}^{2} \rightarrow \Omega_{W}^{1}\big{|}_{X}]$.
And $g^{\ast}$ gives a morphism in $\hom_{\mathcal{K}(T)}(f^{\ast}\mathbb{L}_{X}^{\ge -1}, \mathcal{J}[1])$
\begin{equation}
    \begin{tikzcd}
        {[\mathcal{I}/\mathcal{I}^{2}}\big{|}_{T} \arrow[r] \arrow[d, "g^{\ast}"] &
        {\Omega_{W}^{1}\big{|}_{T}]} \arrow[dl, dashrightarrow, "h"] \\
        {[\mathcal{J}} \arrow[r] &
        {0]}
    \end{tikzcd}
\end{equation}
Hence $g^{\ast}$ can be modified to zero map if and only if $g^{\ast} = 0$ in $\hom_{\mathcal{K}(T)}(f^{\ast}\mathbb{L}_{X}^{\ge -1}, \mathcal{J}[1])$.
By Lemma \ref{Lemma 6.6}, this is also equivalent to that $g^{\ast} = 0$ in $\hom_{\mathcal{D}(T)}(f^{\ast}\mathbb{L}_{X}^{\ge -1}, \mathcal{J}[1])$.
\begin{lemma}
    In the setting above, $\ext_{T}^{1}(Lf^{\ast}\mathbb{L}_{X}^{\ge -1}, \mathcal{J}) = \hom_{\mathcal{D}(T)}(f^{\ast}\mathbb{L}_{X}^{\ge -1}, \mathcal{J}[1])$.
    \label{Lemma 7.6}
\end{lemma}
\begin{proof}
    Take flat resolution $\mathcal{F}^{\bullet}$ of $\mathcal{I}/\mathcal{I}^{2}$.
    As $\Omega_{W}^{1}\big{|}_{X}$ is locally free, we get a quasi-isomorphism with the first row a complex of flat $\mathcal{O}_{X}$-module sheaves
    \begin{equation}
        \begin{tikzcd}
            \mathcal{F}^{-1} \arrow[r] &
            \mathcal{F}^{0} \arrow[r] \arrow[d, twoheadrightarrow] &
            \Omega_{W}^{1}\big{|}_{X} \arrow[r] \arrow[d, equal] &
            0 \\
            0 \arrow[r] &
            \mathcal{I}/\mathcal{I}^{2} \arrow[r] &
            \Omega_{W}^{1}\big{|}_{X} \arrow[r] &
            0
        \end{tikzcd}
        \label{equation:flat_resolution_of_naive_cotangent_complex}
    \end{equation}
    Hence $Lf^{\ast}\mathbb{L}_{X}^{\ge -1} = [\cdots \rightarrow f^{\ast}\mathcal{F}^{-1} \rightarrow f^{\ast}\mathcal{F}^{0} \rightarrow f^{\ast}\Omega_{W}^{1}\big{|}_{X} \rightarrow 0]$.
    Applying pull back $f^{\ast}$ to chain map \ref{equation:flat_resolution_of_naive_cotangent_complex} , we get
    \begin{equation}
        \begin{tikzcd}
            f^{\ast}\mathcal{F}^{-1} \arrow[r] &
            f^{\ast}\mathcal{F}^{0} \arrow[r] \arrow[d] &
            f^{\ast}\Omega_{W}^{1}\big{|}_{X} \arrow[r] \arrow[d, equal] &
            0 \\
            0 \arrow[r] &
            f^{\ast}\mathcal{I}/\mathcal{I}^{2} \arrow[r] &
            f^{\ast}\Omega_{W}^{1}\big{|}_{X} \arrow[r] &
            0
        \end{tikzcd}
    \end{equation}
    As pull back is right exact, the cohomology group homomorphisms at degree $-1$ and $0$ are still isomorphic.
    Hence $\tau^{\ge -1}Lf^{\ast}\mathbb{L}_{X}^{\ge -1} \rightarrow f^{\ast}\mathbb{L}_{X}^{\ge -1}$ is a quasi-isomorphism.
    Thus by Proposition \ref{Proposition 6.2}, 
    \begin{equation}
        \begin{split}
            \ext_{T}^{1}(Lf^{\ast}\mathbb{L}_{X}^{\ge -1}, \mathcal{J}) & = \hom_{\mathcal{D}(T)}(Lf^{\ast}\mathbb{L}_{X}^{\ge -1}, \mathcal{J}[1]) \\
            & = \hom_{\mathcal{D}(T)}(\tau^{\ge -1}Lf^{\ast}\mathbb{L}_{X}^{\ge -1}, \mathcal{J}[1]) \\
            & = \hom_{\mathcal{D}(T)}(f^{\ast}\mathbb{L}_{X}^{\ge -1}, \mathcal{J}[1])
        \end{split}
    \end{equation}
    done!
\end{proof}
\begin{remark}
    However, we do not always have such a setting.
    Firstly, in general, for $T$ not affine, there is not always a morphism $T \rightarrow W$.
    In addition, unlike that in the case when $X = \spec A$ is affine, we can take $W = \spec K[A]$,
    for general scheme $X$, we cannot find such a scheme $W$.
    \par
    To deal with general $X$, we need to consider it as a ringed space.
    Define $\widetilde{X}$ with same underlying topological space as $X$ and structure sheaf $\pi^{-1}\mathcal{O}_{\spec K}[\mathcal{O}_{X}]$ defined as sheafification of the following presheaf of sets
    \begin{equation}
        U \longmapsto (\pi^{-1}\mathcal{O}_{\spec K}(U))[\mathcal{O}_{X}(U)]
    \end{equation}
\end{remark}

\subsection{Local to global spectral sequence}
\label{subsection:Cotangent_Complex_Local_to_global_spectral_sequence}

Let $X$ be a scheme.
Consider total hom functor on the category of $\mathcal{O}_{X}$-module sheaves, 
denoted by $\sheafhom^{\bullet}(\mathcal{F}^{\bullet}, \mathcal{G}^{\bullet})$.
Then we get derived total sheaf hom $R\sheafhom$.
As $\Gamma(X, \cdot) \circ \sheafhom = \hom$, it is natural to question if $R\Gamma(X, \cdot) \circ R\sheafhom = R\hom$.
In fact, this comes from the following lemma.
\begin{lemma}
    Let $X$ be a scheme, $\mathcal{F}$ sheaf of flat module.
    Then for all injective $\mathcal{I}\in \module(\mathcal{O}_{X})$, $\sheafhom_{\mathcal{O}_{X}}(\mathcal{F}, \mathcal{I})$ is still injective.
    \label{Lemma 7.7}
\end{lemma}
\begin{proof}
    Want to show for all injection $\mathcal{A} \hookrightarrow \mathcal{B}$,
    the induced map $\hom_{\mathcal{O}_{X}}(\mathcal{B}, \sheafhom_{\mathcal{O}_{X}}(\mathcal{F}, \mathcal{I})) \rightarrow \hom_{\mathcal{O}_{X}}(\mathcal{A}, \sheafhom_{\mathcal{O}_{X}}(\mathcal{F}, \mathcal{I}))$ is surjective.
    In fact, as sheaf hom and tensor product are adjoint pair, we get commutative diagram
    \begin{equation}
        \begin{tikzcd}
            \hom_{\mathcal{O}_{X}}(\mathcal{B}, \sheafhom_{\mathcal{O}_{X}}(\mathcal{F}, \mathcal{I})) \arrow[r] \arrow[d, equal] &
            \hom_{\mathcal{O}_{X}}(\mathcal{A}, \sheafhom_{\mathcal{O}_{X}}(\mathcal{F}, \mathcal{I})) \arrow[d, equal] \\
            \hom_{\mathcal{O}_{X}}(\mathcal{B} \otimes_{\mathcal{O}_{X}} \mathcal{F}, \mathcal{I}) \arrow[r] &
            \hom_{\mathcal{O}_{X}}(\mathcal{A} \otimes_{\mathcal{O}_{X}} \mathcal{F}, \mathcal{I})
        \end{tikzcd}
    \end{equation}
    Since $\mathcal{F}$ is flat and $\mathcal{I}$ is injective, 
    $\mathcal{A} \otimes_{\mathcal{O}_{X}} \mathcal{F} \rightarrow \mathcal{B} \otimes_{\mathcal{O}_{X}} \mathcal{F}$ is still injective so that the induced map is surjective.
\end{proof}
Recall the following theorem by Grothendieck in homological algebra.
\begin{theorem}[\textbf{\emph{Grothendieck}}]
    Let $\mathscr{A} \overset{G}{\rightarrow} \mathscr{B} \overset{F}{\rightarrow} \mathscr{C}$ be covariant additive functors,
    where $\mathscr{A}, \mathscr{B}$ and $\mathscr{C}$ are abelian categories with enough injective objects.
    Assume that $F$ is left exact and for $I$ injective, $GI$ is $F$-acyclic.
    Then for all object $A\in \mathscr{A}$, there is a first quadrant spectral sequence with
    \begin{equation}
        E_{2}^{p, q} = (R^{p}F)(R^{q}G)A \underset{p}{\Rightarrow} R^{p + q}(FG)A
    \end{equation}
    \vspace{-1.5\baselineskip} % 减少一行间距
    \label{Theorem 7.1}
\end{theorem}
Here we take $\mathscr{A} = \mathscr{B} = \module (\mathcal{O}_{X})$ and $\mathscr{C} = \abeliangroups$ with $F = \Gamma(X, \cdot)$ and $G = \sheafhom_{\mathcal{O}_{X}}(\mathcal{F}, \cdot)$, where $F$ is a flat module sheaves.
Then Lemma \ref{Lemma 7.7} tells us that conditions of Grothendieck Theorem hold. 
Hence we get $H^{p}(X, \sheafext_{\mathcal{O}_{X}}^{q}(\mathcal{F}, \mathcal{G})) \underset{p}{\Rightarrow} \ext_{\mathcal{O}_{X}}^{p + q}(\mathcal{F}, \mathcal{G})$.
And by diagram chasing (cf. Weibel, exercise 5.1.3), we have an exact sequence of low degree terms
\begin{equation}
    \begin{split}
            0 & \rightarrow H^{1}(X, \sheafext_{\mathcal{O}_{X}}^{0}(\mathcal{F}, \mathcal{G})) \rightarrow \ext_{\mathcal{O}_{X}}^{1}(\mathcal{F}, \mathcal{G}) \rightarrow H^{0}(X, \sheafext_{\mathcal{O}_{X}}^{1}(\mathcal{F}, \mathcal{G})) \\
            & \rightarrow H^{2}(X, \sheafext_{\mathcal{O}_{X}}^{0}(\mathcal{F}, \mathcal{G})) \rightarrow \ext_{\mathcal{O}_{X}}^{2}(\mathcal{F}, \mathcal{G}) 
    \end{split}
\end{equation}
Take injective resolution $\mathcal{I}^{\bullet}$ of $\mathcal{G}$.
Similar to idea of exact sequence of low degree items, for all $\mathcal{F}^{\bullet}\in \mathcal{C}^{\le 0}(X)$ complex of flat module sheaves,
consider the following commutative diagram with all rows and columns complexes
\begin{equation}
    \begin{tikzcd}[column sep = small]
        & \cdots &
        \cdots &
        \cdots & \\
        0 \arrow[r] &
        \hom_{\mathcal{O}_{X}}(\mathcal{F}^{-2}, \mathcal{I}^{0}) \arrow[u] \arrow[r] &
        \hom_{\mathcal{O}_{X}}(\mathcal{F}^{-2}, \mathcal{I}^{1}) \arrow[u] \arrow[r] &
        \hom_{\mathcal{O}_{X}}(\mathcal{F}^{-2}, \mathcal{I}^{2}) \arrow[u] \arrow[r] &
        \cdots \\
        0 \arrow[r] &
        \hom_{\mathcal{O}_{X}}(\mathcal{F}^{-1}, \mathcal{I}^{0}) \arrow[u] \arrow[r] &
        \hom_{\mathcal{O}_{X}}(\mathcal{F}^{-1}, \mathcal{I}^{1}) \arrow[u] \arrow[r] &
        \hom_{\mathcal{O}_{X}}(\mathcal{F}^{-1}, \mathcal{I}^{2}) \arrow[u] \arrow[r] &
        \cdots \\
        0 \arrow[r] &
        \hom_{\mathcal{O}_{X}}(\mathcal{F}^{0}, \mathcal{I}^{0}) \arrow[u] \arrow[r] &
        \hom_{\mathcal{O}_{X}}(\mathcal{F}^{0}, \mathcal{I}^{1}) \arrow[u] \arrow[r] &
        \hom_{\mathcal{O}_{X}}(\mathcal{F}^{0}, \mathcal{I}^{2}) \arrow[u] \arrow[r] &
        \cdots \\
        & 0 \arrow[u] &
        0 \arrow[u] &
        0 \arrow[u] &
    \end{tikzcd}
\end{equation}
we can also get an exact sequence of low degree items
\begin{equation}
    \begin{split}
            0 & \rightarrow H^{1}(X, \sheafext_{X}^{0}(\mathcal{F}^{\bullet}, \mathcal{G})) \rightarrow \ext_{X}^{1}(\mathcal{F}^{\bullet}, \mathcal{G}) \rightarrow H^{0}(X, \sheafext_{X}^{1}(\mathcal{F}^{\bullet}, \mathcal{G})) \\
            & \rightarrow H^{2}(X, \sheafext_{X}^{0}(\mathcal{F}^{\bullet}, \mathcal{G})) \rightarrow \ext_{X}^{2}(\mathcal{F}^{\bullet}, \mathcal{G}) 
    \end{split}
    \label{equation:exact_sequence_of_low_degree_items_for_derived_ext}
\end{equation}
Note that for all $\mathcal{A}^{\bullet}\in \mathcal{C}^{\le 0}(X)$, we can take such an $\mathcal{F}^{\bullet}$ with quasi-isomorphism $\mathcal{F}^{\bullet} \rightarrow A^{\bullet}$,
we can replace $\mathcal{F}^{\bullet}$ by $\mathcal{A}^{\bullet}$ in exact sequence \ref{equation:exact_sequence_of_low_degree_items_for_derived_ext}.
In particular, if we take $A^{\bullet} = Lf^{\ast}\mathbb{L}_{X}$, then we would get
\begin{equation}
    \begin{split}
            0 & \rightarrow H^{1}(X, \sheafext_{X}^{0}(Lf^{\ast}\mathbb{L}_{X}, \mathcal{G})) \rightarrow \ext_{X}^{1}(Lf^{\ast}\mathbb{L}_{X}, \mathcal{G}) \rightarrow H^{0}(X, \sheafext_{X}^{1}(Lf^{\ast}\mathbb{L}_{X}, \mathcal{G})) \\
            & \rightarrow H^{2}(X, \sheafext_{X}^{0}(Lf^{\ast}\mathbb{L}_{X}, \mathcal{G})) \rightarrow \ext_{X}^{2}(Lf^{\ast}\mathbb{L}_{X}, \mathcal{G}) 
    \end{split}
\end{equation}
Note that $\ext_{X}^{1}(Lf^{\ast}\mathbb{L}_{X}, \mathcal{G})$ is just where the obstruction lies, 
the exact sequence tells us that intuitively obstruction consists of two parts.
One part is obstruction of locally extension lying in $\ker (H^{0}(X, \sheafext_{X}^{1}(Lf^{\ast}\mathbb{L}_{X}, \mathcal{G})) \rightarrow H^{2}(X, \sheafext_{X}^{0}(Lf^{\ast}\mathbb{L}_{X}, \mathcal{G})))$.
When local obstruction vanishes, local extension always exists.
And the obstruction to gluing up such local extensions lies in $H^{1}(X, \sheafext_{X}^{0}(Lf^{\ast}\mathbb{L}_{X}, \mathcal{G}))$.

\subsection{Deformation of scheme}
\label{subsection:Cotangent_Complex_Deformation_of_scheme}

Given closed immersion $S \hookrightarrow S'$ with ideal sheaf $\mathcal{J}$ of square zero and $X$ flat over S,
we question if there exists $X'$ flat over $S'$ making the following commutes
\begin{equation}
    \begin{tikzcd}
        X \arrow[r, dashrightarrow] \arrow[d, "\pi"] &
        X' \arrow[d, dashrightarrow] \\
        S \arrow[r, hookrightarrow] &
        S'
    \end{tikzcd}
\end{equation}
such that $X \cong X' \times_{S'} S$.
In fact, we have that
\begin{itemize}
    \item automorphisms are $\ext_{X}^{0}(\mathbb{L}_{X/S}, \pi^{\ast}\mathcal{J})$
    \item extensions are pseudo torsor under $\ext_{X}^{1}(\mathbb{L}_{X/S}, \pi^{\ast}\mathcal{J})$
    \item obstruction lies in $\ext_{X}^{2}(\mathbb{L}_{X/S}, \pi^{\ast}\mathcal{J})$
\end{itemize}
In the following, we would explain these results in the ring case.
\begin{definition}
    Let $0 \rightarrow I \rightarrow A' \rightarrow A \rightarrow 0$ be an exact sequence of $\module_{A'}$.
    We say that it is a 1st order thickening if $I^{2} = 0$. 
\end{definition}
Now, given a diagram
\begin{equation}
    \begin{tikzcd}
        0 \arrow[r] &
        N \arrow[r, dashrightarrow] &
        B' \arrow[r, dashrightarrow] &
        B \arrow[r] &
        0 \\
        0 \arrow[r] &
        I \arrow[u] \arrow[r] &
        A' \arrow[u, dashrightarrow] \arrow[r] &
        A \arrow[u] \arrow[r] &
        0
    \end{tikzcd}
    \label{equation:deformation_problem_of_1st_order_thickening}
\end{equation}
with the last row a 1st order thickening,
where $N$ is a $B$-module and $I \rightarrow N$ is $A$-module homomorphism.
Question if there exists $B'$ fitting into the diagram and making the first row also a 1st order thickening.
\begin{remark}
    If we ask $A \rightarrow B$ to be flat, then by the local criterion of flatness Proposition \ref{Proposition 1.1},
    given a solution $B'$ to this problem, the homomorphism $A' \rightarrow B'$ is flat if and only if $I \otimes_{A} B \cong N$.
    Hence the new problem contains the original deformation problem.
\end{remark}
\begin{lemma}
    Given a commutative diagram with each row a 1st order thickening 
    \begin{equation}
        \begin{tikzcd}[sep = small]
            & 0 \arrow[rr] &&
            N_{2} \arrow[rr] &&
            B_{2}' \arrow[rr] &&
            B_{2} \arrow[rr] &&
            0 \\
            0 \arrow[rr] &&
            N_{1} \arrow[ur] \arrow[rr] &&
            B_{1}' \arrow[ur, dashrightarrow] \arrow[rr] &&
            B_{1} \arrow[ur] \arrow[rr] &&
            0 & \\
            & 0 \arrow[rr] &&
            I_{2} \arrow[uu] \arrow[rr] &&
            A_{2}' \arrow[uu] \arrow[rr] &&
            A_{2} \arrow[uu] \arrow[rr] &&
            0 \\
            0 \arrow[rr] &&
            I_{1} \arrow[uu] \arrow[ur] \arrow[rr] &&
            A_{1}' \arrow[uu] \arrow[ur] \arrow[rr] &&
            A_{1} \arrow[uu] \arrow[ur] \arrow[rr] &&
            0 &
        \end{tikzcd}
    \end{equation}
    question if there exists $B_{1}' \rightarrow B_{2}'$ fitting into the diagram.
    Then we have that
    \begin{itemize}
        \item extensions are pseudo torsor under $\ext_{B_{1}}^{0}(\mathbb{L}_{B_{1}/A_{1}}, N_{2})$
        \item obstruction lies in $\ext_{B_{1}}^{1}(\mathbb{L}_{B_{1}/A_{1}}, N_{2})$
    \end{itemize}
    \label{Lemma 7.8}
\end{lemma}
\begin{proof}
    Given two extensions $\varphi_{1}$ and $\varphi_{2}$, $\varphi_{1} - \varphi_{2}$ gives a map from $B_{1}'$ to $N_{2}$.
    For all $b\in B_{1}$, take $b_{1}'\in B_{1}'$ mapping to $b_{1}$.
    Define $\delta: b_{1} \mapsto \varphi_{1} - \varphi_{2}(b_{1}')$.
    By commutativity, it is clear that $\delta$ is well defined derivation of $B_{1}$ to $N_{2}$ over $A_{1}$.
    Hence extensions are classified by $\hom_{B_{1}}(\Omega_{B_{1}/A_{1}}, N_{2})$, 
    by Lemma \ref{Lemma 6.7}, this is just $\ext_{B_{1}}^{0}(\mathbb{L}_{B_{1}/A_{1}}, N_{2})$.
    \par
    Take $P = A_{1}[B_{1}]$ and $P' = A_{1}'[B_{1}]$.
    Denote kernel of $P \twoheadrightarrow B_{1}$ by $J$.
    Then $\mathbb{L}_{B_{1}/A_{1}}^{\ge -1} = [J/J^{2} \rightarrow \Omega_{P/A_{1}} \otimes_{P} B_{1}]$.
    For each $b_{1}\in B_{1}$, we take a $b_{1}'$ in $B_{1}'$ mapping to $b_{1}$ and so we get a map $P' \rightarrow B_{1}'$.
    Similarly, we can also construct a map $P' \rightarrow B_{2}'$.
    \par
    Denote kernel of $P' \twoheadrightarrow B_{1}$ by $J'$.
    Then we get homomorphisms $J' \rightarrow N_{1}$ and $J' \rightarrow N_{2}$.
    Want to show that $B_{i}'$ is push out of $N_{i}$ along $J' \rightarrow P'$.
    Assume that there is a commutative diagram
    \begin{equation}
        \begin{tikzcd}
            N_{i} \arrow[r, "f"] &
            C \\
            J' \arrow[u] \arrow[r] &
            P' \arrow[u, "g"] 
        \end{tikzcd}
    \end{equation}
    Then for all $b_{i}''\in B_{i}'$, $b_{i}''$ is mapping to some $b_{i}$ in $B_{i}$.
    Hence $b_{i}'' - b_{i}'\in N_{i}$.
    Define $B_{i}' \rightarrow C$ by sending $b_{i}''$ to $f(b_{i}'' - b_{i}') + g([b_{i}])$.
    It is easy to check that this map is well defined and is unique to make the following diagram commutes
    \begin{equation}
        \begin{tikzcd}
            && C \\
            N_{i} \arrow[urr, "f", bend left] \arrow[r] &
            B_{i}' \arrow[ur, dashrightarrow, "\exists!"] & \\
            J' \arrow[u] \arrow[r] &
            P' \arrow[u] \arrow[uur, "g" swap, bend right] &
        \end{tikzcd}
    \end{equation}
    so that $B_{i}'$ is push out.
    By universal property of push out, the obstruction to this extension problem is equivalent to the obstruction to the commutativity of the following diagram
    \begin{equation}
        \begin{tikzcd}
            N_{1} \arrow[r] &
            N_{2} \arrow[r] &
            B_{2}' \\
            J' \arrow[u] \arrow[rr] &&
            P' \arrow[u] 
        \end{tikzcd}
        \label{equation:equivalent_version_of_obstruction_problem_in_lemma_7.8}
    \end{equation}
    Denote $\alpha: J' \rightarrow N_{1}$, $\beta: J' \rightarrow N_{2}$ and $c: N_{1} \rightarrow N_{2}$.
    Now consider map $J/J^{2} \rightarrow N_{2}$ sending $\overline{f} \rightarrow c \circ \alpha(f') - \beta(f')$,
    where $f'$ is a lift of $f$ in $J'$.
    For any two lifts $f_{1}'$ and $f_{2}'$, $f_{1}' - f_{2}'$ would be in $I_{1}[B_{1}]$.
    As $N_{2}$ is of square zero.
    The map is well defined.
    And the diagram \ref{equation:equivalent_version_of_obstruction_problem_in_lemma_7.8} is commutative if and only if the map can be modified to a zero map.
    \par
    Now modify each $b_{i}'$ to $b_{i}''$ and the modified map sends $\overline{f}$ to $c \circ \alpha'(f') - \beta'(f')$.
    Write $f = \sum_{k} a_{k}'[b_{k}]$. 
    Then the difference is $\sum_{k} a_{k}'(c(b_{k1}'' - b_{k1}') + (b_{k2}'' - b_{k1}'))$.
    Hence the modification is given by $J/J^{2} \rightarrow \Omega_{P/A_{1}} \otimes_{P} B_{1} \rightarrow N_{2}$ where the second map sends $[b_{1}] \otimes 1$ to $c(b_{1}'' - b_{1}') + b_{2}'' - b_{2}$.
    Conclude that the obstruction lies in $\ext_{B_{1}}^{1}(\mathbb{L}_{B_{1}/A_{1}}, N_{2})$.
\end{proof}
\begin{lemma}
    Let $\mathscr{A}$ be an abelian category with enough injective objects.
    Assume $A \overset{f}{\rightarrow} B$ is a morphism in $\mathscr{A}$with $B$ projective.
    Denote $[0 \rightarrow A \overset{f}{\rightarrow} B \rightarrow 0]$ by $E$.
    Then for all $C\in \mathscr{A}$, $\hom_{\mathcal{D}(\mathscr{A})}(E, C[1]) \cong \hom_{\mathscr{A}}(A, C)/\hom_{\mathscr{A}}(B, C)$. 
    In particular, $\ext_{B}^{1}(\mathbb{L}_{B/A}, N) = \hom_{B}(J/J^{2}, N)/\hom_{B}(\Omega_{P/A}^{1} \otimes_{P} B, N)$.
    \label{Lemma 7.9}
\end{lemma}
\begin{proof}
    Take injective resolution $I^{\bullet}$ of $C$, then $\hom_{\mathcal{D}(\mathscr{A})}(E, C[1]) = \hom_{\mathcal{D}(\mathscr{A})}(E, I^{\bullet}[1]) = \hom_{\mathcal{K}(\mathscr{A})}(E, I^{\bullet}[1])$.
    For any chain map $\varphi$
    \begin{equation}
        \begin{tikzcd}
            0 \arrow[r] &
            A \arrow[r, "f"] \arrow[d, "\varphi_{-1}" swap] &
            B \arrow[r] \arrow[d, "\varphi_{0}"] \arrow[dl, dashrightarrow, "\psi" swap] &
            0 \arrow[d] & \\
            0 \arrow[r] &
            I^{0} \arrow[r, "g"] &
            I^{1} \arrow[r, "h"] &
            I^{2} \arrow[r] &
            \cdots 
        \end{tikzcd}
    \end{equation}
    as $h \circ \varphi_{0} = 0$, $\varphi_{0}$ factors through $\ker h$.
    Since $I^{0} \twoheadrightarrow \im g = \ker h$ and $B$ is projective, 
    there exists $\psi: B \rightarrow I^{0}$ such that $g \circ \psi = \varphi_{0}$.
    Now for all $a\in A$, $g(\varphi_{-1} - \psi \circ f(a)) = g \circ \varphi_{-1} - \varphi_{0} \circ f(a) = 0$,
    hence $\varphi_{-1} - \psi \circ f$ factors through $\ker g = C$.
    \par
    Define map $s$ sending $\varphi$ to $\overline{\varphi_{-1} - \psi \circ f}$ in $\hom_{\mathscr{A}}(A, C)/\hom_{\mathscr{A}}(B, C)$.
    Clearly, $s$ is independent to choice $\psi$ and homotopy and hence well defined.
    Remains to show that $s$ is bijective.
    For surjectivity, for any $\overline{\pi}$, take lift $\pi: A \rightarrow C$.
    Then obviously $\overline{\pi}$ is the image of the following chain map
    \begin{equation}
        \begin{tikzcd}
            0 \arrow[r] &
            A \arrow[r, "f"] \arrow[d, "\pi" swap] &
            B \arrow[r] \arrow[d, "0"]&
            0 \arrow[d] & \\
            0 \arrow[r] &
            I^{0} \arrow[r, "g"] &
            I^{1} \arrow[r, "h"] &
            I^{2} \arrow[r] &
            \cdots 
        \end{tikzcd}
    \end{equation}
    so that $s$ is surjective.
    \par
    For injectivity, assume that $\varphi$ is mapping to $\overline{0}$.
    Then $\varphi_{-1} - \psi \circ f = \phi \circ f$ for some $\phi: B \rightarrow C$.
    Consider $\phi$ as a map from $B$ to $I^{0}$, then $\varphi_{-1} = (\psi + \phi) \circ f$.
    Take chain homotopy $\psi + \phi$ so that $\varphi$ is null-homotopic, done!
\end{proof}
\begin{lemma}
    Solutions to original problem \ref{equation:deformation_problem_of_1st_order_thickening} is pseudo torsor under $\ext_{B}^{1}(\mathbb{L}_{B/A}, N)$.
    \label{Lemma 7.10}
\end{lemma}
\begin{proof}
    Given two solutions $B_{1}'$ and $B_{2}'$, we have a commutative diagram
    \begin{equation}
        \begin{tikzcd}[sep = small]
            & 0 \arrow[rr] &&
            N \arrow[rr] &&
            B_{2}' \arrow[rr] &&
            B \arrow[rr] &&
            0 \\
            0 \arrow[rr] &&
            N \arrow[ur, equal] \arrow[rr] &&
            B_{1}' \arrow[ur, dashrightarrow] \arrow[rr] &&
            B \arrow[ur, equal] \arrow[rr] &&
            0 & \\
            & 0 \arrow[rr] &&
            I \arrow[uu] \arrow[rr] &&
            A' \arrow[uu] \arrow[rr] &&
            A \arrow[uu] \arrow[rr] &&
            0 \\
            0 \arrow[rr] &&
            I \arrow[uu] \arrow[ur, equal] \arrow[rr] &&
            A' \arrow[uu] \arrow[ur, equal] \arrow[rr] &&
            A \arrow[uu] \arrow[ur, equal] \arrow[rr] &&
            0 &
        \end{tikzcd}
    \end{equation}
    Hence by Lemma \ref{Lemma 7.8}, the obstruction to existence of $B_{1}' \rightarrow B_{2}'$ fitting into the diagram lies in $\ext_{B}^{1}(\mathbb{L}_{B/A}, N)$.
    Thus $B_{1}' \cong B_{2}'$ if and only if the obstruction vanishes.
    \par
    On the other hand, given $\delta\in \ext_{B}^{1}(\mathbb{L}_{B/A}, N)$.
    Note that by Lemma \ref{Lemma 7.9}, we can lift $\delta$ to some $\varphi\in \hom_{B}(J/J^{2}, N)$.
    Take $P = A'[B]$, given a commutative diagram
    \begin{equation}
        \begin{tikzcd}
            o \arrow[r] &
            J \arrow[r] \arrow[d, "\theta"] &
            P \arrow[r] \arrow[d] &
            B \arrow[r] \arrow[d, equal] &
            0 \\
            0 \arrow[r] &
            N \arrow[r] &
            B_{1}' \arrow[r] &
            B \arrow[r] &
            0 \\
            0 \arrow[r] &
            I \arrow[u] \arrow[r] &
            A' \arrow[u] \arrow[r] &
            A \arrow[u] \arrow[r] &
            0
        \end{tikzcd}
    \end{equation}
    $\theta + \varphi$ would give a pushout $B_{2}'$ as following
    \begin{equation}
        \begin{tikzcd}
            J \arrow[r] \arrow[d, "\theta + \varphi"] &
            P \arrow[d] \\
            N \arrow[r] &
            B_{2}'
        \end{tikzcd}
    \end{equation}
    Explicitly, $B_{2}' = \{(n, p)\}/(\theta + \varphi(j) - j)$ and the multiplication is defined by
    \begin{equation}
        (n_{1}, p_{1}) \cdot (n_{2}, p_{2}) = (n_{1}p_{1} + n_{2}p_{2}, p_{1}p_{2})
    \end{equation}
    where $np$ is well defined since $N$ is a $B$-module.
    Conclude that solutions are classified by $\ext_{B}^{1}(\mathbb{L}_{B/A}, N)$.
\end{proof}
For trivial case
\begin{equation}
    \begin{tikzcd}
        0 \arrow[r] &
        N \arrow[r, dashrightarrow] &
        B' \arrow[r, dashrightarrow] &
        B \arrow[r] &
        0 \\
        0 \arrow[r] &
        0 \arrow[u] \arrow[r] &
        A \arrow[u, dashrightarrow] \arrow[r, "\identity_{A}"] &
        A \arrow[u] \arrow[r] &
        0
    \end{tikzcd}
\end{equation}
solutions are just $\ext_{B}^{1}(\mathbb{L}_{B/A}, N)$.
In particular, $0\in \ext_{B}^{1}(\mathbb{L}_{B/A}, N)$ corresponds to $N \oplus B$ with ring structure similar to $B_{2}'$ above.
Consider
\begin{equation}
    \begin{tikzcd}
        0 \arrow[r] &
        I \arrow[r] &
        A' \arrow[r] &
        A \arrow[r] &
        0 \\
        0 \arrow[r] &
        0 \arrow[u] \arrow[r] &
        A' \arrow[u, equal] \arrow[r, "\identity_{A'}"] &
        A' \arrow[u] \arrow[r] &
        0
    \end{tikzcd}
\end{equation}
corresponding to some $\xi\in \ext_{A}^{\mathbb{L}_{A/A'}, I}$.
Taking push out of $N$ along $I \rightarrow A'$ and pull back of $A$ along $B' \rightarrow$, 
we get a commutative diagram
\begin{equation}
    \begin{tikzcd}
        0 \arrow[r] &
        I \arrow[r] \arrow[d] &
        A' \arrow[r] \arrow[d] &
        A \arrow[r] \arrow[d, equal] &
        0 \arrow[r, rightsquigarrow] &
        \xi\in \ext_{A}^{1}(\mathbb{L}_{A/A'}, I) \arrow[d, mapsto] \\
        0 \arrow[r] &
        N \arrow[r] \arrow[d, equal] &
        A'' \arrow[r] \arrow[d, equal, "?" description] &
        A \arrow[r] \arrow[d, equal] &
        0 \arrow[r, rightsquigarrow] &
        \xi''\in \ext_{A}^{1}(\mathbb{L}_{A/A'}, N) \arrow[d, equal, "?" description] \\
        0 \arrow[r] &
        N \arrow[r] \arrow[d, equal] &
        B'' \arrow[r] \arrow[d] &
        A \arrow[r] \arrow[d] &
        0 \arrow[r, rightsquigarrow] &
        \xi'''\in \ext_{A}^{1}(\mathbb{L}_{A/A'}, N) \\
        0 \arrow[r] &
        N \arrow[r] &
        B' \arrow[r] &
        B \arrow[r] &
        0 \arrow[r, rightsquigarrow] &
        \xi'\in \ext_{B}^{1}(\mathbb{L}_{B/A'}, N) \arrow[u, mapsto]
    \end{tikzcd}
\end{equation}
Hence by 5 Lemma and universal properties of push out and pull back, 
$B'$ is a solution to original problem if and only if $A'' = B''$ if and only if $\xi'' = \xi'''$.
If denote $\ext_{B}^{1}(\mathbb{L}_{B/A'}, N) \rightarrow \ext_{A}^{1}(\mathbb{L}_{A/A'}, N)$ by $\rho$,
then 
\begin{itemize}
    \item solutions are $\rho^{-1}(\xi'')$ a pseudo torsor under $\ker \rho$.
    \item obstruction class is $[\xi'']\in \coker \rho$.
\end{itemize}
For commutative diagram
\begin{equation}
    \begin{tikzcd}
        B &
        A \arrow[l] \\
        & A' \arrow[ul] \arrow[u]
    \end{tikzcd}
\end{equation}
by Proposition \ref{Proposition 7.1}, there is an exact triangle
\begin{equation}
    \mathbb{L}_{A/A'} \otimes_{A} B \longrightarrow \mathbb{L}_{B/A'} \longrightarrow \mathbb{L}_{B/A} \longrightarrow \mathbb{L}_{A/A'}[1] \otimes_{A} B
\end{equation}
Applying $\hom_{\mathcal{D}(\module_{B})}(\cdot, N)$, by Lemma \ref{Lemma 6.3}, there is a long exact sequence
\begin{equation}
    \cdots \rightarrow \ext_{B}^{1}(\mathbb{L}_{B/A}, N) \rightarrow \ext_{B}^{1}(\mathbb{L}_{B/A'}, N) \overset{\rho}{\rightarrow} \ext_{B}^{1}(\mathbb{L}_{A/A'} \otimes_{A} B, N) \rightarrow \ext_{B}^{2}(\mathbb{L}_{B/A}, N)
\end{equation}
Note that $\ext_{B}^{1}(\mathbb{L}_{A/A'} \otimes_{A} B, N) = \ext_{A}^{1}(\mathbb{L}_{A/A'}, N)$, we can consider the image of $\xi''$ in $\ext_{B}^{2}(\mathbb{L}_{B/A}, N)$, denoted $\zeta$.
As the sequence is exact, there exists some $\xi'$ mapping to $\xi''$ if and only if $\zeta = 0$.
Hence the obstruction class is just $\zeta$.
In addition, $\ker \rho = \im (\ext_{B}^{1}(\mathbb{L}_{B/A}, N) \rightarrow \ext_{B}^{1}(\mathbb{L}_{B/A'}, N))$ so that extensions are classified by $\ext_{B}^{1}(\mathbb{L}_{B/A}, N)$.

\subsection{Deformation of sheaf}
\label{subsection:Cotangent_Complex_Deformation_of_sheaf}

Let $X \hookrightarrow X'$ with ideal sheaf $\mathcal{I}$ be a 1st order thickening.
Given $\mathcal{F},\mathcal{G}$ sheaves of $\mathcal{O}_{X}$-modules on $X$ and map $\mathcal{I} \otimes_{\mathcal{O}_{X}} \mathcal{F} \overset{c}{\mathcal{G}}$,
we question if there exists $\mathcal{O}_{X'}$-module sheaf $\mathcal{F}'$ fitting into 
\begin{equation}
    0 \longrightarrow \mathcal{G} \longrightarrow \mathcal{F}' \longrightarrow \mathcal{F} \longrightarrow 0
\end{equation}  
such that $\mathcal{F}\big{|}_{X} = \mathcal{F}$ and induced map $\mathcal{I} \otimes_{\mathcal{O}_{X}} \mathcal{F} \rightarrow \mathcal{G}$ is the given map $c$.
In fact, we have that
\begin{itemize}
    \item automorphisms are $\ext_{X}^{0}(\mathcal{F}, \mathcal{G})$.
    \item extensions are pseudo torsor under $\ext_{X}^{1}(\mathcal{F}, \mathcal{G})$.
    \item obstruction lies in $\ext_{X}^{2}(\mathcal{F}, \mathcal{G})$.
\end{itemize}

\subsection{Construction of cotangent complex}
\label{subsection:Cotangent_Complex_Construction_of_cotangent_complex}

For ring homomorphism $A \rightarrow B$, consider simplicial resolution
\begin{equation}
    \begin{tikzcd}
        \cdots \arrow[r] &
        A^{3}[B] \arrow[r, shift left = 2ex] \arrow[r] \arrow[r, shift right = 2ex] &
        A^{2}[B] \arrow[r, shift left = 1ex] \arrow[r, shift right = 1ex] \arrow[l, shift left = 1ex, shorten = 2mm] \arrow[l, shift right = 1ex, shorten = 2mm] &
        A[B] \arrow[r, "\varepsilon"] \arrow[l, shorten = 2mm] &
        B \arrow[r] &
        0
    \end{tikzcd}
\end{equation}
Those parallel arrows are given by different way of adding or removing brakets.
For example, the two maps from $A^{2}[A]$ to $A[B]$ is respectively given by $[[b_{1}] + [b_{2}]] \mapsto [b_{1}] + [b_{2}]$ and $[[b_{1}] + [b_{2}]] \mapsto [b_{1} + b_{2}]$.
Apply $\Omega_{\cdot/A} \otimes_{\cdot} B$, we get that
\begin{equation*}
    \begin{tikzcd}
        \cdots \arrow[r] &
        \Omega_{A^{3}[B]/A} \otimes_{A^{3}[B]} B \arrow[r, shift left = 2ex] \arrow[r] \arrow[r, shift right = 2ex] &
        \Omega_{A^{2}[B]/A} \otimes_{A^{2}[B]} B \arrow[r, shift left = 1ex] \arrow[r, shift right = 1ex] \arrow[l, shift left = 1ex, shorten = 2mm] \arrow[l, shift right = 1ex, shorten = 2mm] &
        \Omega_{A[B]/A} \otimes_{A[B]} B \arrow[r, "\varepsilon"] \arrow[l, shorten = 2mm] &
        \Omega_{B/A}
    \end{tikzcd}
\end{equation*}
Taking alternating sums of such maps, we get that
\begin{equation}
    \cdots \longrightarrow \Omega_{A^{3}[B]/A} \otimes_{A^{3}[B]} B \overset{\partial_{0} - \partial_{1} + \partial_{2}}{\longrightarrow} \Omega_{A^{2}[B]/A} \otimes_{A^{2}[B]} B \overset{\partial_{0} - \partial_{1}}{\longrightarrow} \Omega_{A[B]/A} \otimes_{A[B]} B \overset{\partial}{\longrightarrow} \Omega_{B/A}
\end{equation}
And $[\cdots \rightarrow \Omega_{A^{3}[B]/A} \otimes_{A^{3}[B]} B \overset{\partial_{0} - \partial_{1} + \partial_{2}}{\rightarrow} \Omega_{A^{2}[B]/A} \otimes_{A^{2}[B]} B \overset{\partial_{0} - \partial_{1}}{\rightarrow} \Omega_{A[B]/A} \otimes_{A[B]} B]$ is our desired cotangent complex.
\end{document}